%\gls{}         normal zu nutzen (erstes Mal: 'lange Form (kurze Form)'), danach nur 'kurze Form'
%\glspl{}       wie \gls{} nur als Plural
%\acrfull{qrc}  gibt volle Form ('lange Form (kurze Form)') egal wo
%\acrlong{qrc}  gibt lange Form ('lange Form') egal wo
%
%\newacronym{tag}{short}{long}
\newacronym{cff}{CFF}{Citation File Format}
\newacronym{pep}{PEP}{Python Enhancement Proposal}
\newacronym{pypi}{PyPI}{Python Package Index}
\newacronym{cran}{CRAN}{Comprehensive R Archive Network}
\newacronym{ner}{NER}{Named entity recognition}
\newacronym{ned}{NED}{Named entity disambiguation}
\newacronym{oss}{OSS}{Open-Source-Software}
\newacronym[firstplural=Digitalen Objektbezeichner (DOI)]{doi}{DOI}{Digitaler Objektbezeichner}
