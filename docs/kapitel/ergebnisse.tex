\chapter{Ergebnisse}
\label{chap:ergebnisse}
% TODO ggf. noch weitere full cff graphen hinzufügen
% TODO Darauf achten, dass jede Grafik mindestens einmal im Text referenziert wird
% TODO Wichtig ist, dass du deine Forschungsergebnisse im Ergebnisteil noch nicht interpretierst, sondern nur beschreibst.!!!
% TODO gff. noch auf die Daten in #20 eingehen? Das dann mit und ohne Zeitverlauf?
% TODO bei cff sagen, dass von 100 top repos x pypi x cran und x andere sind
% TODO Es werden alle Namen ausgegeben in den Beschreibungen aber auch eben solche, die gar nichts mit dem Paket zu tun haben wie im fall von highr für CRAN: "Provides syntax highlighting for R source Code. Currently it supports LaTeX and HTML output. Source Code of other languages is supported via Andre Simon's highlight package (https://gitlab.com/saalen/highlight)." Es gibt noch weitere Beispiele z. B. in CRAN magrittr -> vllt eher ergebnis?
In diesem Kapitel werden die Ergebnisse der Masterarbeit präsentiert.
Das Kapitel ist in zwei Abschnitte unterteilt.
Zum einen werden die Ergebnisse der Gegenwart in \autoref{sec:neuste_ergebnisse} präsentiert.
Hierbei werden die jeweils neusten Dateien aus der Datenbeschaffung untersucht und Statistiken aufgeführt, welche in Verbindung mit diesen Daten stehen.
Zum anderen werden die Ergebnisse mit Zeitverlauf in \autoref{sec:gesamtheit_ergebnisse} präsentiert.
Hierbei werden die Dateien aus der Datenbeschaffung in allen Versionen analysiert.
Durch die Betrachtung aller Versionen können Veränderungen über die Zeit aufgezeigt werden und somit eine Aussage über die Entwicklung der Zitationen getroffen werden.
In dem \autoref{sec:neuste_ergebnisse} wird zu Beginn auf die Statistiken des Abgleichs eingegangen und präsentiert, wie genau der Abgleich durchgeführt wurde.
Der Abgleich hat einen direkten Einfluss auf viele weitere Statistiken.

Die benötigten Daten für die Untersuchungen wurden in der Zeitspanne September 2024 bis November 2024 beschafft.
Dies liegt daran, dass die Repositorys lokal gespeichert werden und bei der Entwicklung der Arbeit die Daten nicht erneut beschafft werden mussten.
Die Daten einer Liste wurden jeweils an einem Tag beschafft, um eine Vergleichbarkeit zu gewährleisten.
In dem gesamten Kapitel wird \emph{matplotlib} in der Version 3.9.2 verwendet, um die Grafiken zu erstellen \autocite{hunter_matplotlib_2007}.

\section{Ergebnisse der Gegenwart}
\label{sec:neuste_ergebnisse}
Da die Ergebnisse der Masterarbeit direkt abhängig von der Qualität des Abgleichs sind, wurde diese untersucht.
Dies wurde auf zwei Arten durchgeführt.
Zum einen wurde automatisch die Anzahl der abgeglichenen Autoren und der nicht abgeglichenen Autoren untersucht.
Zum anderen wurden manuell für eine ausgewählte Menge an Personen die Ergebnisse händisch überprüft.
Es wurde in beiden Fällen die Überprüfung ausschließlich auf den neusten Versionen der Daten durchgeführt.

Die automatische Überprüfung erfolgt bei der Auswertung der Daten aus der Datenbeschaffung.
Dabei wird für jede Liste und für jede Quelle z. B. \gls{cff} die Anzahl der abgeglichenen Autoren und der gesamt vorhandenen Autoren ermittelt.
Hierbei wurde die gesamte Menge an Autoren betrachtet.
Die Ergebnisse für die Listen \gls{pypi} \gls{cff} und \gls{cran} \gls{cff} sind in \autoref{tab:matching_results_auto} dargestellt.
Für die weiteren Listen sind die Ergebnisse in \autoref{tab:matching_results_auto_anhang} zu finden.

\begin{table}
    \centering
    \setlength{\tabcolsep}{8pt}
    \begin{tabular}{c|c|c}
        \toprule
        \textbf{Quelle} & \textbf{\gls{pypi} \gls{cff}} & \textbf{\gls{cran} \gls{cff}} \\ \midrule
        \emph{\gls{cran} Autoren} & & 204/238 (85.71 \%) \\
        \emph{\gls{cran} Maintainer} & & 99/100 (99.00 \%) \\
        \emph{Beschreibung} & 580/1070 (54.21 \%) & 21/80 (26.25 \%) \\
        \emph{README} & 868/1396 (62.18 \%) & 143/1008 (14.19 \%) \\
        \emph{\gls{cff}} & 477/598 (79.77 \%) & 184/233 (78.97 \%) \\
        \emph{\gls{cff} preferred citation} & 294/380 (77.37 \%) & 109/136 (80.15 \%) \\
        \emph{\gls{pypi} Maintainer} & 206/228 (90.35 \%) & \\
        \emph{Python Autoren} & 105/131 (80.15 \%) & \\
        \emph{Python Maintainer} & 15/25 (60.00 \%) & \\
        \emph{\hologo{BibTeX}} & 0/1 (0.00 \%) & \\ \midrule
        \emph{Summe} & 2545/3829 (66.47 \%) & 760/1795 (42.34 \%) \\
        \bottomrule
    \end{tabular}
    \caption{Automatische Ergebnisse des Abgleichs}
    \label{tab:matching_results_auto}
    \small
    \raggedright
    In der Tabelle sind die Ergebnisse des automatischen Abgleichs für die Listen \gls{pypi} \gls{cff} und \gls{cran} \gls{cff} dargestellt. Die Werte geben an wie viele der genannten Autoren in den Git Repositorys gefunden wurden.
\end{table}

In den Ergebnissen ist auffällig, dass die Autoren in der README Datei und in der Beschreibung der Pakete am schlechtesten abgeglichen werden konnten.
Dies liegt daran, dass hierbei die \gls{ner} verwendet wurde und diese nicht immer korrekt arbeitet.
Hierbei kommt es häufig vor, dass Entitäten erkannt werden, welche keine Person darstellen.
Außerdem werden Personen mit weiteren Zusätzen wie z. B. dem Anfang einer Internetadresse zurückgegeben, was den Abgleich erschwert.
Zusätzlich werden in der README und der Beschreibung häufig Personen genannt, welche z. B. vorarbeit für die Software geleistet haben, aber in der Software selbst keinen Code beigetragen haben.
Diese Personen können nicht abgeglichen werden.
Die Python Maintainer konnten ebenfalls schlecht abgeglichen werden, da viele Pakete in diesem Feld keine Personen sondern Organisationen nennen.

Allgemein bietet dieses Vorgehen keine Aussage darüber, ob die Personen nicht abgeglichen worden sind, weil sie keinen Code beigetragen haben oder weil sie nicht gefunden wurden.
Außerdem werden Personen, welche fälschlicherweise mit einer anderen Person abgeglichen wurden in diesem Verfahren als gut bewertet.
Um diese Fälle ebenfalls betrachten zu können, wurde manuell eine Auswahl an Autoren überprüft.
Es wurden nicht alle Autoren überprüft, da dies den zeitlichen Rahmen der Arbeit gesprengt hätte.
Aus diesem Grund wurden die Autoren für jedes Paket und jeder Quelle zufällig neu angeordnet und gespeichert.
Anschließend wurden jeweils die ersten beiden Autoren anhand von vier Kriterien überprüft.
Falls nur ein Autor in einer Quelle genannt wurde, wurde nur dieser überprüft.
Jeder Autor wurde einer der vier Kategorien \glqq Richtig Positiv (TP)\grqq{}, \glqq Falsch Negativ (FN)\grqq{}, \glqq Falsch Positiv (FP)\grqq{} oder \glqq Richtig Negativ (TN)\grqq{} zugeordnet.
Aus diesen Daten wurde der F1-Score berechnet.

Außerdem wurde angegeben, ob es sie um keine Person handelt, sondern z. B. um eine Organisation.
In diesen Fällen wurde der Eintrag dennoch in eine der vier Kategorien eingeteilt, wobei primär die Kategorie FP verwendet wurde, falls es eine Zuordnung gab.
Falls die Zuordnung allerdings mit einem Git Autoren erfolgt ist, welcher beispielsweise ein Bot der Organisation ist wurde der Eintrag als TP eingeteilt, da es keinen Mechanismus in der Datenbeschaffung gibt, welcher Personen von nicht Personen unterscheiden kann.
Es wurde lediglich in der Datenbeschaffung darauf geachtet möglichst keine Quellen zu berücksichtigen, welche keine Personen enthalten.
Die Ergebnisse für die \gls{cff} Liste sind in \autoref{tab:cff_matching_results_manual} dargestellt.
% TODO weitere Tabellen im Anhang referenzieren
% Die weiteren Ergebnisse für die anderen Listen sind in den Tabellen zu finden.
% TODO confusion matrix und f1 score für alle listen zusammen (addiert) angeben

\begin{table}
    \centering
    \setlength{\tabcolsep}{8pt}
    \begin{tabular}{c|c|c|c|c|c|c}
        \toprule
        \textbf{Quelle} & \textbf{TP} & \textbf{FN} & \textbf{FP} & \textbf{TN} & \textbf{Keine Person} & \textbf{F1-Score} \\ \midrule
        Beschreibung                 & 12 & 1  & 6  & 26 & 2  & 0.7742 \\
        README                       & 13 & 1  & 4  & 17 & 5  & 0.8387 \\
        \gls{cff}                    & 26 & 2  & 0  & 4  & 0  & 0.9630 \\
        \gls{cff} preferred citation & 7  & 1  & 1  & 4  & 0  & 0.8750 \\
        \gls{pypi} Maintainer        & 61 & 0  & 9  & 5  & 7  & 0.9313 \\
        Python Autoren               & 29 & 0  & 13 & 6  & 17 & 0.8169 \\
        Python Maintainer            & 3  & 0  & 2  & 2  & 3  & 0.7500 \\
        \hologo{BibTeX}              & 0  & 1  & 0  & 0  & 0  & 0.0000 \\ \midrule
        Summe                        & 151 & 6 & 35 & 64 & 34 & 0.8805 \\
        \bottomrule
    \end{tabular}
    \caption{Manuelle Ergebnisse des Abgleichs für die \gls{cff} Liste}
    \label{tab:cff_matching_results_manual}
    \small
    \raggedright
    In der Tabelle sind die Ergebnisse des manuellen Abgleichs für die \gls{cff} Liste dargestellt. Dabei wurde für jede Quelle die Anzahl der Richtig Positiven (TP), Falsch Negativen (FN), Falsch Positiven (FP) und Richtig Negativen (TN) Autoren ermittelt. Zusätzlich wurde angegeben, ob es sich um keine Person handelt und der F1-Score berechnet.
\end{table}

Aus den gesammelten Daten wurden weitere Statistiken berechnet.
Dabei muss berücksichtigt werden, dass einige Statistiken auf dem Abgleich basieren und somit nur so gut sind, wie der Abgleich durchgeführt wurde.
Außerdem basieren einige Statistiken auf den Metriken der Commits und geänderten Zeilen.
Die Abbildung \autoref{fig:commits_vs_changed_lines} stellt das Verhältnis der Commits und der geänderten Zeilen für die Autoren in den Paketen der Listen \gls{pypi} \gls{cff} und \gls{cran} \gls{cff} dar.
Für die anderen Listen sind die Ergebnisse in \autoref{fig:commits_vs_changed_lines_anhang} dargestellt.
Ein Punkt stellt dabei einen Autor dar.
Es ist zu erkennen, dass einige Autoren nur einen Commit getätigt haben und dabei viele Zeilen geändert haben.
Beispielsweise hat der Autor François Lagunas in dem Paket \emph{huggingface/datasets} lediglich einen Commit getätigt, aber dabei ungefähr 29.000 Zeilen hinzugefügt.
Inhalt sind in diesem konkreten Fall automatisch generierte README Dateien für Datensätze, welche zuvor gefehlt haben.

\begin{figure}
    \begin{subfigure}{.5\textwidth}
        \centering
        \includesvg[width=.95\linewidth,inkscapelatex=false]{bilder/commits_vs_changed_lines/commits_vs_changed_lines_pypi_cff.svg}
        \caption{\gls{pypi} \gls{cff}}
        \label{fig:commits_vs_changed_lines_pypi_cff}
    \end{subfigure}%
    \begin{subfigure}{.5\textwidth}
        \centering
        \includesvg[width=.95\linewidth,inkscapelatex=false]{bilder/commits_vs_changed_lines/commits_vs_changed_lines_cran_cff.svg}
        \caption{\gls{cran} \gls{cff}}
        \label{fig:commits_vs_changed_lines_cran_cff}
    \end{subfigure}
    \caption{Commits und geänderte Zeilen gegenübergestellt}
    \label{fig:commits_vs_changed_lines}
    \small
    \raggedright
    Die Abbildungen zeigen für die Autoren in den Paketen in den Listen \gls{pypi} \gls{cff} und \gls{cran} \gls{cff} die Anzahl der Commits gegenüber der Anzahl der geänderten Zeilen. Die x Achse stellt die Anzahl der Commits dar und die y Achse die Anzahl der geänderten Zeilen. Es wird eine logarithmische Skalierung verwendet.
\end{figure}

Die \autoref{fig:common_authors} zeigt den Anteil der Top Git Autoren in den einzelnen Quellen wie beispielsweise der \gls{cff}.
Dabei werden mehrere Abbildungen dargestellt für die unterschiedlichen untersuchten Listen.
In der konkreten Abbildung sind die Listen \gls{pypi} \gls{cff} und \gls{cran} \gls{cff} dargestellt.
Die Abbildungen für die weiteren Listen sind in \autoref{fig:common_authors_anhang} dargestellt.
Außerdem werden für jede Liste zwei unterschiedliche Abbildungen dargestellt.
Die eine Abbildung bemisst die Top Git Autoren an der Anzahl der Commits und die jeweils andere an der Anzahl der geänderten Zeilen.
Falls beispielsweise der Autor mit den meisten Commits in einer Quelle z. B. \gls{cff} genannt wird und die Autoren an den Commits bemessen werden und zusätzlich nur der Top Autor betrachtet wird, so wird in der Abbildung bei $x=1$ dargestellt, dass 100 \% der Git Autoren in der \gls{cff} genannt werden.

Bei der Betrachtung von immer mehr Git Autoren sinkt der Anteil der genannten Autoren in den Quellen.
Beispielsweise werden in \autoref{fig:common_authors_pypi_cff} von 100 betrachteten Top Git Autoren gemessen nach Commits nahezu 0 \% als Python Maintainer genannt, jedoch werden knapp 10 \% in der \gls{cff} preferred citation angegeben.
Dabei muss berücksichtigt werden, dass unterschiedliche Quellen unterschiedlich häufig vorkommen.
Beispielsweise haben weniger Pakete \gls{cff} preferred citation Autoren angegeben als \gls{cff} Autoren, da diese zwingend angegeben werden müssen.
Dadurch kann es dazu kommen, dass nur ein Paket eine bestimmte Quelle angegeben hat und wenn dieses Paket in der Quelle viele Autoren nennt, ist der Anteil der genannten Autoren in der gesamten Quelle hoch.

Im hintergrund der Abbildungen werden weitere Linien präsentiert, welche nicht in der Legende aufgeführt sind.
Bei jeder Linie handelt es sich um eine einzige Quelle eines einzigen Pakets.
Ein Paket wird somit mit mehreren Linien dargestellt.
Auffällig ist dabei, dass einige Linien erst bei einem Wert von $x>1$ beginnen.
Dies ist der Fall, falls ein Paket in einer Quelle den Top Autor nicht nennt.
Beispielsweise gibt das Paket \emph{faker} in der \gls{cff} als einzigen Autor \glqq Daniele Faraglia\grqq{} an.
Dieser Autor ist jedoch auf Rang 37 der Top Git Autoren gemessen nach Commits.

Eine weitere Auffälligkeit, welche beispielsweise in der \autoref{fig:common_authors_pypi_cff} vorkommt ist, dass in der Legende \hologo{BibTeX} aufgeführt ist, jedoch keine Linie in der Abbildung dargestellt wird.
Dies liegt daran, dass keiner der Top 100 Git Autoren aller Pakete mit \hologo{BibTeX} in einer \hologo{BibTeX} als Autor genannt wird.
Im konkreten Fall liegt das daran, dass nur eines der Pakete eine \hologo{BibTeX} Datei enthält und in dieser Datei nur ein Autor genannt wird, welcher nicht automatisch abgeglichen wurde.
Dies ist \autoref{tab:pypi_matching_results_manual} ebenfalls zu entnehmen.
Dadurch liegt die Linie über alle x Werte bei 0 \% und ist nicht sichtbar.

Die letzte Auffälligkeit in den Graphen ist, dass einige Linien erst fallen und anschließend parallel zur x Achse verlaufen.
Andere wiederum fallen gar nicht und verlaufen direkt parallel zur x Achse.
Dies ist der Fall, falls ab einer bestimmten Anzahl an Git Autoren alle weiteren Git Autoren genannt werden.
Dies tritt besonders dann auf, falls es wenig Git Autoren gibt.
Beispielsweise hat das Paket \emph{netron}, welches in der \gls{pypi} \gls{cff} Liste vorkommt, nur einen Git Autoren.
Dieser wird ebenfalls beispielsweise als \gls{pypi} Maintainer genannt.
Dadurch sind alle Git Autoren zu 100 \% als \gls{pypi} Maintainer genannt und es kommt zu der Linie bei 100 \% in \autoref{fig:common_authors_pypi_cff}.

\begin{figure}
    \begin{subfigure}{.5\textwidth}
        \centering
        \includesvg[width=.95\linewidth,inkscapelatex=false]{bilder/common_authors/1_pypi_cff.svg}
        \caption{\gls{pypi} \gls{cff} nach Commits}
        \label{fig:common_authors_pypi_cff}
    \end{subfigure}%
    \begin{subfigure}{.5\textwidth}
        \centering
        \includesvg[width=.95\linewidth,inkscapelatex=false]{bilder/common_authors_by_lines/1_pypi_cff_by_lines.svg}
        \caption{\gls{pypi} \gls{cff} nach geänderten Zeilen}
        \label{fig:common_authors_by_lines_pypi_cff}
    \end{subfigure}
    \begin{subfigure}{.5\textwidth}
        \centering
        \includesvg[width=.95\linewidth,inkscapelatex=false]{bilder/common_authors/1_cran_cff.svg}
        \caption{\gls{cran} \gls{cff} nach Commits}
        \label{fig:common_authors_cran_cff}
    \end{subfigure}%
    \begin{subfigure}{.5\textwidth}
        \centering
        \includesvg[width=.95\linewidth,inkscapelatex=false]{bilder/common_authors_by_lines/1_cran_cff_by_lines.svg}
        \caption{\gls{cran} \gls{cff} nach geänderten Zeilen}
        \label{fig:common_authors_by_lines_cran_cff}
    \end{subfigure}
    \caption{Anteil der Top Git Autoren an den genannten Autoren}
    \label{fig:common_authors}
    \small
    \raggedright
    Die Abbildungen zeigen für die Pakete in den Listen \gls{pypi} \gls{cff} und \gls{cran} \gls{cff} den Anteil der Top Git Autoren in der Zitation. Die Linien stellen die unterschiedlichen Quellen dar. Auf der x Achse wird die Anzahl der betrachteten Git Autoren für \autoref{fig:common_authors_pypi_cff} und \autoref{fig:common_authors_cran_cff} sortiert nach der Anzahl der Commits und für \autoref{fig:common_authors_by_lines_pypi_cff} und \autoref{fig:common_authors_by_lines_cran_cff} sortiert nach der Anzahl der geänderten Zeilen angegeben. Die y Achse stellt jeweils den Anteil der genannten Autoren an den Git Autoren dar. Für die blaue Linie in \autoref{fig:common_authors_pypi_cff} gilt: Die Top x Commiter sind zu y \% in der \gls{cff} gelistet.
\end{figure}

Weitere Abbildungen, welche den Anteil zwischen genannten Autoren und den Top Git Autoren darstellen sind in \autoref{fig:common_authors_2} abgebildet.
Die Abbildung stellt die Graphen für die Listen \gls{pypi} \gls{cff} und \gls{cran} \gls{cff} dar.
Die Ergebnisse für die weiteren Listen sind in \autoref{fig:common_authors_2_anhang} dargestellt.
In den Abbildungen wurden jeweils wieder zwei Graphen für jede Liste erstellt.
Zum einen wurden die Autoren nach der Anzahl der Commits und zum anderen nach der Anzahl der geänderten Zeilen sortiert.
In allen Abbildungen wird der Anteil der genannten Autoren unter den Top Git Autoren dargestellt.
Durch dieses Verhältnis steigen die Linien, da mehr genannte Autoren unter den Top Git Autoren bei steigender Anzahl derer sind.

Die Abbildungen zeigen beispielsweise für $x=0$ immer, dass 0 \% der genannten Autoren unter den Top 0 Git Autoren sind.
Der Anteil kann dabei 100 \% erreichen, falls alle genannten Autoren in einer Quelle auch tatsächlich unter den Top 200 Git Autoren sind und somit Quellcode entwickelt haben.
Dabei muss berücksichtigt werden, dass nicht abgeglichene Autoren ebenfalls berücksichtigt werden.
Falls ein Autor nicht abgeglichen wurde, wird dieser als nicht unter den Top Git Autoren betrachtet und der Graph kann keine 100 \% für die Quelle erreichen.

Um konkrete Werte zu nennen ist in \autoref{fig:common_authors_2_pypi_cff} beispielsweise dargestellt, dass ungefähr 80 \% der als \gls{pypi} Maintainer genannten Autoren unter den Top 100 Git Autoren nach der Anzahl der Commits sind und nur ungefähr 40 \% der in der README genannten Autoren unter den Top 100 Git Autoren nach Commits sind.
Die Linien im Hintergrund stellen erneut ein einzelnes Paket einer einzelnen Quelle dar, sodass ein Paket mit mehreren Linien für alle Quellen dargestellt wird.

In den Graphen sind erneut Auffälligkeiten zu erkennen.
Beispielsweise existieren erneut Linien, welche für jeden x Wert 0 \% oder 100 \% darstellen.
In \autoref{fig:common_authors_2_pypi_cff} ist beispielsweise erneut die Linie für \hologo{BibTeX} nicht zu erkennen, da diese erneut auf der x Achse verläuft.
Dies liegt daran, dass nur ein Autor in einer \hologo{BibTeX} Datei über alle Pakete genannt wird, welcher nicht abgeglichen werden konnte und somit nicht unter den Top Git Autoren ist.
Für die Linien, welche bei 100 \% verlaufen, handelt es sich um Pakete, bei denen beispielsweise nur ein Autor in der Quelle genannt wird und dieser Autor auf Rang eins der Top Git Autoren ist.
Es kann aber auch komplexer sein beispielsweise, dass drei Autoren in der Quelle genannt werden und alle drei unter den Top drei Git Autoren sind.
Die Reihenfolge in der sie in der Quelle genannt werden ist dabei irrelevant.
Es muss dabei jedoch berücksichtigt werden, dass auch diese Linien nicht direkt bei 100 \% starten, da bei der Betrachtung von null Git Autoren keine genannten Autoren unter den Top Git Autoren sein können.

Ebenfalls ist auffällig, dass einige Linien erst bei einem Wert von $x>2$ beginnen.
Dies ist der Fall, falls kein Autor in einer Quelle unter den Top x Committern ist.
Beispielsweise wird in dem \gls{pypi} Paket \emph{faker} in der \gls{cff} nur der Autor \glqq Daniele Faraglia\grqq{} genannt.
Dieser ist jedoch erst auf Rang 37 der Top Git Autoren gemessen nach Commits gelistet.

\begin{figure}
    \begin{subfigure}{.5\textwidth}
        \centering
        \includesvg[width=.95\linewidth,inkscapelatex=false]{bilder/common_authors_2/2_pypi_cff.svg}
        \caption{\gls{pypi} \gls{cff} nach Commits}
        \label{fig:common_authors_2_pypi_cff}
    \end{subfigure}%
    \begin{subfigure}{.5\textwidth}
        \centering
        \includesvg[width=.95\linewidth,inkscapelatex=false]{bilder/common_authors_2_by_lines/2_pypi_cff_by_lines.svg}
        \caption{\gls{pypi} \gls{cff} nach geänderten Zeilen}
        \label{fig:common_authors_2_by_lines_pypi_cff}
    \end{subfigure}
    \begin{subfigure}{.5\textwidth}
        \centering
        \includesvg[width=.95\linewidth,inkscapelatex=false]{bilder/common_authors_2/2_cran_cff.svg}
        \caption{\gls{cran} \gls{cff} nach Commits}
        \label{fig:common_authors_2_cran_cff}
    \end{subfigure}%
    \begin{subfigure}{.5\textwidth}
        \centering
        \includesvg[width=.95\linewidth,inkscapelatex=false]{bilder/common_authors_2_by_lines/2_cran_cff_by_lines.svg}
        \caption{\gls{cran} \gls{cff} nach geänderten Zeilen}
        \label{fig:common_authors_2_by_lines_cran_cff}
    \end{subfigure}
    \caption{Anteil der genannten Autoren unter den Top Git Autoren}
    \label{fig:common_authors_2}
    \small
    \raggedright
    Die Abbildungen zeigen für die Pakete in den Listen \gls{pypi} \gls{cff} und \gls{cran} \gls{cff} den Anteil der genannten Autoren unter den Top Git Autoren. Die Linien stellen die unterschiedlichen Quellen dar. Auf der x Achse wird die Anzahl der betrachteten Git Autoren für \autoref{fig:common_authors_pypi_cff} und \autoref{fig:common_authors_cran_cff} sortiert nach der Anzahl der Commits und für \autoref{fig:common_authors_by_lines_pypi_cff} und \autoref{fig:common_authors_by_lines_cran_cff} sortiert nach der Anzahl der geänderten Zeilen angegeben. Die y Achse stellt jeweils den Anteil der genannten Autoren an den Git Autoren dar. Für die blaue Linie in \autoref{fig:common_authors_pypi_cff} gilt: Autoren in der \gls{cff} sind zu y \% unter den Top x Commitern.
\end{figure}

In \autoref{fig:total_authors_no_commits} wird der Anteil der genannten Autoren in unterschiedlichen Quellen ohne Commits in den Paketen der Listen \gls{pypi} \gls{cff} und \gls{cran} \gls{cff} dargestellt.
Die Ergebnisse für die weiteren Listen sind in \autoref{fig:total_authors_no_commits_anhang} dargestellt.
In den Abbildungen werden die vergangenen zehn Jahre dargestellt.
Dabei steigt mit steigender Jahreszahl der Anteil der Autoren, welche keinen Commit getätigt haben.
Das Ende der x Achse stellt den Tag der Datenbeschaffung dar.
Aus diesem Grund ist der Anteil der Autoren ohne Commits am Ende der x Achse bei 100 \%, da es nicht möglich ist in der Zukunft Commits zu tätigen.

In der \autoref{fig:total_authors_no_commits_pypi_cff} lässt sich erkennen, dass nahezu alle genannten Python Maintainer bis zum Jahr 2020 mindestens einen Commit getätigt haben und das ab diesem Jahr die in der README genannten Autoren im Vergleich zu den anderen Quellen anteilig die meisten Autoren ohne Commits genannt haben.
Die Linien im Hintergrund stellen erneut ein einzelnes Paket einer einzelnen Quelle dar, sodass ein Paket mit mehreren Linien für alle Quellen dargestellt wird.

\begin{figure}
    \begin{subfigure}{.5\textwidth}
        \centering
        \includesvg[width=.95\linewidth,inkscapelatex=false]{bilder/total_authors_no_commits/3_pypi_cff.svg}
        \caption{\gls{pypi} \gls{cff}}
        \label{fig:total_authors_no_commits_pypi_cff}
    \end{subfigure}%
    \begin{subfigure}{.5\textwidth}
        \centering
        \includesvg[width=.95\linewidth,inkscapelatex=false]{bilder/total_authors_no_commits/3_cran_cff.svg}
        \caption{\gls{cran} \gls{cff}}
        \label{fig:total_authors_no_commits_cran_cff}
    \end{subfigure}
    \caption{Autoren ohne Commits}
    \label{fig:total_authors_no_commits}
    \small
    \raggedright
    Die Abbildungen zeigen für die Pakete in den Listen \gls{pypi} \gls{cff} und \gls{cran} \gls{cff} den Anteil der Autoren ohne Commits. Die Linien stellen die unterschiedlichen Quellen dar. Auf der x Achse werden die Jahre dargestellt. Die y Achse stellt den Anteil der Autoren ohne Commits dar. Für die blaue Linie gilt in beiden Abbildungen: y \% der Autoren in der \gls{cff} haben seit dem Jahr x keinen Commit getätigt.
\end{figure}

Die \autoref{fig:overall_valid_cff_not_full} zeigt den Anteil der validen und invaliden \gls{cff} Dateien in allen fünf Listen.
Dabei wird außerdem unterschieden in Dateien, welche Anzeichen auf die Benutzung von \emph{cffinit} aufweisen und Dateien ohne die Verwendung von dem Programm.
Außerdem lässt sich in der Abbildung erkennen wie viele Pakete in welcher Liste das \gls{cff} verwendet haben.
Dabei ist auffällig, dass in der \gls{cran} \gls{cff} Liste keine 100 Pakete gelistet werden, obwohl die Liste 100 Pakete enthalten sollte, welche ein \gls{cff} verwenden.
Dies liegt daran, dass in der Liste, welche von Herrn Druskat bereitgestellt wurde, einige Pakete enthalten sind, welche keine \gls{cff} Datei (mehr) enthalten.
Konkret betrifft das in diesem Fall die drei Pakete \emph{gtsummary}, \emph{ggpp} und \emph{gginnards}.
Aus diesem Grund sind in den weiteren Abbildungen ebenfalls keine 100 Pakete für die \gls{cran} \gls{cff} Liste dargestellt.

Außerdem ist auffällig, dass in der \gls{cff} Liste und der \gls{pypi} \gls{cff} Liste jeweils ein Paket vorhanden ist, welches eine invalide \gls{cff} Datei enthält, obwohl für die Erstellung \emph{cffinit} verwendet wurde.
Konkret handelt es sich in beiden Fällen um das \gls{pypi} Paket \emph{modelyst-sqlmodel}.
Das Paket ist in beiden Listen enthalten, da in der \gls{cff} Liste ausschließlich \gls{pypi} Pakete enthalten sind, da diese die meisten Sterne auf GitHub haben.
In diesem Fall ist das Paket unter den Top 100 \gls{cff} Paketen auf GitHub und somit ebenfalls in der \gls{pypi} \gls{cff} Liste enthalten.
Im Allgemeinen sind alle \gls{pypi} Pakete, welche in der \gls{cff} Liste enthalten sind, ebenfalls in der \gls{pypi} \gls{cff} Liste enthalten.
In der \gls{cff} Liste ist kein einziges \gls{cran} Paket enthalten, da die \gls{cran} Pakete zu wenig Sterne auf GitHub haben, um unter den Top 100 \gls{cff} Paketen zu sein.

Die \autoref{fig:overall_valid_cff_full} zeigt die gleichen Informationen wie die zuvor beschriebene Abbildung.
Diesmal wurde die Analyse allerdings nicht auf einer Liste sondern für alle auf GitHub verfügbaren \gls{cff} Dateien, welche analysiert werden konnten durchgeführt.
Dabei ist auffällig, dass der Anteil der validen \gls{cff} Dateien zwar größer ist, jedoch der Anteil der invaliden \gls{cff} ebenfalls weiter gestiegen ist.

\begin{figure}
    \begin{subfigure}{.5\textwidth}
        \centering
        \includesvg[width=.95\linewidth,inkscapelatex=false]{bilder/overall_valid_cff.svg}
        \caption{Validität der \gls{cff} Dateien in den Listen}
        \label{fig:overall_valid_cff_not_full}
    \end{subfigure}%
    \begin{subfigure}{.5\textwidth}
        \centering
        \includesvg[width=.95\linewidth,inkscapelatex=false]{bilder/overall_valid_cff_full.svg}
        \caption{Validität aller \gls{cff} Dateien auf GitHub}
        \label{fig:overall_valid_cff_full}
    \end{subfigure}
    \caption{Validität der \gls{cff} Dateien}
    \label{fig:overall_valid_cff}
    \small
    \raggedright
    Die \autoref{fig:overall_valid_cff_not_full} zeigt für die verschiedenen analysierten Listen, wie viele der vorhandenen \gls{cff} Dateien valide sind, und ob \emph{cffinit} verwendet wurde oder nicht. Die \autoref{fig:overall_valid_cff_full} zeigt die gleichen Informationen für alle \gls{cff} Dateien auf GitHub.
\end{figure}

\autoref{fig:citation_counts} zeigt für die Quellen \gls{cff}, \glqq preferred-citation\grqq{} \gls{cff} und \hologo{BibTeX} den angegebenen Typ der Zitation.
Anhand der Abbildungen ist außerdem zu erkennen, wie oft welche Quelle in den Listen vorkommt.
Beispielsweise haben zwei von den Top 100 \gls{cran} Paketen eine \gls{cff} angegeben und fünf sind es bei den Top 100 \gls{pypi} Paketen.
Von diesen fünf Paketen hat kein Paket eine \glqq preferred-citation\grqq{} \gls{cff} angegeben.
In den \gls{cran} Listen haben beide Pakete hingegen ebenfalls eine \glqq preferred-citation\grqq{} \gls{cff} angegeben.
Außerdem ist auffällig, dass in allen Listen die Anzahl der Pakete, welche eine \hologo{BibTeX} Datei verwenden gering ist.
Bei dieser Betrachtung ist darauf zu achten, dass in dem \gls{cff} nicht zwingend eine \glqq preferred-citation\grqq{} angegeben werden muss.
Aus diesem Grund sind in \autoref{fig:citation_counts_preferred_citation_cff} weniger Pakete dargestellt aus in \autoref{fig:citation_counts_cff}.

In \autoref{fig:citation_counts_cff} ist auffällig, dass ein \gls{cran} Paket aus der \gls{cran} \gls{cff} Liste als Typ der Zitation \emph{article} angegeben hat.
Dies ist in einer \gls{cff} Datei nicht zulässig und führt zu einer ungültigen \gls{cff} Datei.
Dies betrifft das Paket \emph{worcs}.
Bei der Betrachtung aller \gls{cff} Dateien auf GitHub ist zu erkennen, dass von 18.520 17470 Pakete \emph{software} angegeben haben und 412 \emph{dataset}.
Alle anderen 638 Pakete haben einen anderen Typ der Zitation angegeben und sind dadurch ungültig.

\begin{figure}
    \begin{subfigure}{.5\textwidth}
        \centering
        \includesvg[width=.95\linewidth,inkscapelatex=false]{bilder/citation_counts_cff.svg}
        \caption{\gls{cff}}
        \label{fig:citation_counts_cff}
    \end{subfigure}%
    \begin{subfigure}{.5\textwidth}
        \centering
        \includesvg[width=.95\linewidth,inkscapelatex=false]{bilder/citation_counts_preferred_citation_cff.svg}
        \caption{\glqq preferred-citation\grqq{} \gls{cff}}
        \label{fig:citation_counts_preferred_citation_cff}
    \end{subfigure}
    \centering
    \begin{subfigure}{.5\textwidth}
        \centering
        \includesvg[width=.95\linewidth,inkscapelatex=false]{bilder/citation_counts_bib.svg}
        \caption{\hologo{BibTeX}}
        \label{fig:citation_counts_bib}
    \end{subfigure}
    \caption{Typ der angegebenen Zitationen der einzelnen Quellen}
    \label{fig:citation_counts}
    \small
    \raggedright
    Die Abbildungen zeigen für die drei unterschiedlichen Quellen, jeweils für alle fünf untersuchten Listen, welcher Typ von Zitation angegeben wurde. Es ist darauf zu achten, dass die y Achsen unterschiedlich skaliert sind.
\end{figure}

In \autoref{fig:doi} ist dargestellt wie viele der Quellen, welche potenziell eine \gls{doi} enthalten könnten, eine \gls{doi} in den Zitationen angegeben haben.
Dabei wird in Quellen in denen es möglich ist zwischen den unterschiedlichen Angaben einer \gls{doi} unterschieden.
In \autoref{fig:cff_doi} ist beispielsweise zu erkennen, dass in der \gls{cran} \gls{cff} Liste 57 Pakete eine \gls{doi} in dem \gls{cff} angegeben haben und 40 Pakete keine \gls{doi} angegeben haben.
Auffällig ist außerdem, dass in der \glqq preferred-citation\grqq{} in \autoref{fig:preferred_citation_doi} kein einziges Paket eine Collection \gls{doi} angegeben hat und ebenfalls nur ein Paket in den \glqq identifiers\grqq{} eine \gls{doi} angegeben hat.
Zusätzlich wird in \autoref{fig:full_doi} dargestellt wie viele aller \gls{cff} Dateien auf GitHub eine \gls{doi} angegeben haben.
Der Aufbau ist dabei wie zuvor.

\begin{figure}
    \begin{subfigure}{.5\textwidth}
        \centering
        \includesvg[width=.95\linewidth,inkscapelatex=false]{bilder/cff_doi.svg}
        \caption{\gls{cff}}
        \label{fig:cff_doi}
    \end{subfigure}%
    \begin{subfigure}{.5\textwidth}
        \centering
        \includesvg[width=.95\linewidth,inkscapelatex=false]{bilder/preferred_citation_doi.svg}
        \caption{\glqq preferred-citation\grqq{} \gls{cff}}
        \label{fig:preferred_citation_doi}
    \end{subfigure}
    \centering
    \begin{subfigure}{.5\textwidth}
        \centering
        \includesvg[width=.95\linewidth,inkscapelatex=false]{bilder/bib_doi.svg}
        \caption{\hologo{BibTeX}}
        \label{fig:bib_doi}
    \end{subfigure}
    \caption{Verwendung von \gls{doi}s in den Zitationen der einzelnen Quellen}
    \label{fig:doi}
    \small
    \raggedright
    Die Abbildungen zeigen für die drei unterschiedlichen Quellen, jeweils für alle fünf untersuchten Listen, wie oft eine \gls{doi} auf verschiedenen Arten in den Zitationen angegeben wurde.
\end{figure}

\begin{figure}
    \begin{subfigure}{.5\textwidth}
        \centering
        \includesvg[width=.95\linewidth,inkscapelatex=false]{bilder/full_cff_doi.svg}
        \caption{\gls{cff}}
        \label{fig:full_cff_doi}
    \end{subfigure}%
    \begin{subfigure}{.5\textwidth}
        \centering
        \includesvg[width=.95\linewidth,inkscapelatex=false]{bilder/full_preferred_citation_doi.svg}
        \caption{\glqq preferred-citation\grqq{} \gls{cff}}
        \label{fig:full_preferred_citation_doi}
    \end{subfigure}
    \caption{Verwendung von \gls{doi}s in allen \gls{cff} auf GitHub}
    \label{fig:full_doi}
    \small
    \raggedright
    Die Abbildungen zeigen für die \gls{cff} und \glqq preferred-citation\grqq{} Quellen, jeweils für alle alle \gls{cff} Dateien auf GitHub, wie oft eine \gls{doi} auf verschiedenen Arten in den Zitationen angegeben wurde.
\end{figure}

Zusätzlich zu den Graphen wurden weitere Ergebnisse aus den Daten berechnet.
Einerseits wurde berechnet, wie viel Zeit im Durchschnitt seit der letzten Aktualisierung vom Tag der Datenbeschaffung einer Quelle vergangen ist.
In \autoref{sec:gesamtheit_ergebnisse} wird zusätzlich darauf eingegangen, mit welcher Frequenz die Quellen aktualisiert werden.
Die Berechnung ist dabei jeweils nur für das \gls{cff}, das \hologo{BibTeX} Format und die README Dateien möglich, da in den anderen Quellen kein Zeitstempel ermittelt werden kann.
Die Ergebnisse sind in \autoref{tab:average_time_last_update} dargestellt.
Für \gls{cran} ist jeweils keine Zeit für \hologo{BibTeX} angegeben, da keines der Pakete in den Listen die Quelle verwendet.

In der Tabelle lässt sich beispielsweise ablesen, dass die README über alle Pakete in der \gls{pypi} \gls{cff} Liste im Durchschnitt 83 Tage nicht aktualisiert wurde.
Es ist außerdem auffällig, dass die README Dateien am aktuellsten sind und die \hologo{BibTeX} Dateien am längsten nicht aktualisiert wurden.

\begin{table}
    \centering
    \setlength{\tabcolsep}{8pt}
    \begin{tabular}{c|c|c|c|c|c}
        \toprule
        \textbf{Quelle} & \textbf{\gls{cran}} & \textbf{\gls{pypi}} & \textbf{\gls{cff}} & \textbf{\gls{pypi} \gls{cff}} & \textbf{\gls{cran} \gls{cff}} \\ \midrule
        README          & 257 & 184 & 73   & 83   & 133 \\
        \gls{cff}       & 484 & 326 & 555  & 488  & 240 \\
        \hologo{BibTeX} &     & 768 & 1159 & 1159 &     \\
        \bottomrule
    \end{tabular}
    \caption{Durchschnittliche Zeit seit der letzten Aktualisierung}
    \label{tab:average_time_last_update}
    \small
    \raggedright
    Die Tabelle gibt für die Quellen für die es möglich ist an, wie viel Zeit im Durchschnitt über alle Pakete seit der letzten Aktualisierung der Quelle vergangen ist. Die Zeit ist in Tagen angegeben.
\end{table}

In der \autoref{fig:similarities} wird für die Listen \gls{pypi} \gls{cff} und \gls{cran} \gls{cff} angegeben, wie ähnlich die Autoren in den Paketen innerhalb der Quellen sind.
Für die weiteren Listen sind die Grafiken in \autoref{fig:similarities_anhang} dargestellt.
Dabei werden nur jene Autoren berücksichtigt, welche abgeglichen werden konnten.
Es wird somit kein erneuter Abgleich zwischen den einzelnen Quellen anhand der Namen vorgenommen.

Die Histogramme sind jeweils in 20 Bins unterteilt, sodass die Ähnlichkeit in 5 \% Schritten angegeben wird.
Für jedes einzelne Paket wurde die Ähnlichkeit der Autoren untereinander berechnet und in die entsprechenden Bins einsortiert.
Die Häufigkeit ist dabei in absoluten Werten angegeben.

In der \autoref{fig:similarity_pypi_cff} ist beispielsweise zu erkennen, dass vier Pakete in der \gls{pypi} \gls{cff} Liste eine Ähnlichkeit von 100 \% aufweisen.
Das bedeutet, dass vier mal folgender Fall aufgetreten ist: Alle Quellen eines Pakets haben die gleichen Autoren gelistet, wobei kein Autor in einer Quelle fehlt und auch kein Autor in einer Quelle zu viel gelistet ist.
Dabei muss beachtet werden, dass dies davon abhängig ist, ob der Abgleich erfolgreich war oder nicht.
Außerdem ist auffällig, dass 14 der Pakete in der \gls{pypi} \gls{cff} Liste eine Ähnlichkeit aller Autoren in den Quellen von weniger als 5 \% aufweisen.

\begin{figure}
    \begin{subfigure}{.5\textwidth}
        \centering
        \includesvg[width=.95\linewidth,inkscapelatex=false]{bilder/similarity/similarity_pypi_cff.svg}
        \caption{\gls{pypi} \gls{cff}}
        \label{fig:similarity_pypi_cff}
    \end{subfigure}%
    \begin{subfigure}{.5\textwidth}
        \centering
        \includesvg[width=.95\linewidth,inkscapelatex=false]{bilder/similarity/similarity_cran_cff.svg}
        \caption{\gls{cran} \gls{cff}}
        \label{fig:similarity_cran_cff}
    \end{subfigure}
    \caption{Übereinstimmung der Autoren in den Paketen}
    \label{fig:similarities}
    \small
    \raggedright
    Das Histogramm zeigt für die \gls{pypi} \gls{cff} und \gls{cran} \gls{cff} Listen, wie ähnlich die Autoren in den Paketen innerhalb der Quellen sind. Auf der x Achse wird die Ähnlichkeit in Prozent angegeben. Die y Achse stellt die absolute Anzahl der Pakete dar, welche die entsprechende Ähnlichkeit aufweisen.
\end{figure}

\section{Ergebnisse mit Zeitverlauf}
\label{sec:gesamtheit_ergebnisse}
% TODO darauf eingehen, dass in diesem Abschnitt anders als zuvor auf die Daten mit zeitbetrachtung eingegangen wird. Also jene Daten mit den Zeitstempeln
% TODO auf average_time_between_updates_cff eingehen
% TODO auf average_time_between_updates_bib eingehen
% TODO auf average_time_between_updates_readme eingehen
% TODO auf average_lifespans eingehen
% TODO dabei darauf eingehen, dass erneut ein Namensabgleich zwischen den autoren gemacht wird in einer Quelle
% TODO auf Anhang referenzieren
% TODO bei den Grafiken wieder so wie bei den anderen auch auf die sachen gehen -> siehe TODOs in git history
% TODO Zeit Graph mit cff valide beschreiben wie die Berechnung grundlegend funktioniert also das wenn eins von valide zu invalide wird z.b. die beiden Linien sich gleichzeitig verändern. Jedes Paket wird nur einmal betrachtet
\begin{figure}
    \begin{subfigure}{.5\textwidth}
        \centering
        \includesvg[width=.95\linewidth,inkscapelatex=false]{bilder/valid_cff_by_time/overall_valid_cff_pypi_cff.svg}
        \caption{\gls{pypi} \gls{cff}}
        \label{fig:valid_cff_by_time_pypi_cff}
    \end{subfigure}%
    \begin{subfigure}{.5\textwidth}
        \centering
        \includesvg[width=.95\linewidth,inkscapelatex=false]{bilder/valid_cff_by_time/overall_valid_cff_cran_cff.svg}
        \caption{\gls{cran} \gls{cff}}
        \label{fig:valid_cff_by_time_cran_cff}
    \end{subfigure}
    \centering
    \begin{subfigure}{.5\textwidth}
        \centering
        \includesvg[width=.95\linewidth,inkscapelatex=false]{bilder/valid_cff_by_time/overall_valid_cff_cff_full.svg}
        \caption{Alle \gls{cff} Dateien}
        \label{fig:valid_cff_by_time_full}
    \end{subfigure}
    \caption{Validität der \gls{cff} Dateien über die Zeit}
    \label{fig:valid_cff_by_time}
    \small
    \raggedright
    % TODO Beschreibung
\end{figure}

\begin{figure}
    \begin{subfigure}{.5\textwidth}
        \centering
        \includesvg[width=.95\linewidth,inkscapelatex=false]{bilder/added_removed_authors.svg}
        \caption{Inklusive README Autoren und ersten Autoren}
        \label{fig:added_removed_authors}
    \end{subfigure}%
    \begin{subfigure}{.5\textwidth}
        \centering
        \includesvg[width=.95\linewidth,inkscapelatex=false]{bilder/added_removed_authors_without_readme.svg}
        \caption{Exklusive README Autoren und inklusive ersten Autoren}
        \label{fig:added_removed_authors_without_readme}
    \end{subfigure}
    \begin{subfigure}{.5\textwidth}
        \centering
        \includesvg[width=.95\linewidth,inkscapelatex=false]{bilder/added_authors_without_first_timestamp.svg}
        \caption{Inklusive README Autoren und exklusive ersten Autoren}
        \label{fig:added_authors_without_first_timestamp}
    \end{subfigure}%
    \begin{subfigure}{.5\textwidth}
        \centering
        \includesvg[width=.95\linewidth,inkscapelatex=false]{bilder/added_authors_without_readme_without_first_timestamp.svg}
        \caption{Exklusive README Autoren und ersten Autoren}
        \label{fig:added_authors_without_readme_without_first_timestamp}
    \end{subfigure}
    \caption{Hinzugefügte und gelöschte Autoren}
    \small
    Die beiden Abbildungen \autoref{fig:added_removed_authors} und \autoref{fig:added_removed_authors_without_readme} zeigen wie viele Autoren über die Zeit in den Quellen hinzugefügt und gelöscht wurden. \autoref{fig:added_removed_authors} inkludiert Autoren in der README Datei, \autoref{fig:added_removed_authors_without_readme} exkludiert diese. Die Abbildungen \autoref{fig:added_authors_without_first_timestamp} und \autoref{fig:added_authors_without_readme_without_first_timestamp} zeigen ausschließlich die hinzugefügten Autoren, wobei die Autoren, die zu Beginn direkt in die Quelle eingetragen wurden, nicht berücksichtigt werden.
\end{figure}
