\chapter{Ergebnisse}
\label{chap:ergebnisse}
% TODO Beschreiben wie die Ergebnisse dargestellt wurden mittels Matplotlib in der Version x.x.x -> vllt. eher in Ergebnisse?
% TODO sagen von wann die REPO Daten sind
\section{Ergebnisse der Gegenwart}
\label{sec:neuste_ergebnisse}
\begin{figure}
    \begin{subfigure}{.5\textwidth}
        \centering
        \includesvg[width=.95\linewidth,inkscapelatex=false]{bilder/common_authors/1_pypi.svg}
        \caption{\gls{pypi}}
        \label{fig:common_authors_pypi}
    \end{subfigure}%
    \begin{subfigure}{.5\textwidth}
        \centering
        \includesvg[width=.95\linewidth,inkscapelatex=false]{bilder/common_authors/1_cran.svg}
        \caption{\gls{cran}}
        \label{fig:common_authors_cran}
    \end{subfigure}
    \begin{subfigure}{.5\textwidth}
        \centering
        \includesvg[width=.95\linewidth,inkscapelatex=false]{bilder/common_authors/1_cff.svg}
        \caption{\gls{cff}}
        \label{fig:common_authors_cff}
    \end{subfigure}%
    \begin{subfigure}{.5\textwidth}
        \centering
        \includesvg[width=.95\linewidth,inkscapelatex=false]{bilder/common_authors/1_pypi_cff.svg}
        \caption{\gls{pypi} \gls{cff}}
        \label{fig:common_authors_pypi_cff}
    \end{subfigure}
    \centering
    \begin{subfigure}{.5\textwidth}
        \centering
        \includesvg[width=.95\linewidth,inkscapelatex=false]{bilder/common_authors/1_cran_cff.svg}
        \caption{\gls{cran} \gls{cff}}
        \label{fig:common_authors_cran_cff}
    \end{subfigure}
    \caption{Verhältnis der Top Git Autoren in der Zitation}
    \small
    Die Abbildungen zeigen für die verschiedenen analysierten Listen das Verhältnis der genannten Autoren an den Git Autoren. Sie zeigen also für die ersten eins bis 100 Top Git Autoren gemessen an der Anzahl der Commits, zu wie viel Prozent diese in der jeweiligen Quelle z.B. \gls{cff} genannt werden. Falls der Autor mit den meisten Commits beispielsweise in der \gls{cff} genannt wird, und nur ein Git Autor betrachtet wird, ist an dieser Stelle der Wert 1.0 und somit 100 \%.
\end{figure}

\begin{figure}
    \begin{subfigure}{.5\textwidth}
        \centering
        \includesvg[width=.95\linewidth,inkscapelatex=false]{bilder/common_authors_2/2_pypi.svg}
        \caption{\gls{pypi}}
        \label{fig:common_authors_2_pypi}
    \end{subfigure}%
    \begin{subfigure}{.5\textwidth}
        \centering
        \includesvg[width=.95\linewidth,inkscapelatex=false]{bilder/common_authors_2/2_cran.svg}
        \caption{\gls{cran}}
        \label{fig:common_authors_2_cran}
    \end{subfigure}
    \begin{subfigure}{.5\textwidth}
        \centering
        \includesvg[width=.95\linewidth,inkscapelatex=false]{bilder/common_authors_2/2_cff.svg}
        \caption{\gls{cff}}
        \label{fig:common_authors_2_cff}
    \end{subfigure}%
    \begin{subfigure}{.5\textwidth}
        \centering
        \includesvg[width=.95\linewidth,inkscapelatex=false]{bilder/common_authors_2/2_pypi_cff.svg}
        \caption{\gls{pypi} \gls{cff}}
        \label{fig:common_authors_2_pypi_cff}
    \end{subfigure}
    \centering
    \begin{subfigure}{.5\textwidth}
        \centering
        \includesvg[width=.95\linewidth,inkscapelatex=false]{bilder/common_authors_2/2_cran_cff.svg}
        \caption{\gls{cran} \gls{cff}}
        \label{fig:common_authors_2_cran_cff}
    \end{subfigure}
    \caption{Verhältnis der angegebenen Autoren am Quellcode}
    \small
    Die Abbildungen zeigen für die verschiedenen analysierten Listen das Verhältnis der angegebenen Autoren am Quellcode. Sie zeigen also für die ersten null bis 200 Git Autoren gemessen an der Anzahl der Commits, zu wie viel Prozent die genannten Autoren unter Berücksichtigung der n Git Autoren tatsächlich Quellcode entwickelt haben. Für null Git Autoren ist dieser Wert immer null, da kein genannter Autor unter den Top 0 Git Autoren sein kann. Falls 10 Autoren genannt werden und diese alle Quellcode entwickelt haben und zusätzlich unter den Top n Git Autoren sind, ist der Wert, ab dem Zeitpunkt zu dem alle Autoren in der Liste der Git Autoren vorhanden sind, 1.0. Dieser Graph kann also 1.0 erreichen, falls alle genannten Autoren tatsächlich Quellcode entwickelt haben und in der Datenbeschaffung abgeglichen werden konnten.
\end{figure}

\begin{figure}
    \begin{subfigure}{.5\textwidth}
        \centering
        \includesvg[width=.95\linewidth,inkscapelatex=false]{bilder/total_authors_no_commits/3_pypi.svg}
        \caption{\gls{pypi}}
        \label{fig:total_authors_no_commits_pypi}
    \end{subfigure}%
    \begin{subfigure}{.5\textwidth}
        \centering
        \includesvg[width=.95\linewidth,inkscapelatex=false]{bilder/total_authors_no_commits/3_cran.svg}
        \caption{\gls{cran}}
        \label{fig:total_authors_no_commits_cran}
    \end{subfigure}
    \begin{subfigure}{.5\textwidth}
        \centering
        \includesvg[width=.95\linewidth,inkscapelatex=false]{bilder/total_authors_no_commits/3_cff.svg}
        \caption{\gls{cff}}
        \label{fig:total_authors_no_commits_cff}
    \end{subfigure}%
    \begin{subfigure}{.5\textwidth}
        \centering
        \includesvg[width=.95\linewidth,inkscapelatex=false]{bilder/total_authors_no_commits/3_pypi_cff.svg}
        \caption{\gls{pypi} \gls{cff}}
        \label{fig:total_authors_no_commits_pypi_cff}
    \end{subfigure}
    \centering
    \begin{subfigure}{.5\textwidth}
        \centering
        \includesvg[width=.95\linewidth,inkscapelatex=false]{bilder/total_authors_no_commits/3_cran_cff.svg}
        \caption{\gls{cran} \gls{cff}}
        \label{fig:total_authors_no_commits_cran_cff}
    \end{subfigure}
    \caption{Autoren ohne Commits}
    \small
    Die Abbildungen zeigen für die verschiedenen analysierten Listen wie viel Prozent der genannten Autoren in der jeweiligen Quelle aktiv sind. 0 Tage stellt dabei den Tag dar, an dem die Datenbeschaffung ausgeführt wurde. An diesem Tag sind nahezu 100 \% der Autoren nicht aktiv, da sie an diesem Tag bereits ein Commit getätigt haben müssten. Mit steigender Anzahl an Tagen werden immer mehr Autoren aktiv.
\end{figure}

\begin{figure}
    \includesvg[inkscapelatex=false]{bilder/overall_valid_cff.svg}
    \label{fig:overall_valid_cff}
    \caption{Validität der \gls{cff} Dateien}
    \small
    Die Abbildung zeigt für die verschiedenen analysierten Listen, wie viele der vorhandenen \gls{cff} Dateien valide sind, und ob \emph{cffinit} verwendet wurde oder nicht. Es wird nur die jeweils neuste \gls{cff} Datei betrachtet.
\end{figure}

\begin{figure}
    \begin{subfigure}{.5\textwidth}
        \centering
        \includesvg[width=.95\linewidth,inkscapelatex=false]{bilder/citation_counts_cff.svg}
        \caption{\gls{cff}}
        \label{fig:citation_counts_cff}
    \end{subfigure}%
    \begin{subfigure}{.5\textwidth}
        \centering
        \includesvg[width=.95\linewidth,inkscapelatex=false]{bilder/citation_counts_preferred_citation_cff.svg}
        \caption{\glqq preferred-citation\grqq{} \gls{cff}}
        \label{fig:citation_counts_preferred_citation_cff}
    \end{subfigure}
    \centering
    \begin{subfigure}{.5\textwidth}
        \centering
        \includesvg[width=.95\linewidth,inkscapelatex=false]{bilder/citation_counts_bib.svg}
        \caption{\hologo{BibTeX}}
        \label{fig:citation_counts_bib}
    \end{subfigure}
    \caption{Typ der angegebenen Zitationen der einzelnen Quellen}
    \small
    Die Abbildungen zeigen für die drei unterschiedlichen Quellen, jeweils für alle fünf untersuchten Listen, welcher Typ von Zitation angegeben wurde. Es werden nur die neusten Versionen betrachtet.
\end{figure}

\section{Ergebnisse mit Zeitverlauf}
\label{sec:gesamtheit_ergebnisse}
% TODO Pakete, welche händisch geprüft wurden hier erwähnen git #20
% TODO Das Ergebnis explizit zeigen/ Die Situation in den Tabellen explizit erklären!/ Haben wir überhaupt einen Grund das zu machen was wir machen?/ Tabellen erklären! -> zeigen und erklären, dass einige nicht genannt werden
% TODO Auf das Prinzip 1 eingehen, dass wichtigkeit besonders vernachlässigt wird bei prefrerred Zitation -> zeige ich in den Ergebnissen wobei nur weil preferred citation angegeben ist heißt es nicht, dass paper nur das zitieren sie sollten beides zitieren. Aber wenn die Standard zitation article angibt und keine preferred citation auf die software verweist dann ist kacka
% TODO bei cff sagen, dass von 100 top repos x pypi x cran und x andere sind
% TODO Es werden alle Namen ausgegeben in den Beschreibungen aber auch eben solche, die gar nichts mit dem Paket zu tun haben wie im fall von highr für CRAN: "Provides syntax highlighting for R source Code. Currently it supports LaTeX and HTML output. Source Code of other languages is supported via Andre Simon's highlight package (https://gitlab.com/saalen/highlight)." Es gibt noch weitere Beispiele z.B. in CRAN magrittr -> vllt eher ergebnis?
% TODO sagen wann jeweils die Daten beschafft wurden also wann die Repos heruntergeladen wurden
% TODO auf die manuell beschafften Daten eingehen

\begin{figure}
    \includesvg[inkscapelatex=false]{bilder/overall_full_valid_cff.svg}
    \label{fig:overall_full_valid_cff}
    \caption{Validität der \gls{cff} Dateien mit Zeitverlauf}
    \small
    Die Abbildung zeigt für die verschiedenen analysierten Listen, wie viele der vorhandenen \gls{cff} Dateien valide sind, und ob \emph{cffinit} verwendet wurde oder nicht. Dabei wurden alle Versionen der \gls{cff} Dateien betrachtet.
\end{figure}

\begin{figure}
    \begin{subfigure}{.5\textwidth}
        \centering
        \includesvg[width=.95\linewidth,inkscapelatex=false]{bilder/citation_counts_full_cff.svg}
        \caption{\gls{cff}}
        \label{fig:citation_counts_cff}
    \end{subfigure}%
    \begin{subfigure}{.5\textwidth}
        \centering
        \includesvg[width=.95\linewidth,inkscapelatex=false]{bilder/citation_counts_full_preferred_citation_cff.svg}
        \caption{\glqq preferred-citation\grqq{} \gls{cff}}
        \label{fig:citation_counts_preferred_citation_cff}
    \end{subfigure}
    \centering
    \begin{subfigure}{.5\textwidth}
        \centering
        \includesvg[width=.95\linewidth,inkscapelatex=false]{bilder/citation_counts_full_bib.svg}
        \caption{\hologo{BibTeX}}
        \label{fig:citation_counts_bib}
    \end{subfigure}
    \caption{Typ der angegebenen Zitationen der einzelnen Quellen mit Zeitverlauf}
    \small
    Die Abbildungen zeigen für die drei unterschiedlichen Quellen, jeweils für alle fünf untersuchten Listen, welcher Typ von Zitation angegeben wurde. Es werden alle Versionen der Dateien betrachtet.
\end{figure}

\begin{figure}
    \begin{subfigure}{.5\textwidth}
        \centering
        \includesvg[width=.95\linewidth,inkscapelatex=false]{bilder/added_removed_authors.svg}
        \caption{Inklusive README Autoren}
        \label{fig:added_removed_authors}
    \end{subfigure}%
    \begin{subfigure}{.5\textwidth}
        \centering
        \includesvg[width=.95\linewidth,inkscapelatex=false]{bilder/added_removed_authors_without_readme.svg}
        \caption{Exklusive README Autoren}
        \label{fig:added_removed_authors_without_readme}
    \end{subfigure}
    \caption{Hinzugefügte und gelöschte Autoren}
    \small
    Die beiden Abbildungen zeigen wie viele Autoren über die Zeit in den Quellen hinzugefügt und gelöscht wurden. \autoref{fig:added_removed_authors} inkludiert Autoren in der README Datei, \autoref{fig:added_removed_authors_without_readme} exkludiert diese.
\end{figure}
