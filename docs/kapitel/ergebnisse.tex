\chapter{Ergebnisse}
\label{chap:ergebnisse}
% TODO Pakete, welche händisch geprüft wurden hier erwähnen git #20
% TODO Das Ergebnis explizit zeigen/ Die Situation in den Tabellen explizit erklären!/ Haben wir überhaupt einen Grund das zu machen was wir machen?/ Tabellen erklären! -> zeigen und erklären, dass einige nicht genannt werden
% TODO Auf das Prinzip 1 eingehen, dass wichtigkeit besonders vernachlässigt wird bei prefrerred Zitation -> zeige ich in den Ergebnissen wobei nur weil preferred citation angegeben ist heißt es nicht, dass paper nur das zitieren sie sollten beides zitieren. Aber wenn die Standard zitation article angibt und keine preferred citation auf die software verweist dann ist kacka
% TODO bei cff sagen, dass von 100 top repos x pypi x cran und x andere sind
% TODO Es werden alle Namen ausgegeben in den Beschreibungen aber auch eben solche, die gar nichts mit dem Paket zu tun haben wie im fall von highr für CRAN: "Provides syntax highlighting for R source Code. Currently it supports LaTeX and HTML output. Source Code of other languages is supported via Andre Simon's highlight package (https://gitlab.com/saalen/highlight)." Es gibt noch weitere Beispiele z.B. in CRAN magrittr -> vllt eher ergebnis?
% TODO sagen wann jeweils die Daten beschafft wurden also wann die Repos heruntergeladen wurden
% TODO auf die manuell beschafften Daten eingehen
