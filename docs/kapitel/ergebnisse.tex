\chapter{Ergebnisse}
\label{chap:ergebnisse}
% TODO darauf achten, dass jede Grafik mindestens einmal im Text referenziert wird
In diesem Kapitel werden die Ergebnisse der Masterarbeit präsentiert.
Das Kapitel ist in zwei Abschnitte unterteilt.
Zum einen werden die Ergebnisse der Gegenwart in \autoref{sec:neuste_ergebnisse} präsentiert.
Hierbei werden die jeweils neusten Dateien aus der Datenbeschaffung untersucht und Statistiken aufgeführt, welche in Verbindung mit diesen Daten stehen.
Zum anderen werden die Ergebnisse mit Zeitverlauf in \autoref{sec:gesamtheit_ergebnisse} präsentiert.
Hierbei werden die Dateien aus der Datenbeschaffung in allen Versionen analysiert.
Durch die Betrachtung aller Versionen können Veränderungen über die Zeit aufgezeigt werden und somit eine Aussage über die Entwicklung der Zitationen getroffen werden.
In beiden Abschnitten wird zu Beginn auf die Statistiken des Abgleichs eingegangen und präsentiert, wie genau der Abgleich durchgeführt wurde.
Der Abgleich hat einen direkten Einfluss auf viele weitere Statistiken.

Die benötigten Daten für die Untersuchungen wurden in der Zeitspanne September 2024 bis November 2024 beschafft.
Dies liegt daran, dass die Repositorys lokal gespeichert werden und bei der Entwicklung der Arbeit die Daten nicht erneut beschafft werden mussten.
Die Daten einer Liste wurden jeweils an einem Tag beschafft, um eine Vergleichbarkeit zu gewährleisten.
In dem gesamten Kapitel wird \emph{matplotlib} in der Version 3.9.2 verwendet, um die Grafiken zu erstellen \autocite{hunter_matplotlib_2007}.

\section{Ergebnisse der Gegenwart}
\label{sec:neuste_ergebnisse}
% TODO auf meine Ergebnisse vom Matching eingehen (False Positives, False Negatives, True Positives, True Negatives) -> Auf meine Ergebnisse nur für eine Liste eingehen? Und FP,FN,TP,TN auch nur für eine Quelle?
% TODO Auf average_time_last_update_cff eingehen
% TODO auf average_time_last_update_bib eingehen
% TODO auf average_time_last_update_readme eingehen
% TODO auf similarity_with_non_matches eingehen
% TODO auf similarity_without_non_matches eingehen
% TODO Die komplizierten Grafiken gut erklären! Auch mit eins zwei beispielen z.B. bei 20 betrachteten autoren...  sind 40% blablabla (Aufpassen achsen sind nicht prozent das erklären)



% TODO anekdotisch untermauern mit eins zwei bsp. warum einige linien unten erst bei x=1 oder 2 beginnen. Eins zwei pakete nennen bei denen die ersten commiter nicht genannt werden.
% TODO in x Achsen beschreibung einfügen "Anzahl der betrachteten Git Autoren sortiert nach Commits"
\begin{figure}
    % TODO Daneben direkt bilder/common_authors_by_files/1_pypi_by_files.svg anzeigen und das für eins zwei ausgewählte listen machen und nicht für alle. sondern nur für listen um meine Aussage zu zeigen. Die anderen Graphen in den Anhang
    \begin{subfigure}{.5\textwidth}
        \centering
        \includesvg[width=.95\linewidth,inkscapelatex=false]{bilder/common_authors/1_pypi.svg}
        \caption{\gls{pypi}}
        \label{fig:common_authors_pypi}
    \end{subfigure}%
    \begin{subfigure}{.5\textwidth}
        \centering
        \includesvg[width=.95\linewidth,inkscapelatex=false]{bilder/common_authors/1_cran.svg}
        \caption{\gls{cran}}
        \label{fig:common_authors_cran}
    \end{subfigure}
    \begin{subfigure}{.5\textwidth}
        \centering
        \includesvg[width=.95\linewidth,inkscapelatex=false]{bilder/common_authors/1_cff.svg}
        \caption{\gls{cff}}
        \label{fig:common_authors_cff}
    \end{subfigure}%
    \begin{subfigure}{.5\textwidth}
        \centering
        \includesvg[width=.95\linewidth,inkscapelatex=false]{bilder/common_authors/1_pypi_cff.svg}
        \caption{\gls{pypi} \gls{cff}}
        \label{fig:common_authors_pypi_cff}
    \end{subfigure}
    \centering
    \begin{subfigure}{.5\textwidth}
        \centering
        \includesvg[width=.95\linewidth,inkscapelatex=false]{bilder/common_authors/1_cran_cff.svg}
        \caption{\gls{cran} \gls{cff}}
        \label{fig:common_authors_cran_cff}
    \end{subfigure}
    \caption{Anteil der Top Git Autoren nach Commits in der Zitation}
    \label{fig:common_authors}
    \small
    \raggedright
    Die Abbildungen zeigen für die verschiedenen analysierten Listen den Anteil der genannten Autoren an den Git Autoren. Sie zeigen also für die ersten eins bis 100 Top Git Autoren gemessen an der Anzahl der Commits, zu wie viel Prozent diese in der jeweiligen Quelle z.B. \gls{cff} genannt werden. Falls der Autor mit den meisten Commits beispielsweise in der \gls{cff} genannt wird, und nur ein Git Autor betrachtet wird, ist an dieser Stelle der Wert 1.0 und somit 100 \%.
\end{figure}

% TODO die Bilder neben den sortiert nach commits packen
% x Achsen beschreibung: "Anzahl der betrachteten Git Autoren sortiert nach changed_lines"
\begin{figure}
    \begin{subfigure}{.5\textwidth}
        \centering
        \includesvg[width=.95\linewidth,inkscapelatex=false]{bilder/common_authors_by_files/1_pypi_by_files.svg}
        \caption{\gls{pypi}}
        \label{fig:common_authors_by_files_pypi}
    \end{subfigure}%
    \begin{subfigure}{.5\textwidth}
        \centering
        \includesvg[width=.95\linewidth,inkscapelatex=false]{bilder/common_authors_by_files/1_cran_by_files.svg}
        \caption{\gls{cran}}
        \label{fig:common_authors_by_files_cran}
    \end{subfigure}
    \begin{subfigure}{.5\textwidth}
        \centering
        \includesvg[width=.95\linewidth,inkscapelatex=false]{bilder/common_authors_by_files/1_cff_by_files.svg}
        \caption{\gls{cff}}
        \label{fig:common_authors_by_files_cff}
    \end{subfigure}%
    \begin{subfigure}{.5\textwidth}
        \centering
        \includesvg[width=.95\linewidth,inkscapelatex=false]{bilder/common_authors_by_files/1_pypi_cff_by_files.svg}
        \caption{\gls{pypi} \gls{cff}}
        \label{fig:common_authors_by_files_pypi_cff}
    \end{subfigure}
    \centering
    \begin{subfigure}{.5\textwidth}
        \centering
        \includesvg[width=.95\linewidth,inkscapelatex=false]{bilder/common_authors_by_files/1_cran_cff_by_files.svg}
        \caption{\gls{cran} \gls{cff}}
        \label{fig:common_authors_by_files_cran_cff}
    \end{subfigure}
    \caption{Anteil der Top Git Autoren nach geänderten Zeilen in der Zitation}
    \label{fig:common_authors_by_files}
    \small
    \raggedright
    Die Abbildungen zeigen die gleichen Graphen wie \autoref{fig:common_authors}, jedoch werden hier die Git Autoren über die Anzahl der geänderten Zeilen gemessen.
\end{figure}

% TODO anekdotisch untermauern mit eins zwei bsp. warum einige linien unten erst bei x=1 oder 2 beginnen. Eins zwei pakete nennen bei denen die ersten commiter nicht genannt werden.
% x Achsen beschreibung: "Anzahl der betrachteten Git Autoren sortiert nach Commits"
\begin{figure}
        % TODO Daneben direkt bilder/common_authors_2_by_files/2_pypi_by_files.svg anzeigen und das für eins zwei ausgewählte listen machen und nicht für alle. sondern nur für listen um meine Aussage zu zeigen. Die anderen Graphen in den Anhang
    \begin{subfigure}{.5\textwidth}
        \centering
        \includesvg[width=.95\linewidth,inkscapelatex=false]{bilder/common_authors_2/2_pypi.svg}
        \caption{\gls{pypi}}
        \label{fig:common_authors_2_pypi}
    \end{subfigure}%
    \begin{subfigure}{.5\textwidth}
        \centering
        \includesvg[width=.95\linewidth,inkscapelatex=false]{bilder/common_authors_2/2_cran.svg}
        \caption{\gls{cran}}
        \label{fig:common_authors_2_cran}
    \end{subfigure}
    \begin{subfigure}{.5\textwidth}
        \centering
        \includesvg[width=.95\linewidth,inkscapelatex=false]{bilder/common_authors_2/2_cff.svg}
        \caption{\gls{cff}}
        \label{fig:common_authors_2_cff}
    \end{subfigure}%
    \begin{subfigure}{.5\textwidth}
        \centering
        \includesvg[width=.95\linewidth,inkscapelatex=false]{bilder/common_authors_2/2_pypi_cff.svg}
        \caption{\gls{pypi} \gls{cff}}
        \label{fig:common_authors_2_pypi_cff}
    \end{subfigure}
    \centering
    \begin{subfigure}{.5\textwidth}
        \centering
        \includesvg[width=.95\linewidth,inkscapelatex=false]{bilder/common_authors_2/2_cran_cff.svg}
        \caption{\gls{cran} \gls{cff}}
        \label{fig:common_authors_2_cran_cff}
    \end{subfigure}
    \caption{Anteil der angegebenen Autoren nach Commits am Quellcode}
    \label{fig:common_authors_2}
    \small
    \raggedright
    % TODO die Beschreibungen alle gleich aufbauen. Achsen beschreiben, Linien beschreiben, auf a, b, c, d, e eingehen und einen prägnanten satz wie: " In der CFF sind y% unter den Top x commitern.
    % TODO die Große beschreibung im Text mit beispielen erklären
    Die Abbildungen zeigen für die verschiedenen analysierten Listen den Anteil der angegebenen Autoren am Quellcode. Sie zeigen also für die ersten null bis 200 Git Autoren gemessen an der Anzahl der Commits, zu wie viel Prozent die genannten Autoren unter Berücksichtigung der n Git Autoren tatsächlich Quellcode entwickelt haben. Für null Git Autoren ist dieser Wert immer null, da kein genannter Autor unter den Top 0 Git Autoren sein kann. Falls 10 Autoren genannt werden und diese alle Quellcode entwickelt haben und zusätzlich unter den Top n Git Autoren sind, ist der Wert, ab dem Zeitpunkt zu dem alle Autoren in der Liste der Git Autoren vorhanden sind, 1.0. Dieser Graph kann also 1.0 erreichen, falls alle genannten Autoren tatsächlich Quellcode entwickelt haben und in der Datenbeschaffung abgeglichen werden konnten.
\end{figure}

% x Achsen beschreibung: "Anzahl der betrachteten Git Autoren sortiert nach changed_lines"
\begin{figure}
    \begin{subfigure}{.5\textwidth}
        \centering
        \includesvg[width=.95\linewidth,inkscapelatex=false]{bilder/common_authors_2_by_files/2_pypi_by_files.svg}
        \caption{\gls{pypi}}
        \label{fig:common_authors_2_by_files_pypi}
    \end{subfigure}%
    \begin{subfigure}{.5\textwidth}
        \centering
        \includesvg[width=.95\linewidth,inkscapelatex=false]{bilder/common_authors_2_by_files/2_cran_by_files.svg}
        \caption{\gls{cran}}
        \label{fig:common_authors_2_by_files_cran}
    \end{subfigure}
    \begin{subfigure}{.5\textwidth}
        \centering
        \includesvg[width=.95\linewidth,inkscapelatex=false]{bilder/common_authors_2_by_files/2_cff_by_files.svg}
        \caption{\gls{cff}}
        \label{fig:common_authors_2_by_files_cff}
    \end{subfigure}%
    \begin{subfigure}{.5\textwidth}
        \centering
        \includesvg[width=.95\linewidth,inkscapelatex=false]{bilder/common_authors_2_by_files/2_pypi_cff_by_files.svg}
        \caption{\gls{pypi} \gls{cff}}
        \label{fig:common_authors_2_by_files_pypi_cff}
    \end{subfigure}
    \centering
    \begin{subfigure}{.5\textwidth}
        \centering
        \includesvg[width=.95\linewidth,inkscapelatex=false]{bilder/common_authors_2_by_files/2_cran_cff_by_files.svg}
        \caption{\gls{cran} \gls{cff}}
        \label{fig:common_authors_2_by_files_cran_cff}
    \end{subfigure}
    \caption{Anteil der angegebenen Autoren nach geänderten Zeilen am Quellcode}
    \label{fig:common_authors_2_by_files}
    \small
    \raggedright
    Die Abbildungen zeigen die gleichen Graphen wie \autoref{fig:common_authors_2}, jedoch werden hier die Git Autoren über die Anzahl der geänderten Zeilen gemessen.
\end{figure}

\begin{figure}
    \begin{subfigure}{.5\textwidth}
        \centering
        \includesvg[width=.95\linewidth,inkscapelatex=false]{bilder/total_authors_no_commits/3_pypi.svg}
        \caption{\gls{pypi}}
        \label{fig:total_authors_no_commits_pypi}
    \end{subfigure}%
    \begin{subfigure}{.5\textwidth}
        \centering
        \includesvg[width=.95\linewidth,inkscapelatex=false]{bilder/total_authors_no_commits/3_cran.svg}
        \caption{\gls{cran}}
        \label{fig:total_authors_no_commits_cran}
    \end{subfigure}
    \begin{subfigure}{.5\textwidth}
        \centering
        \includesvg[width=.95\linewidth,inkscapelatex=false]{bilder/total_authors_no_commits/3_cff.svg}
        \caption{\gls{cff}}
        \label{fig:total_authors_no_commits_cff}
    \end{subfigure}%
    \begin{subfigure}{.5\textwidth}
        \centering
        \includesvg[width=.95\linewidth,inkscapelatex=false]{bilder/total_authors_no_commits/3_pypi_cff.svg}
        \caption{\gls{pypi} \gls{cff}}
        \label{fig:total_authors_no_commits_pypi_cff}
    \end{subfigure}
    \centering
    \begin{subfigure}{.5\textwidth}
        \centering
        \includesvg[width=.95\linewidth,inkscapelatex=false]{bilder/total_authors_no_commits/3_cran_cff.svg}
        \caption{\gls{cran} \gls{cff}}
        \label{fig:total_authors_no_commits_cran_cff}
    \end{subfigure}
    \caption{Autoren ohne Commits}
    \small
    \raggedright
    % TODO y% der Autoren haben in den letzten x Tagen ...
    % y Achse in Jahren darstellen nicht in Tagen
    Die Abbildungen zeigen für die verschiedenen analysierten Listen wie viel Prozent der genannten Autoren in der jeweiligen Quelle aktiv sind. 0 Tage stellt dabei den Tag dar, an dem die Datenbeschaffung ausgeführt wurde. An diesem Tag sind nahezu 100 \% der Autoren nicht aktiv, da sie an diesem Tag bereits ein Commit getätigt haben müssten. Mit steigender Anzahl an Tagen werden immer mehr Autoren aktiv.
\end{figure}

\begin{figure}
    \includesvg[inkscapelatex=false]{bilder/overall_valid_cff.svg}
    \label{fig:overall_valid_cff}
    \caption{Validität der \gls{cff} Dateien}
    \small
    \raggedright
    Die Abbildung zeigt für die verschiedenen analysierten Listen, wie viele der vorhandenen \gls{cff} Dateien valide sind, und ob \emph{cffinit} verwendet wurde oder nicht. Es wird nur die jeweils neuste \gls{cff} Datei betrachtet.
\end{figure}

\begin{figure}
    \begin{subfigure}{.5\textwidth}
        \centering
        \includesvg[width=.95\linewidth,inkscapelatex=false]{bilder/citation_counts_cff.svg}
        \caption{\gls{cff}}
        \label{fig:citation_counts_cff}
    \end{subfigure}%
    \begin{subfigure}{.5\textwidth}
        \centering
        \includesvg[width=.95\linewidth,inkscapelatex=false]{bilder/citation_counts_preferred_citation_cff.svg}
        \caption{\glqq preferred-citation\grqq{} \gls{cff}}
        \label{fig:citation_counts_preferred_citation_cff}
    \end{subfigure}
    \centering
    \begin{subfigure}{.5\textwidth}
        \centering
        \includesvg[width=.95\linewidth,inkscapelatex=false]{bilder/citation_counts_bib.svg}
        \caption{\hologo{BibTeX}}
        \label{fig:citation_counts_bib}
    \end{subfigure}
    \caption{Typ der angegebenen Zitationen der einzelnen Quellen}
    \small
    \raggedright
    Die Abbildungen zeigen für die drei unterschiedlichen Quellen, jeweils für alle fünf untersuchten Listen, welcher Typ von Zitation angegeben wurde. Es werden nur die neusten Versionen betrachtet.
    % TODO Achtung verschiedene y Achsen erwähnen!
\end{figure}

\begin{figure}
    \begin{subfigure}{.5\textwidth}
        \centering
        \includesvg[width=.95\linewidth,inkscapelatex=false]{bilder/cff_doi.svg}
        \caption{\gls{cff}}
        \label{fig:cff_doi}
    \end{subfigure}%
    \begin{subfigure}{.5\textwidth}
        \centering
        \includesvg[width=.95\linewidth,inkscapelatex=false]{bilder/preferred_citation_doi.svg}
        \caption{\glqq preferred-citation\grqq{} \gls{cff}}
        \label{fig:preferred_citation_doi}
    \end{subfigure}
    \centering
    \begin{subfigure}{.5\textwidth}
        \centering
        \includesvg[width=.95\linewidth,inkscapelatex=false]{bilder/bib_doi.svg}
        \caption{\hologo{BibTeX}}
        \label{fig:bib_doi}
    \end{subfigure}
    \caption{Verwendung von DOIs in den Zitationen der einzelnen Quellen}
    \small
    \raggedright
    Die Abbildungen zeigen für die drei unterschiedlichen Quellen, jeweils für alle fünf untersuchten LIsten, wie oft eine DOI in verschiedenen Versionen in den Zitationen angegeben wurde. Es werden nur die neusten Versionen betrachtet.
\end{figure}

\section{Ergebnisse mit Zeitverlauf}
\label{sec:gesamtheit_ergebnisse}
% TODO Pakete, welche händisch geprüft wurden hier erwähnen git #20
% TODO Das Ergebnis explizit zeigen/ Die Situation in den Tabellen explizit erklären!/ Haben wir überhaupt einen Grund das zu machen was wir machen?/ Tabellen erklären! -> zeigen und erklären, dass einige nicht genannt werden
% TODO Auf das Prinzip 1 eingehen, dass wichtigkeit besonders vernachlässigt wird bei prefrerred Zitation -> zeige ich in den Ergebnissen wobei nur weil preferred citation angegeben ist heißt es nicht, dass paper nur das zitieren sie sollten beides zitieren. Aber wenn die Standard zitation article angibt und keine preferred citation auf die software verweist dann ist kacka
% TODO bei cff sagen, dass von 100 top repos x pypi x cran und x andere sind
% TODO Es werden alle Namen ausgegeben in den Beschreibungen aber auch eben solche, die gar nichts mit dem Paket zu tun haben wie im fall von highr für CRAN: "Provides syntax highlighting for R source Code. Currently it supports LaTeX and HTML output. Source Code of other languages is supported via Andre Simon's highlight package (https://gitlab.com/saalen/highlight)." Es gibt noch weitere Beispiele z.B. in CRAN magrittr -> vllt eher ergebnis?
% TODO sagen wann jeweils die Daten beschafft wurden also wann die Repos heruntergeladen wurden
% TODO auf die manuell beschafften Daten eingehen
% TODO in der Grafik mit den validen/ nicht validen CFF darauf eingehen, warum bei CRAN CFF keine 100 Pakete insgesamt sind -> weil gtsummary, ggpp, gginnards in der Liste von Druskat ist aber in GitHub ist gar keine CITATION.cff vorhanden.
% TODO Auf meine Ergebnisse vom Matching mit zeitverlauf eingehen. Nur die Daten zeigen, welche automatisch beschafft werden konnten.
% TODO auf average_time_between_updates_cff eingehen
% TODO auf average_time_between_updates_bib eingehen
% TODO auf average_time_between_updates_readme eingehen
% TODO auf average_lifespans eingehen
