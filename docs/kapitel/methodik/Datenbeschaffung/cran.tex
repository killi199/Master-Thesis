\subsection{CRAN}
\label{subsec:datenbeschaffung_cran}
Ähnlich wie bei \gls{pypi} werden zu Beginn die Daten der Top 100 Pakete aus der zuvor beschafften Datei eingelesen.
In dieser Datei sind ebenfalls ausschließlich der Name des Pakets und die Anzahl der Downloads auf \gls{cran} enthalten.
Aus diesem Grund wird zu Beginn eine Anfrage mittels \emph{aiohttp} an die von METACRAN bereitgestellte API gestellt, um die Metadaten des Pakets zu erhalten.
In den Metadaten kann der Link eines GitHub Repositorys enthalten sein.
Falls kein Link zu einem GitHub Repository vorhanden ist, wird das Paket ebenfalls nicht weiter betrachtet.

Anschließend werden die weiteren Daten verarbeitet, welche von der API bereitgestellt werden.
Von der API wird das Feld \glqq Authors@R\grqq{} bereitgestellt.
Dieses Feld beinhaltet die Autoren mit dem Namen, der E-Mail und einer ORCID-ID des Pakets in einer in R Formatierten Zeichenfolge.
Dabei müssen nicht zwingend alle Informationen vorhanden sein.
Des weiteren haben Autoren eine Rolle zugeordnet.
Die Rolle ist nicht fest definiert und kann von den Autoren frei gewählt werden.
Es existieren allerdings Standards welche eingehalten werden sollten.
Einem Autor können mehrere Rollen zugewiesen sein.
Im R Journal wurden folgende Rollen definiert \autocite{hornik_who_2011}:

\begin{itemize}
    \item \glqq \emph{aut}\grqq{} (Autor): Vollständige Autoren, die wesentliche Beiträge zu dem Paket geleistet haben und in der Zitation des Pakets auftauchen sollten.
    \item \glqq \emph{com}\grqq{} (Complier): Personen, die Code (möglicherweise in anderen Sprachen) gesammelt, aber keine weiteren wesentlichen Beiträge zum Paket geleistet haben.
    \item \glqq \emph{ctb}\grqq{} (Mitwirkender): Autoren, die kleinere Beiträge geleistet haben (z. B. Code-Patches usw.), die aber nicht in der Auflistung der Autoren auftauchen sollten.
    \item \glqq \emph{cph}\grqq{} (Urheberrechtsinhaber): Personen, die das Urheberrecht an dem Paket besitzen.
    \item \glqq \emph{cre}\grqq{} (Maintainer): Paket Maintainer
    \item \glqq \emph{ths}\grqq{} (Betreuer der Thesis): Betreuer der Thesis, wenn das Paket Teil einer Thesis ist.
    \item \glqq \emph{trl}\grqq{} (Übersetzer): Übersetzer nach R, wenn der R-Code eine Übersetzung aus einer anderen Sprache (typischerweise S) ist.
\end{itemize}

Die Daten aus der Zeichenfolge werden mit der Software \emph{rpy2} in der Version 3.5.16 verarbeitet \autocite{noauthor_rpy2rpy2_2024}.
\emph{Rpy2} ist eine Software, welche es ermöglicht R-Code in Python auszuführen.
Die Software wird mit dem R Befehl \code{eval(parse(text = '\{cran\_author\}'))} ausgeführt, wobei \code{cran\_author} die R Zeichenfolge der Autoren ist.
Anschließend werden die Autoren, welche die Rolle \emph{aut} zugeordnet haben nach dem Abgleich mit den Git Daten in der Datei \texttt{cran\_authors.csv} gespeichert.
In der Datei wird, falls vorhanden, der Name, die E-Mail-Adresse und die ORCID-ID des Autors gespeichert.

% CRAN Authors (Author) (cran_authors) Inhalt ist Name ORCID
% CRAN Maintainers (Maintainer) (cran_maintainers) Inhalt ist Name Email
% CRAN Description (Description) (description_authors) Inhalt ist Name
