\subsection{PyPI}
\label{subsec:datenbeschaffung_pypi}
% TODO Daten aus PyPi werden benötigt um das GitHub repo zu finden -> sagen, dass die nicht immer vorhanden sind
% TODO erklären warum BigQuery nicht eingesetzt wird aktuell aber dennoch sinnvoll eingesetzt werden könnte bei anderen anforderungen (z.B. liste mit repo links auf github)
% TODO erklären warum die API verwendet wird und nicht nur die TOML beispielsweise ausgelesen wird -> habe ein pypi paket und brauche die GitHub URL
% TODO sagen warum beides also verifizierte Owner und Maintainer sowhl als auch die Daten aus der TOML abgefragt werden -> sind unterschiedliche Leute und werden unterschiedlich angegeben die einen dürfen sachen auf pypi machen die anderen werden in der toml angegeben und dürfen ggf. nichts in pypi machen
% Checken was verifizierte Nutzer im PyPI Universum bedeutet. Sind es nur verifizierte User die z.B. eine E-Mail hinterlegt haben oder sind es Nutzer die an dem Projekt arbeiten und von PyPI verifiziert sind? Falls es mehrwert hat diese Daten ebenfalls automatisch abfragen und auch in der MA beschreiben, was es nun ist und warum es Mehrwert hat oder auch nicht.
% TODO In MA beschreiben, warum oder warum nicht Mehrwert (PyPI Verifizierte Nutzer)
% TODO Sagen warum die Owner nicht berücksichtigt werden hatte dafür auch irgendwo eine Qulle -> es sind immer org.
% TODO erklären, dass ich den Namen des Betreuers brauche und nicht nur den Benutzernamen über die API und wie ich das gelöst habe mit einem WebScraper
% TODO erklären auf welchen branch ich untersuche ist es immer der Github default? oder nehme ich main? oder master? wie wird das entschieden?
% aiohttp==3.10.3, beautifulsoup4==4.12.3, spacy==3.7.6
