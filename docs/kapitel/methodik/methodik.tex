\chapter{Methodik}
\label{chap:methodik}
% TODO Grafik hier einfügen vom Prozess
In diesem Kapitel wird beschrieben, wie die Daten der einzelnen Quellen beschafft, abgeglichen und anschließend ausgewertet werden.
Die Datenbeschaffung wird in \autoref{sec:datenbeschaffung}, der Abgleich in \autoref{sec:abgleich} und die Auswertung in \autoref{sec:auswertung} beschrieben.

Die Datenbeschaffung in wurde in die einzelnen Quellen untergliedert.
Einige Methoden zur Datenbeschaffung sind dabei ähnlich, worauf im konkreten Fall eingegangen wird.

Der Abgleich findet jeweils zwischen Git und einer weiteren Quelle statt.
Es existiert kein Abgleich zwischen einzelnen Quellen wie den Daten aus \gls{pypi} und der Beschreibung.
Der Abgleich wird in jeder Datenbeschaffung außer der von Git automatisch durchgeführt.
Die Ergebnisse des Abgleichs werden in einer CSV-Datei gespeichert.
Die Datei wird nur erstellt, falls mindestens ein Eintrag enthalten ist.
Allgemeine Daten, beispielsweise ob die Quelle valide ist, werden ebenfalls in einer CSV-Datei gespeichert.
Falls in einer Quelle keine Daten vorhanden sind, wird keine CSV-Datei für diese erstellt.

Sämtliche Ergebnisse werden, falls verfügbar, zu verschiedenen Zeitpunkten in denen Änderungen an der Quelle vorgenommen wurden ermittelt und gespeichert.
Falls aus der Quelle verschiedene Zeitpunkte der Änderungen vorliegen, wird der Abgleich mit Git jeweils mit der neuesten Version durchgeführt und mit der Version, welche zu dem Zeitpunkt der Änderung in der Quelle vorhanden war.
Dadurch entstehen für ein Paket mehrere Dateien, welche unterschiedliche Werte enthalten.
Es entsteht die in \autoref{fig:datenbeschaffung_ergebnisse} dargestellte Ordnerstruktur, welche die Ergebnisse der Datenbeschaffung darstellt.

\begin{figure}
    \centering
    \dirtree{%
        .1 /\DTcomment{Wurzelverzeichnis}.
        .2 \path{20210819_161452-0400_bib_authors.csv}\DTcomment{\hologo{BibTeX} Autoren abgeglichen mit den Git-Werten zu diesem Zeitpunkt}.
        .2 \path{20210819_161452-0400_bib_authors_new.csv}\DTcomment{\hologo{BibTeX} Autoren abgeglichen mit den Git-Werten zum neuesten Zeitpunkt}.
        .2 \path{20210819_161452-0400_git_contributors.csv}\DTcomment{Git-Autoren bis zu diesem Zeitpunkt}.
        .2 \path{20221010_124020-0400_readme_authors.csv}\DTcomment{Autoren in der Beschreibung abgeglichen mit den Git-Werten zu diesem Zeitpunkt}.
        .2 \path{20221010_124020-0400_readme_authors_new.csv}\DTcomment{Autoren in der Beschreibung abgeglichen mit den Git-Werten zum neuesten Zeitpunkt}.
        .2 \path{20221010_124020-0400_git_contributors.csv}\DTcomment{Git-Autoren bis zu diesem Zeitpunkt}.
        .2 \path{bib.csv}\DTcomment{Allgemeine informationen zur \hologo{BibTeX}-Datei}.
        .2 \path{readme.csv}\DTcomment{Allgemeine Informationen zur Beschreibung}.
        .2 \path{git_contributors.csv}\DTcomment{Git-Autoren zum neuesten Zeitpunkt}.
        .2 \path{pypi_maintainers.csv}\DTcomment{\gls{pypi} Maintainer}.
        .2 \path{python_authors.csv}\DTcomment{In Python angegebene Autoren}.
    }
    \caption{Ergebnisse der Datenbeschaffung}
    \label{fig:datenbeschaffung_ergebnisse}
    \small
    \raggedright
    Die Abbildung stellt einen Ausschnitt der CSV-Dateien der Datenbeschaffung dar.
\end{figure}

Der Prozess findet für jedes zu untersuchende Paket statt.
Die Pakete werden dabei aus dem \gls{pypi} und dem \gls{cran} Software-Verzeichnis entnommen.
Die Ergebnisse des Prozesses werden in fünf Ordnern gespeichert jeweils für \gls{pypi}, \gls{cran}, \gls{cff}, \gls{pypi} \gls{cff} und \gls{cran} \gls{cff}.
Die Ordner stellen die 5 Top 100 Listen dar, welche in \autoref{sec:datenbeschaffung} beschrieben werden.
In jedem Ordner sind jeweils Unterordner für die einzelnen Pakete.
Diese Daten werden für die anschließende Auswertung in \autoref{sec:auswertung} verwendet.
Hier werden die Ergebnisse der Abgleiche ausgewertet und zusammengefasst, um über alle Pakete hinweg Aussagen treffen zu können.
Die Datenbeschaffung, der Abgleich und die Auswertung sind in Python programmiert und verwenden \gls{oss} auf welche in den jeweiligen Abschnitten eingegangen wird.
Der entwickelte Quellcode ist in einem Gitea-Repository verfügbar \autocite{jahrens_t20240710-softwareauthors-kj_2024}.
Über alle Abschnitte hinweg wird Pandas in der Version 2.2.2 verwendet, um die Tabellen zu erstellen und zu verarbeiten \autocite{the_pandas_development_team_pandas-devpandas_2024}.

\section{Datenbeschaffung}
\label{sec:datenbeschaffung}
\subsection{Git}
\label{subsec:git}
% TODO Beschreiben wie die Commits gezählt wurden.
\subsection{PyPi}
\label{subsec:pypi}
% TODO auf verifizierte PyPi Nutzer eingehen
% Checken was verifizierte Nutzer im PyPi Universum bedeutet. Sind es nur verifizierte User die z.B. eine E-Mail hinterlegt haben oder sind es Nutzer die an dem Projekt arbeiten und von PyPi verifiziert sind? Falls es mehrwert hat diese Daten ebenfalls automatisch abfragen und auch in der MA beschreiben, was es nun ist und warum es Mehrwert hat oder auch nicht.
% In MA beschreiben, warum oder warum nicht Mehrwert (PyPi Verifizierte Nutzer)
% Verifizierte Daten sind Daten, welche PyPi als "richtig" ansieht. Aktuell ist es im Prozess, dass immer mehr Daten diesen Status erreichen. Aktuell können nur der Owner und Maintainer diesen Status erreichen. Verifizierte Maintainer sind dabei Personen, welche bei PyPi registriert sind und dem Projekt zugeordnet sind. Also das Projekt auf PyPi verwalten können.
% https://github.com/pypi/warehouse/issues/8635
% https://github.com/pypi/warehouse/issues/15903
% https://peps.python.org/pep-0740/
% https://github.com/pypi/warehouse/issues/15871 (Roadmap Pep 740)
% Aktuell gibt es noch keinen weg die verified details über die API zu beziehen (https://github.com/pypi/warehouse/issues/14799) aus diesem Grund muss es aus dem HTML code extrahiert werden.
% Verifizierte "Maintainer" in PyPi sind Personen, welche neue Releases veröffentlichen dürfen (https://github.com/pypi/warehouse/issues/13366)
% Die nicht verifizierten Daten stammen aus der Python toml
% Es existiert aktuell keine API für user (https://github.com/pypi/warehouse/issues/15769) aus diesem Grund muss ich die aus dem HTML besorgen.
\subsection{CRAN}
\label{subsec:cran}
\subsection{Beschreibung}
\label{subsec:beschreibung}
% TODO Es werden alle Namen ausgegeben aber auch eben solche, die gar nichts mit dem Paket zu tun haben wie im fall von highr für CRAN: "Provides syntax highlighting for R source code. Currently it supports LaTeX and HTML output. Source code of other languages is supported via Andre Simon's highlight package (https://gitlab.com/saalen/highlight)." Es gibt noch weitere Beispiele z.B. in CRAN magrittr
\subsection{Citation File Format}
\label{subsec:cff}
\subsection{BibTeX}
\label{subsec:bibtex}

\section{Abgleich}
\label{sec:abgleich}
% TODO Grafik erstellen, die das ganze nochmal visualisiert
In diesem Abschnitt wird beschrieben, wie die Autoren aus den unterschiedlichen Quellen, welche in der Datenbeschaffung erläutert wurden, abgeglichen werden.
Dabei werden jeweils die Git Autoren mit den Autoren aus den anderen Quellen abgeglichen.
Für diesen Prozess wird keine \gls{ned} verwendet, sondern ein eigener Algorithmus, welcher auf die Daten angepasst ist.
Dieser wurde entwickelt, indem die Autoren, welche durch die Datenbeschaffung erhalten wurden, in den Paketen eingesehen wurden.
So wurde durch viel Probieren eine möglichst gute Lösung gefunden, welche im Folgenden beschrieben wird.
Dies führt allerdings dazu, dass der entwickelte Algorithmus nicht in der Lage ist zwei Autoren mit dem gleichen Namen zu unterscheiden, falls keine weiteren Daten wie eine E-Mail vorhanden sind.

Als Eingabe erhält der Algorithmus eine Liste von Autoren, welche in einer Quelle (z. B. \gls{cff}) gefunden wurden und jeweils die Liste der Git Autoren für das zu untersuchende Paket.
In der Datenbeschaffung wurde gezeigt, dass die Autoren aus den Quellen einen Namen, eine E-Mail und einen Benutzernamen enthalten können, abhängig von der Quelle.
In dem Algorithmus wird ermittelt, wie viele von den Daten vorhanden sind, um im Verlauf einen Score berechnen zu können wie viele der Daten übereinstimmen.
Falls Beispielsweise die Quelle \gls{cff} ist, kann der Name und die E-Mail vorhanden sein.
Dadurch ergibt sich, dass maximal zwei Daten mit denen von Git übereinstimmen können.
Falls die Quelle die Beschreibung ist, ist maximal eine Übereinstimmung möglich.
Diese Daten werden für den Abgleich verwendet.
Die eingegebene Liste der Git Autoren muss für den Algorithmus nach der Anzahl der Commits sortiert sein, da der Algorithmus Personen mit den meisten Commits bevorzugt.
Anschließend wird jeder Autor aus der Quelle mit jedem Git Autor abgeglichen.
Dabei werden alle Werte in Kleinbuchstaben umgewandelt, um zu gewährleisten, dass auch bei unterschiedlicher Schreibweise eine Übereinstimmung gefunden werden kann.

Dabei wird so vorgegangen, dass die Daten, welche vorhanden sind, zum Vergleich herangezogen werden.
Falls der Name vorhanden ist, wird dieser mit dem Namen des Git Autors über das Keyword \emph{in} in Python verglichen.
Dieses sorgt dafür, dass nur ein Teil der Zeichenkette in der anderen Zeichenkette vorhanden sein muss.
Außerdem wird die andere Richtung ebenfalls probiert.
Zusätzlich wird das Programm \emph{thefuzz} in der Version 0.22.1 verwendet, um eine unscharfe Suche zu ermöglichen.
Hierbei wird der Name des Autors aus der Quelle mit dem Namen des Git Autors verglichen und auf eine Übereinstimmung von 80 \% oder mehr geprüft.
Die Datenbeschaffung hat gezeigt, dass die Namen der Autoren manchmal Klammern enthalten, welche den beschriebenen Abgleich stören.
Aus diesem Grund wird der zuvor beschriebene Abgleich erneut durchgeführt, wobei die Klammern in dem Namen entfernt worden sind.

Falls eine E-Mail vorhanden ist, wird diese mit der E-Mail des Git Autors über das Keyword \emph{in} verglichen.
Die andere Richtung wird ebenfalls probiert, also die E-Mail des Git Autors mit der E-Mail des Autors aus der Quelle über das Keyword \emph{in} zu vergleichen.
Zusätzlich wird erneut die unscharfe Suche zwischen der E-Mail des Autors aus der Quelle und der E-Mail des Git Autors durchgeführt und auf eine Übereinstimmung von 80 \% oder mehr geprüft.
Falls kein Abgleich über den Namen stattgefunden hat, wird ein Vergleich zwischen der E-Mail des Autors aus der Quelle und dem Namen des Git Autors über das Keyword \emph{in} durchgeführt.
Außerdem wird der Abgleich erneut über die unscharfe Suche mit einer Übereinstimmung von 80 \% oder mehr durchgeführt.
Dies führt dazu, dass weitere Übereinstimmungen gefunden werden können, falls der Name zu keiner Übereinstimmung geführt hat.
Dies liegt daran, dass viele Autoren in ihrer E-Mail-Adresse ihren Namen enthalten haben.
Außerdem hat die Datenbasis gezeigt, dass es vorkommt, dass in dem E-Mail Feld keine E-Mail angegeben wurde, sondern ein Name und als Resultat das Namensfeld nicht ausgefüllt wurde.

Falls ein Benutzername vorhanden ist, wird dieser mit der E-Mail des Git Autors verglichen, da aus den Git Daten kein Benutzername extrahiert werden kann.
Dies ist möglich, da viele Autoren für ihren Benutzernamen den lokal Teil der E-Mail verwenden.
Für den Vergleich wird die E-Mail, falls sie ein @ Symbol enthält, an dieser Stelle getrennt.
Anschließend wird der vordere lokale Teil für den Vergleich mit dem Benutzernamen aus der Quelle verwendet.
Dabei wird der Vergleich erneut über das Keyword \emph{in} in beiden Richtungen durchgeführt.
Außerdem wird über die unscharfe Suche mit einer Übereinstimmung von 80 \% oder mehr geprüft, ob eine Übereinstimmung gefunden wurde.
Dieser Prozess wird ebenfalls erneut für den Domain-Teil der E-Mail durchgeführt, sodass dieser ebenfalls mit dem Benutzernamen verglichen wird.

Falls keiner dieser drei Vergleiche zu einem Erfolg geführt hat, wird der Name des Autors mit dem lokalen Teil der E-Mail des Git Autors über die unscharfe Suche verglichen.
Dabei muss erneut eine Übereinstimmung von 80 \% oder mehr vorhanden sein, um den Vergleich als gelungen zu bewerten.

Anschließend wird der Score berechnet.
Dabei wird die Anzahl der Übereinstimmungen durch die Anzahl der maximal möglichen Übereinstimmungen geteilt.
Falls beispielsweise der Name und die E-Mail vorhanden sind, sind maximal zwei Übereinstimmungen möglich.
Bei einer Übereinstimmung des Namens und die E-Mail ergibt sich ein Score von 1.
Falls nur der Name übereinstimmt, ergibt sich ein Score von 0,5.
Dieser Score wird für jeden Autor aus der Quelle mit jedem Git Autor berechnet und in der Tabelle der Autoren aus der Quelle gespeichert.
Anschließend wird der Autor aus der Quelle mit dem Git Autor mit dem besten Score ausgewählt.
Falls es zwei Einträge in der Git Liste gibt, welche einen gleichen Score erreichen, wird der Autor ausgewählt, welcher die meisten Commits hat.
Außerdem wird der Rang, welcher der Autor in der Git Autoren Liste belegt zurückgegeben.
Als Ergebnis des Abgleichs werden alle Autoren aus der Quelle, welche in den Algorithmus eingegeben wurde, zurückgegeben.
Die Autoren werden dabei mit den Ergebnissen des Abgleichs verbunden, welche in \autoref{tab:abgleich_felder} dargestellt sind.
Das Ergebnis des Abgleichs ist in \autoref{tab:abgleich_felder} dargestellt.
Falls kein Abgleich für einen Autor möglich war, werden die Felder für diesen Autor leer gelassen.
Anschließend wird das Ergebnis nach dem Rang sortiert, also nach der Anzahl der Commits, welche der Autor hat.

\section{Auswertung}
\label{sec:auswertung}
In diesem Abschnitt wird beschrieben, wie die Daten aus dem \autoref{sec:datenbeschaffung} ausgewertet werden.
Dabei wird der Abschnitt aufgeteilt in die Aggregierung der Daten, welche zuvor beschrieben wurden und anschließend wird kurz auf die Darstellung der aggregierten Daten eingegangen.

Die Auswertung der Daten erfolgt erneut mittels Python.
Für die Auswertung werden alle Daten verarbeitet, welche durch die Datenbeschaffung gesammelt wurden.
Einzige Ausnahme ist, dass immer die Dateien mit der \emph{new} Endung betrachtet werden.
Dies liegt daran, dass die Auswertung aktuell nur auf den neuesten Git-Daten durchgeführt wird und die Git-Daten zu den jeweiligen Zeitpunkten der Änderung nicht betrachtet werden.
Außerdem ist darüber ein Abgleich zwischen den Git-Autoren und den Autoren aus den anderen Quellen möglich.
Aber auch ein Abgleich zwischen den einzelnen Quellen ist möglich, da die Autoren in der neuesten Version immer die gleichen Git-Daten wie beispielsweise die gleiche Anzahl an Commits in allen Quellen haben.
Der Abgleich der Autoren ist hierbei nur möglich, falls in der Datenbeschaffung die Autoren abgeglichen werden konnten.

Falls ein Abgleich zwingend erforderlich ist, beispielsweise bei der Untersuchung, wie lange ein Autor durchschnittlich in einer Quelle genannt ist, wird ein einfacher Vergleich des Namens über die Quelle durchgeführt.
Dieser Vergleich ist jedoch nicht immer korrekt, da beispielsweise Namensänderungen in den Quellen so nicht berücksichtigt werden können, falls kein Abgleich in der Datenbeschaffung möglich war.

Das Skript ist so aufgebaut, dass nicht alle Daten auf einmal geladen werden, sondern jede Datei einzeln geladen und verarbeitet wird.
Dies bietet den Vorteil, dass auch mit weniger Arbeitsspeicher gearbeitet werden kann und die Daten nicht alle auf einmal im Arbeitsspeicher gehalten werden müssen.
Jedoch birgt dies auch Nachteile, beispielsweise dass zu keinem Zeitpunkt bekannt ist, wie viele Autoren maximal in einer Quelle vorkommen.
Diese Information ist erst vorhanden, wenn alle Daten vollständig verarbeitet worden sind.

Zusätzlich ist die Aggregierung der Daten unterteilt in die Aggregierung der vollständigen Daten und die Aggregierung der neuesten Daten.
In der vollständigen Datenverarbeitung werden Ergebnisse extrahiert, die die gesamte Historie der Daten betrachten.
Hier werden alle Daten mit ihren Zeitstempeln verarbeitet.
In der neuesten Datenverarbeitung werden nur die Daten betrachtet, die auf dem neuesten Stand der untersuchten Software sind.
Dabei wird jeweils die neueste Datei in den Quellen mithilfe des Zeitstempels ausgewählt und anschließend verarbeitet.
In diese beiden Kategorien ist das \autoref{chap:ergebnisse} ebenfalls unterteilt.

