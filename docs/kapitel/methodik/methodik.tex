\chapter{Methodik}
\label{chap:methodik}
In diesem Kapitel wird beschrieben, wie die Daten der einzelnen Quellen beschafft, abgeglichen und anschließend ausgewertet werden.
Die Datenbeschaffung wird in \autoref{sec:datenbeschaffung}, der Abgleich in \autoref{sec:abgleich} und die Auswertung in \autoref{sec:auswertung} beschrieben.

Die Datenbeschaffung wurde in die einzelnen Quellen untergliedert.
Einige Methoden zur Datenbeschaffung sind dabei ähnlich, worauf im konkreten Fall eingegangen wird.

Der Abgleich findet jeweils zwischen Git und einer weiteren Quelle statt.
Es existiert kein Abgleich zwischen einzelnen Quellen wie den Daten aus \gls{pypi} und der Beschreibung.
Der Abgleich wird in jeder Datenbeschaffung außer der von Git automatisch durchgeführt.
Die Ergebnisse des Abgleichs werden in einer CSV-Datei gespeichert.
Die Datei wird nur erstellt, falls mindestens ein Eintrag enthalten ist.
Allgemeine Daten, beispielsweise ob die Quelle valide ist, werden ebenfalls in einer CSV-Datei gespeichert.
Falls in einer Quelle keine Daten vorhanden sind, wird keine CSV-Datei für diese erstellt.

Sämtliche Ergebnisse werden, falls verfügbar, zu verschiedenen Zeitpunkten in denen Änderungen an der Quelle vorgenommen wurden ermittelt und gespeichert.
Falls aus der Quelle verschiedene Zeitpunkte der Änderungen vorliegen, wird der Abgleich mit Git jeweils mit der neuesten Version durchgeführt und mit der Version, welche zu dem Zeitpunkt der Änderung in der Quelle vorhanden war.
Dadurch entstehen für ein Paket mehrere Dateien, welche unterschiedliche Werte enthalten.
Es entsteht die in \autoref{fig:datenbeschaffung_ergebnisse} dargestellte Ordnerstruktur, welche die Ergebnisse der Datenbeschaffung darstellt.

\begin{figure}
    \dirtree{%
        .1 /\DTcomment{Wurzelverzeichnis}.
        .2 \path{20210819_161452-0400_bib_authors.csv}\DTcomment{\hologo{BibTeX} Autoren abgeglichen mit den Git-Werten zu diesem Zeitpunkt}.
        .2 \path{20210819_161452-0400_bib_authors_new.csv}\DTcomment{\hologo{BibTeX} Autoren abgeglichen mit den Git-Werten zum neuesten Zeitpunkt}.
        .2 \path{20210819_161452-0400_git_contributors.csv}\DTcomment{Git-Autoren bis zu diesem Zeitpunkt}.
        .2 \path{20221010_124020-0400_readme_authors.csv}\DTcomment{Autoren in der Beschreibung abgeglichen mit den Git-Werten zu diesem Zeitpunkt}.
        .2 \path{20221010_124020-0400_readme_authors_new.csv}\DTcomment{Autoren in der Beschreibung abgeglichen mit den Git-Werten zum neuesten Zeitpunkt}.
        .2 \path{20221010_124020-0400_git_contributors.csv}\DTcomment{Git-Autoren bis zu diesem Zeitpunkt}.
        .2 \path{bib.csv}\DTcomment{Allgemeine informationen zur \hologo{BibTeX}-Datei}.
        .2 \path{readme.csv}\DTcomment{Allgemeine Informationen zur Beschreibung}.
        .2 \path{git_contributors.csv}\DTcomment{Git-Autoren zum neuesten Zeitpunkt}.
        .2 \path{pypi_maintainers.csv}\DTcomment{\gls{pypi} Maintainer}.
        .2 \path{python_authors.csv}\DTcomment{In Python angegebene Autoren}.
    }
    \caption{Ergebnisse der Datenbeschaffung}
    \label{fig:datenbeschaffung_ergebnisse}
    \small
    Die Abbildung stellt einen Ausschnitt der CSV-Dateien der Datenbeschaffung dar.
\end{figure}

Der Prozess findet für jedes zu untersuchende Paket statt und ist in \autoref{fig:programmablauf} visualisiert.
Die Pakete stammen aus den Software-Verzeichnissen \gls{pypi} und \gls{cran}.
Die Ergebnisse des Prozesses werden in fünf Ordnern gespeichert jeweils für \gls{pypi}, \gls{cran}, \gls{cff}, \gls{pypi} \gls{cff} und \gls{cran} \gls{cff}.
Diese Ordner bilden die fünf Top-100-Listen, die in \autoref{sec:datenbeschaffung} näher beschrieben sind.
In jedem Ordner sind jeweils Unterordner für die einzelnen Pakete.
Diese Daten werden für die anschließende Auswertung in \autoref{sec:auswertung} verwendet.
In diesem Abschnitt werden die Ergebnisse der Abgleiche analysiert und zusammengefasst, um Aussagen über alle Pakete hinweg zu treffen.
Die Datenbeschaffung, der Abgleich und die Auswertung sind in Python programmiert und verwenden \gls{oss} auf welche in den jeweiligen Abschnitten eingegangen wird.
Der entwickelte Quellcode ist in einem Git-Repository verfügbar \autocite{jahrens_t20240710-softwareauthors-kj_2025}.
Über alle Abschnitte hinweg wird Pandas in der Version 2.2.2 verwendet, um die Tabellen zu erstellen und zu verarbeiten \autocite{the_pandas_development_team_pandas_2024}.

\begin{figure}
    \begin{center}
        \includesvg[width=.95\linewidth,inkscapelatex=false]{bilder/Ablauf.svg}
    \end{center}
    \caption{Visualisierter Prozess}
    \label{fig:programmablauf}
    \small
    In der Abbildung ist der Prozess visualisiert. Die runden Knoten stellen Prozesse dar. Rautenförmige Knoten repräsentieren Software-Verzeichnisse und rechteckige Knoten repräsentieren die einzelnen zu untersuchenden Quellen, in welchen Autoren genannt werden können. Kanten stellen den Informationsfluss dar, wobei die Kantenbeschriftung die jeweiligen Informationen repräsentiert.
\end{figure}

\section{Datenbeschaffung}
\label{sec:datenbeschaffung}
% TODO MEINE SOFTWARE ZITIEREN! Sonst bin ich nicht besser als alle anderen und auch Probleme und Konfiguration darstellen -> steht in dem Paper smith_software_2016 wie zitiert werden soll? \autocite{richardson_beautifulsoup4_2024}
In diesem Abschnitt wird beschrieben wie das Skript zur Datenbeschaffung aus den einzelnen Quellen aufgebaut ist.
Es wird tqdm in der Version 4.66.5 verwendet, um den Fortschritt der Datenbeschaffung anzuzeigen \autocite{costa-luis_tqdm_2024}.
Die Datenbeschaffung wird in die einzelnen Quellen untergliedert.

Aus den Quellen \nameref{subsec:datenbeschaffung_git}, \nameref{subsec:datenbeschaffung_beschreibung}, \nameref{subsec:datenbeschaffung_cff} und \nameref{subsec:datenbeschaffung_bibtex} können zeitliche Informationen extrahiert werden, da diese in Git verwaltet werden.
Aus diesem Grund werden die Daten jeweils zu der Änderung der Quelle gespeichert.
Dabei ist die maximale Anzahl der Änderungen in die Vergangenheit auf 50 beschränkt, um die Laufzeit des Skripts zu begrenzen.

In \nameref{subsec:datenbeschaffung_cran} ist es nicht möglich die Änderungen in der Zeit zu betrachten.
In der \nameref{subsec:datenbeschaffung_pypi} Quelle ist es teilweise möglich die Änderungszeitpunkte zu erhalten, jedoch ist dies mit Kosten verbunden und erfordert eine andere Vorgehensweise als bei den anderen zeitlichen Daten, da diese nicht direkt aus Git extrahiert werden können.
Die beiden Quellen werden aus diesem Grund nur in der neusten Version betrachtet und enthalten keine Änderungshistorie.

\subsection{Git} % 2 Seiten
\label{subsec:datenbeschaffung_git}
% TODO Beschreiben wie die Commits gezählt wurden. -> Mit oder ohne merges?
% TODO email in komplett kleinschreibung umgewandelt anschließend grouby auf email -> kurz erwähnen weil leute frei entscheiden können was sie angeben
Die Git Daten sind die grundlegenden Daten, welche für die weiteren Schritte benötigt werden.
Sämtliche anderen Quellen werden mit den Git Daten über den in \autoref{sec:abgleich} beschriebenen Prozess abgeglichen.
Zu Beginn muss das Repository von GitHub geclont werden, um die Daten Lokal verarbeiten zu können.
Dabei kommt die \gls{oss} GitPython in der Version 3.1.43 zum Einsatz, welches eine Bibliothek für die einfache Interaktion mittels Python und Git Befehlen darstellt \autocite{thiel_gitpython-developersgitpython_2024}.
Für das Clonen wird der Link zum GitHub Repository benötigt, welcher aus der \gls{pypi} oder \gls{cran} Quelle stammt.
Auf diese Quellen wird in dem \autoref{subsec:datenbeschaffung_pypi} und \autoref{subsec:datenbeschaffung_cran} eingegangen.

Die Auswertung des Repositorys wird mit git-quick-stats in der Version 2.3.0 durchgeführt \autocite{arzzen_git-quick-statsgit-quick-stats_2021}.
Git-Quick-stats bietet einfache und effiziente Möglichkeiten um verschiedene Statistiken in einem Git Repository zu ermitteln.
Das Tool wird in dem Python Skript mit dem Befehl \texttt{git-quick-stats -T} aufgerufen, um detaillierte Statistiken zu erhalten.
Ausgegeben wird eine Liste aller Autoren, welche in dem Repository Änderungen vorgenommen haben.
Diese Liste enthält unter anderem den Namen, die E-Mail, die Anzahl der Einfügungen, Löschungen, geänderten Zeilen, Dateien, Commits, sowie den ersten und letzten Commit.
Die Daten werden in der Datei \texttt{git\_contributors.csv} gespeichert.

Außerdem bietet das Tool die Möglichkeit mit der gesetzten Umgebungsvariablen \textit{\_GIT\_UNTIL=}, alle Änderungen nur bis zu einem bestimmten Zeitpunkt zu betrachten.
Diese Funktion wird verwendet, um die Änderungen bis zu der Aktualisierung einer Quelle zu betrachten.
Die Daten werden beispielsweise in der Datei \texttt{20210819\_161452-0400\_git\_contributors.csv} gespeichert, wobei der erste Teil des Dateinamens den konkreten Tag und Uhrzeit mit zugehöriger Zeitzone angibt.
Für die Verarbeitung der Zeiten in unterschiedlichen Zeitzonen wird das Modul pytz in der Version 2024.2 verwendet \autocite{bishop_stub42pytz_2024}.

\subsection{PyPI} % 2 Seiten
\label{subsec:datenbeschaffung_pypi}
% TODO Daten aus PyPi werden benötigt um das GitHub repo zu finden -> sagen, dass die nicht immer vorhanden sind
% TODO erklären warum BigQuery nicht eingesetzt wird aktuell aber dennoch sinnvoll eingesetzt werden könnte bei anderen anforderungen (z.B. liste mit repo links auf github)
% TODO erklären warum die API verwendet wird und nicht nur die TOML beispielsweise ausgelesen wird -> habe ein pypi paket und brauche die GitHub URL
% TODO sagen warum beides also verifizierte Owner und Maintainer sowhl als auch die Daten aus der TOML abgefragt werden -> sind unterschiedliche Leute und werden unterschiedlich angegeben die einen dürfen sachen auf pypi machen die anderen werden in der toml angegeben und dürfen ggf. nichts in pypi machen
% Checken was verifizierte Nutzer im PyPI Universum bedeutet. Sind es nur verifizierte User die z.B. eine E-Mail hinterlegt haben oder sind es Nutzer die an dem Projekt arbeiten und von PyPI verifiziert sind? Falls es mehrwert hat diese Daten ebenfalls automatisch abfragen und auch in der MA beschreiben, was es nun ist und warum es Mehrwert hat oder auch nicht.
% TODO In MA beschreiben, warum oder warum nicht Mehrwert (PyPI Verifizierte Nutzer)
% TODO Sagen warum die Owner nicht berücksichtigt werden hatte dafür auch irgendwo eine Qulle -> es sind immer org.
% TODO erklären, dass ich den Namen des Betreuers brauche und nicht nur den Benutzernamen über die API und wie ich das gelöst habe mit einem WebScraper
% TODO erklären auf welchen branch ich untersuche ist es immer der Github default? oder nehme ich main? oder master? wie wird das entschieden?
% aiohttp==3.10.3, beautifulsoup4==4.12.3, spacy==3.7.6

\subsection{CRAN} % 2 Seiten
\label{subsec:datenbeschaffung_cran}
% TODO Daten aus CRAN werden benötigt um das GitHub repo zu finden -> sagen, dass die nicht immer vorhanden sind
% TODO rpy2==3.5.16, aiohttp==3.10.3

\subsection{Beschreibung} % 2 Seiten
\label{subsec:datenbeschaffung_beschreibung}
% TODO Es werden alle Namen ausgegeben aber auch eben solche, die gar nichts mit dem Paket zu tun haben wie im fall von highr für CRAN: "Provides syntax highlighting for R source Code. Currently it supports LaTeX and HTML output. Source Code of other languages is supported via Andre Simon's highlight package (https://gitlab.com/saalen/highlight)." Es gibt noch weitere Beispiele z.B. in CRAN magrittr -> vllt eher ergebnis?
% spacy==3.7.6

\subsection{Citation File Format} % 2 Seiten
\label{subsec:datenbeschaffung_cff}
% TODO darauf eingehen, dass ich händisch aktuell geprüft habe ob das GH repo in cran oder pypi ist und daraus meine listen gebaut habe. Dies wäre aber auch automatisch möglich aber mit viel Aufwand und man bräuchte google big query für 100 pakete war dieser weg schneller außerdem gibt es auf PyPi mehrere Pakete die auf ein GitHub repo verlinken. Da müsste einiges beachtet werden um das genau zu machen für 100 war dies besser. Falls es für mehr gemacht werden sollte müsste das überdacht werden.
% TODO darauf eingehen wie die CFF geholt und verarbeitet wurde mit stars verknüpft und dann händisch pypi cran raus gesucht und den namen auf pypi
% TODO in der gegebenen Liste waren mehrere Repos auf GH doppelt mit unterschiedlichen Links. diese wurden entfernt.
% TODO sagen, dass nur die erste identifier-doi beschafft wird, da nur geguckt wird in den Ergebnissen ob überhaupt eine vorhanden ist
% pyyaml==6.0.2, cffconvert==2.0.0, jsonschema==4.23.0, pykwalify==1.8.0
% CFF und preferred citation getrennt

\subsection{\hologo{BibTeX}} % 2 Seiten
\label{subsec:datenbeschaffung_bibtex}
% bibtexparser @ git+https://github.com/sciunto-org/python-bibtexparser@main

\section{Abgleich}
\label{sec:abgleich}
% TODO Erwähnen, dass immer der mit dem besten Score ausgewählt wird. Und falls beide den gleichen Score dann der mit mehr commits.
% TODO Es wird immer passieren, dass manche Leute falsch zugeordnet werden. Aktuelles bsp. in scipy heißt jemand pv auf PyPI und jemand hat eine E-Mail mit: pvanmulbregt@users.noreply.github.com die werden gematcht... Also nur weil ich einen Match habe bedeutet es nicht, dass es ein richtiger ist! Dies unbedingt in der Arbeit berücksichtigen. Ebenfalls, dass falls kein match gefunden wurde heißt es nicht zwangsweise, dass mein Script schlecht arbeitet es kann auch sein, das jemand als Autor genannt wird aber keinen Code geschrieben hat oder das eine Organisation als Autor angegeben wurde.
% TODO beschreiben, dass all contributors ein Problem im  matching darstellen, da autoren ohne commits natürlich nicht mit den contributorn von github gematch werden können -> daher händisch durchgegangen für 200 pakete und geguckt welche nicht gemachts werden können um gefühlt für die dimension zu erhalten -> auf Ergebnis nochmal referenzieren wo die daten angegeben sind

\section{Auswertung} % 2 Seiten
\label{sec:auswertung}

