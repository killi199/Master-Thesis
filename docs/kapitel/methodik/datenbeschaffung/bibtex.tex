\subsection{\hologo{BibTeX}}
\label{subsec:datenbeschaffung_bibtex}
In diesem Unterabschnitt wird wie bereits zuvor eine Datei untersucht, welche in Git verwaltet wird.
Hierbei wird die \path{CITATION.bib} Datei zu jedem Änderungszeitpunkt untersucht, falls sie in dem zu untersuchenden Paket vorhanden ist.
Es wird die Software \emph{bibtexparser} verwendet, um die \hologo{BibTeX} Datei zu analysieren.
Die Software bietet Möglichkeiten, um in Python \hologo{BibTeX} Dateien zu lesen und zu schreiben.
Die Software wird nicht in einer speziellen Version verwendet, da Features, welche benötigt werden noch nicht im offiziellen Release vorhanden sind.
Aus diesem Grund wird das Programm direkt aus dem Git Repository geladen und mit dem letzten Commit zum Zeitpunkt des Schreibens ausgeführt.
Die verwendete Commit-ID ist: 214ef38eabd60538cc96349bc31d219897029f70.

Die Autoren werden aus der \hologo{BibTeX} Datei extrahiert und mit den neusten Git Daten abgeglichen.
Die erstellte Liste wird in der Datei \path{TIMESTAMP_bib_authors_new.csv} gespeichert, wobei \emph{TIMESTAMP} durch den Zeitpunkt der Änderung ersetzt wird.
Dabei wird in der Datei ausschließlich der Name des Autors zuzüglich der Daten aus dem Abgleich mit Git gespeichert.

Ebenfalls wird erneut der Git Prozess aus \autoref{subsec:datenbeschaffung_git} bis zu diesem Zeitpunkt durchlaufen.
Die Liste der Autoren wird in der Datei \path{TIMESTAMP_git_contributors.csv} gespeichert.
Der Inhalt ist identisch, wie bereits in den anderen Quellen beschrieben.
Außerdem wird die Liste der Autoren, welche aus der \hologo{BibTeX} extrahiert wurden, mit den Git Autoren bis zu diesem Zeitpunkt abgeglichen.
Die Daten werden in der Datei \path{TIMESTAMP_bib_authors.csv} gespeichert.
Hierbei wird ebenfalls ausschließlich der Name des Autors und die Daten aus dem Git Abgleich gespeichert.

Außerdem können erneut allgemeine Daten für die \hologo{BibTeX} Datei extrahiert werden.
Hierbei wird eine weitere CSV-Datei erstellt, welche die allgemeinen Daten enthält.
Die Datei wird \path{bib.csv} genannt und enthält wie bereits in den anderen Quellen die Zeitpunkte \emph{commit date} und \emph{author date}.
Außerdem wird der Literaturtyp aus der Datei extrahiert und in der Datei gespeichert.
Die Literaturtypen können dabei unterschiedlich sein, verschiedene Typen sind in den Grundlagen erläutert.
Zusätzlich wird das Jahr und der Monat der Veröffentlichung extrahiert und gespeichert.
Diese Werte müssen nicht für jeden Literaturtypen vorhanden sein und können aus diesem Grund leer sein.
Ebenfalls müssen die DOI und die ISBN nicht für jeden Literaturtypen vorhanden sein, falls sie jedoch vorhanden sind, werden sie extrahiert und gespeichert.
