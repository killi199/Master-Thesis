\subsection{Beschreibung}
\label{subsec:datenbeschaffung_beschreibung}
In diesem Kapitel wird anders als zuvor keine API angefragt, sondern auf den bereits heruntergeladenen Git Daten die Analyse durchgeführt.
Untersucht wird die Beschreibung, welche in der \texttt{README.md} Datei des Repositorys enthalten ist.
Dabei wird die Beschreibung wie zuvor mit \emph{spaCy} untersucht.
Anders ist jedoch, dass die \texttt{README.md} Datei in Git verwaltet wird und somit die Änderungen in der Zeit betrachtet werden können.
Aus diesem Grund wird das Programm \emph{GitPython} verwendet, um die letzten 50 Änderungen der Datei auf dem Standardbranch zu erhalten.
Anschließend wird für jede Änderung die Beschreibung wie zuvor mit \emph{spaCy} analysiert.
Für jeden Zeitpunkt wird eine CSV-Datei mit dem Namen \texttt{TIMESTAMP\_readme\_authors\_new.csv} erstellt, wobei \emph{TIMESTAMP} durch den konkreten Zeitpunkt der Änderung der Datei ersetzt wird.
Hierbei wird der \emph{commit date} Zeitpunkt aus Git verwendet.
Die erstellte Datei enthält die extrahierten Namen aus der Beschreibung, welche mit den neusten Git Daten abgeglichen wurden.
Der neuste Zeitpunkt entspricht hierbei dem Zeitpunkt, an dem die Repositorys heruntergeladen wurden, also der Tag an dem der Prozess durchlaufen wurde.

Zusätzlich wird zu jedem Änderungszeitpunkt der Git Prozess aus \autoref{subsec:datenbeschaffung_git} bis zu diesem Zeitpunkt durchlaufen.
Dies dient dazu, eine Liste aller Autoren zu erhalten, welche bis zu diesem Zeitpunkt Änderungen vorgenommen haben.
Die Liste wird in der Datei \texttt{TIMESTAMP\_git\_contributors.csv} gespeichert, wobei \emph{TIMESTAMP} durch den konkreten Zeitpunkt ersetzt wird.
Der Inhalt ist der gleiche wie in \autoref{subsec:datenbeschaffung_git}, jedoch nur bis zu dem Zeitpunkt, an dem die \texttt{README.md} Datei jeweils geändert wurde.
Diese Daten werden ebenfalls mit den Autoren, welche aus der Beschreibung extrahiert wurden, abgeglichen und in der Datei \texttt{TIMESTAMP\_readme\_authors.csv} gespeichert.
Inhalt ist hierbei der Name des Autors und weitere Informationen, welche durch den Abgleich ermittelt wurden.

Im Prozess werden ebenfalls allgemeine Informationen zur Beschreibung beschafft und anschließend in der Datei \emph{readme.csv} gespeichert.
Die Datei beinhaltet die Zeitpunkte \emph{commit date} und \emph{author date} für jede Änderung der Beschreibung.
