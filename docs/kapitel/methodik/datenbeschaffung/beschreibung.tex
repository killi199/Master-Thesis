\subsection{Beschreibung}
\label{subsec:datenbeschaffung_beschreibung}
% TODO darauf eingehen, dass der Git Prozess hier wieder für dein Zeitpunkt bis zur aktualisierung der Quelle angestoßen wird und eine Datei 2021.0819_16 .. (siehe einleitung) erstellt wird -> vllt einleitung ändern auf Beschreibung -> oder noch besser vllt einfach zu bib auch noch die beschreibungsdateien hinzufügen
% TODO wie beschreibung ausgewertet wird wurde schon etwas beschrieben wichtig ist hier wie das mit git funktioniert in den unterschiedlichen Zeitpunkten
In diesem Kapitel wird anders als zuvor keine API angefragt, sondern auf den bereits heruntergeladenen Git Daten die Analyse durchgeführt.
Untersucht wird die Beschreibung, welche in der \texttt{README.md} Datei des Repositorys enthalten ist.
Dabei wird die Beschreibung wie zuvor mit \emph{spaCy} untersucht.
Anders ist jedoch, dass die \texttt{README.md} Datei in Git verwaltet wird und somit die Änderungen in der Zeit betrachtet werden können.
Aus diesem Grund wird das Programm \emph{GitPython} verwendet, um die letzten 50 Änderungen der Datei auf dem Standardbranch zu erhalten.
Anschließend wird für jede Änderung die Beschreibung wie zuvor mit \emph{spaCy} analysiert.
Für jeden Zeitpunkt wird eine CSV-Datei mit dem Namen \emph{TIMESTAMP\_readme\_authors\_new.csv} erstellt, wobei TIMESTAMP durch den konkreten Zeitpunkt der Änderung der Datei ersetzt wird.
Hierbei wird der \emph{commit date} Zeitpunkt aus Git verwendet.
Die erstellte Datei enthält die extrahierten Namen aus der Beschreibung, welche mit den neusten Git Daten abgeglichen wurden.
Der neuste Zeitpunkt entspricht hierbei dem Zeitpunkt, an dem die Repositorys heruntergeladen wurden, also der Tag an dem der Prozess durchlaufen wurde.

% TODO TIMESTAMP_readme_authors.csv Abgleich mit git TIMESTAMP_git_contributors.csv/ Namen
% TODO TIMESTAMP_git_contributors.csv Gleiche Werte wie in Abschnit 3.1.1 werden gespeichert
% TODO readme.csv committed_datetime und authored_datetime
