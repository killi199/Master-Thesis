\subsection{Citation File Format}
\label{subsec:datenbeschaffung_cff}
In diesem Unterabschnitt wird ähnlich wie im vorherigen Unterabschnitt die \gls{cff} Datei untersucht.
Der Unterabschnitt ist dabei aufgeteilt in die allgemeinen Daten, welche für \gls{cff} und \grqq preferred-citation\glqq{} gelten und die spezifischen Daten, welche ausschließlich für die \grqq preferred-citation\glqq{} gelten.
Die beiden Daten werden jeweils in einer \path{CITATION.cff} Datei gespeichert, wie es in den Grundlagen erläutert wurde.
Es wird ebenfalls ein zeitlicher Verlauf der Daten betrachtet, um die Änderungen in der Zeit zu analysieren.
Dabei wird der Verlauf nicht auf 50 Zeitpunkte beschränkt, sondern in der gesamten Historie betrachtet.
Da die \gls{cff} Datei in YAML geschrieben wird, wird die Bibliothek \emph{pyyaml} in der Version 6.0.2 verwendet, um die Datei zu lesen \autocite{noauthor_yamlpyyaml_2024}.

In der \gls{cff} können für die eigentliche Zitation und für die \grqq preferred-citation\glqq{} Autoren angegeben werden.
Diese werden jeweils getrennt extrahiert anschließend mit den neusten Git Daten abgeglichen und in separaten Dateien gespeichert.
Dabei werden nur die Autoren betrachtet, welche Personen sind und keine Entitäten wie beispielsweise Organisationen.
Die Autoren werden in der Datei \path{TIMESTAMP_cff_authors_new.csv} gespeichert, wobei \emph{TIMESTAMP} durch den Zeitpunkt der Änderung ersetzt wird.
Die Autoren aus der \grqq preferred-citation\glqq{} werden in der Datei \path{TIMESTAMP_cff_preferred_citation_authors_new.csv} gespeichert.
In beiden Dateien sind der Name, die E-Mail, die ORCID-ID und die Git Daten, welche durch den Abgleich ermittelt wurden, enthalten.
Die Felder der Dateien sind in \autoref{tab:cran_authors} aufgelistet.

Zusätzlich zu den neusten Daten wird zu jedem Änderungszeitpunkt der Git Prozess aus \autoref{subsec:datenbeschaffung_git} bis zu diesem Zeitpunkt durchlaufen.
Die Liste der Autoren wird in der Datei \path{TIMESTAMP_git_contributors.csv} gespeichert.
Die Liste enthält die Autoren, welche bis zu diesem Zeitpunkt Änderungen vorgenommen haben.
Außerdem wird die Liste der Autoren, welche aus der \gls{cff} extrahiert wurden, mit den Git Autoren bis zu diesem Zeitpunkt abgeglichen.
Dadurch entstehen die beiden Dateien \path{TIMESTAMP_cff_authors.csv} und \path{TIMESTAMP_cff_preferred_citation_authors.csv}.
Enthalten sind die gleichen Informationen wie in den \glqq new\grqq{} Dateien.

Außerdem können in beiden Fällen ähnlich wie bei der Beschreibung allgemeine Daten extrahiert werden.
Hierbei wird wie in der Beschreibung eine weitere CSV-Datei erstellt, welche die allgemeinen Daten enthält.
In dem Fall der \gls{cff} werden zwei allgemeine Dateien erstellt einmal für die \gls{cff} und einmal für die \glqq preferred-citation\grqq{}.
Die Dateien heißen \path{cff.csv} und \path{cff_preferred_citation.csv}.
Die Felder der \path{cff.csv} Datei sind in \autoref{tab:cff} aufgelistet und die der \path{cff_preferred_citation.csv} Datei in \autoref{tab:cff_preferred_citation}.
In den beiden Dateien werden ebenfalls die Zeitpunkte \emph{commit date} und \emph{author date} gespeichert.

In den Dateien wird ebenfalls gespeichert, ob die \gls{cff} Datei valide ist.
Hierbei wird das Programm \emph{cffconvert} in der Version 2.0.0 verwendet, um die Validität zu überprüfen \autocite{spaaks_cffconvert_2021}.
Außerdem benötigt \emph{cffconvert} die Programme \emph{jsonschema} und \emph{pykwalify}, welche in der Masterarbeit in der Version 4.23.0 und 1.8.0 verwendet werden \autocites{noauthor_python-jsonschemajsonschema_2024}{grokzen_grokzenpykwalify_2024}.
Diese werden von \emph{cffconvert} benötigt, um die \gls{cff} Datei zu validieren.
Zusätzlich wird gespeichert, ob die \gls{cff} Datei mit dem Tool \emph{cffinit} erstellt wurde \autocite{spaaks_cffinit_2023}.
Dies ist möglich, da \emph{cffinit} in der \gls{cff} Datei mehrere Kommentare einfügt, welche auf das Tool hinweisen.
Es wird davon ausgegangen, dass \emph{cffinit} benutzt wurde, falls der Kommentar \glqq This CITATION.cff file was generated with cffinit.\grqq{} in der \gls{cff} Datei vorhanden ist.
Dies ist jedoch kein sicherer Indikator, da der Kommentar auch manuell hinzugefügt und vor allem entfernt werden kann.
Diese Werte werden jeweils in den allgemeinen Dateien gespeichert.
Dies führt zu einer doppelten Speicherung, allerdings ist dadurch in beiden Dateien separat zu erkennen, ob die \gls{cff} Datei aus denen die Daten extrahiert wurden, valide ist und ob \emph{cffinit} benutzt wurde.

Alle weiteren Informationen werden direkt aus der \gls{cff} Datei extrahiert und in die eigene CSV Datenstruktur übertragen.
Falls ein Wert dabei nicht in der \gls{cff} Datei vorhanden ist, wird der Eintrag in der CSV Datei leer gelassen.
Im Folgenden werden die Daten, welche direkt extrahiert werden, beschrieben.
Eine extrahierte Information, welche in beiden Dateien, also der \path{cff.csv} und \path{cff_preferred_citation.csv} gespeichert wird, ist die \gls{doi} aus dem \gls{cff} Feld \glqq identifiers\grqq{}.
In der Masterarbeit wird ausschließlich die erste \gls{doi} in der Liste der \glqq identifiers\grqq{} betrachtet.
Diese wird in der allgemeinen Datei gespeichert, um später zu überprüfen, ob eine \gls{doi} in diesem Feld vorhanden ist.

Zusätzlich wird in beiden fällen das Feld \glqq date-released\grqq{} und die \glqq doi\grqq{} extrahiert und gespeichert.
Die \glqq doi\grqq{} wird in den allgemeinen Dateien gespeichert, um im weiteren Verlauf zu überprüfen, ob eine \gls{doi} in der \gls{cff} Datei vorhanden ist.
Als letzte gemeinsame Information wird die \glqq type\grqq{} extrahiert und in den allgemeinen Dateien gespeichert.
Dabei können wie in den Grundlagen erwähnt für die allgemeinen Informationen in der \gls{cff} nur \glqq software\grqq{} und \glqq dataset\grqq{} vorkommen.
Für die allgemeinen Daten der \glqq preferred-citation\grqq{} können weitere Typen vorkommen, welche in den Grundlagen erläutert wurden.

Alle weiteren Daten sind spezifisch für die \glqq preferred-citation\grqq{}.
Hierbei wird das Feld \glqq date-published\grqq{} extrahiert und in der allgemeinen Datei \path{cff_preferred_citation.csv} gespeichert.
Die Felder \glqq year\grqq{}, \glqq month\grqq{} und \glqq collection-doi\grqq{} werden ebenfalls extrahiert und in der allgemeinen Datei für die \glqq preferred-citation\grqq{} gespeichert.
Die Felder der Datei sind in \autoref{tab:cff_preferred_citation} aufgelistet.

\begin{table}
    \centering
    \setlength{\tabcolsep}{8pt}
    \begin{tabular}{p{4cm}|p{9.5cm}}
        \toprule
        \textbf{Feld} & \textbf{Beschreibung} \\ \midrule
        \emph{cff\_valid} & Gibt an, ob die \gls{cff} Datei valide ist \\
        \emph{cff\_init} & Gibt an, ob \emph{cff\_init} benutzt wurde \\
        \emph{type} & Typ der \gls{cff} Datei \\
        \emph{date-released} & Datum an dem das Werk zugänglich gemacht wurde \\
        \emph{doi} & Zu zitierende \gls{doi} \\
        \emph{identifier-doi} & Die erste \gls{doi} in den \glqq identifiers\grqq{} \\
        \emph{committed\_datetime} & Zeitpunkt des \emph{commit date} \\
        \emph{authored\_datetime} & Zeitpunkt des \emph{author date} \\
        \bottomrule
    \end{tabular}
    \caption{Felder der \texttt{cff.csv} Datei}
    \label{tab:cff}
    \small
    \raggedright
    Die Tabelle zeigt die Felder der \path{cff.csv} Datei. Für jede Änderung der \gls{cff} Datei werden die dargestellten Werte gespeichert. Die Felder \emph{date-released}, \emph{doi} und \emph{identifier-doi} sind optional.
\end{table}

\begin{table}
    \centering
    \setlength{\tabcolsep}{8pt}
    \begin{tabular}{p{4cm}|p{9.5cm}}
        \toprule
        \textbf{Feld} & \textbf{Beschreibung} \\ \midrule
        \emph{cff\_valid} & Gibt an, ob die \gls{cff} Datei valide ist \\
        \emph{cff\_init} & Gibt an, ob \emph{cff\_init} benutzt wurde \\
        \emph{type} & Typ der \gls{cff} Datei \\
        \emph{date-released} & Datum an dem das Werk zugänglich gemacht wurde \\
        \emph{date-published} & Veröffentlichungsdatum \\
        \emph{year} & Veröffentlichungsjahr \\
        \emph{month} & Veröffentlichungsmonat \\
        \emph{doi} & Zu zitierende \gls{doi} \\
        \emph{collection-doi} & \gls{doi} der Sammlung \\
        \emph{identifier-doi} & Die erste \gls{doi} in den \glqq identifiers\grqq{} \\
        \emph{committed\_datetime} & Zeitpunkt des \emph{commit date} \\
        \emph{authored\_datetime} & Zeitpunkt des \emph{author date} \\
        \bottomrule
    \end{tabular}
    \caption{Felder der \texttt{cff\_preferred\_citation.csv} Datei}
    \label{tab:cff_preferred_citation}
    \small
    \raggedright
    Die Tabelle zeigt die Felder der \path{cff_preferred_citation.csv} Datei. Für jede Änderung der \gls{cff} Datei werden die dargestellten Werte gespeichert, falls das Feld \grqq preferred-citation\glqq{} angegeben ist. Die Felder \emph{date-released}, \emph{date-published}, \emph{year}, \emph{month}, \emph{doi}, \emph{collection-doi} und \emph{identifier-doi} sind optional.
\end{table}
