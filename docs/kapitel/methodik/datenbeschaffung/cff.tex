\subsection{Citation File Format}
\label{subsec:datenbeschaffung_cff}
% pyyaml==6.0.2, cffconvert==2.0.0, jsonschema==4.23.0, pykwalify==1.8.0

% Allgemein für CFF und Preferred Citation
% TODO Untersucht wird die CITATION.cff Datei
% TODO sagen, dass nur die erste identifier-doi beschafft wird, da nur geguckt wird in den Ergebnissen ob überhaupt eine vorhanden ist
% TODO Wie in Beschreibung wird für jeden Änderungszeitpunkt die Daten beschafft diesmal aber nicht nur für 50 sondern für alle Änderungen
% TODO für jede Änderung wieder die git contributors in TIMESTAMP_git_contributors.csv

% CFF
% TODO gespeichert in TIMESTAMP_cff_authors.csv und TIMESTAMP_cff_authors_new.csv (Inhalt: name, email, orcid, git Daten durch abgleich)
% TODO allgemeine Infos in cff.csv (Inhalt: cff_valid, cff_init, type, daterelease, doi, identifier-doi, committed_datetime, authored_datetime)

% Preferred Citation
% TODO gespeichert in TIMESTAMP_cff_preferred_citation_authors.csv und TIMESTAMP_cff_preferred_citation_authors_new.csv (Inhalt: name, email, orcid, git Daten durch abgleich) 
% TODO allgemeine Infos in cff_preferred_citation.csv (Inhalt: cff_valid, cff_init, type, date-released, date-published, year, month, doi, collection-doi, identifier-doi, committed_datetime, authored_datetime)
