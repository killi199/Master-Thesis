\section{Datenbeschaffung}
\label{sec:datenbeschaffung}
\subsection{Git}
\label{subsec:git}
% TODO Beschreiben wie die Commits gezählt wurden.
\subsection{PyPi}
\label{subsec:pypi}
% TODO auf verifizierte PyPi Nutzer eingehen
% Checken was verifizierte Nutzer im PyPi Universum bedeutet. Sind es nur verifizierte User die z.B. eine E-Mail hinterlegt haben oder sind es Nutzer die an dem Projekt arbeiten und von PyPi verifiziert sind? Falls es mehrwert hat diese Daten ebenfalls automatisch abfragen und auch in der MA beschreiben, was es nun ist und warum es Mehrwert hat oder auch nicht.
% In MA beschreiben, warum oder warum nicht Mehrwert (PyPi Verifizierte Nutzer)
% Verifizierte Daten sind Daten, welche PyPi als "richtig" ansieht. Aktuell ist es im Prozess, dass immer mehr Daten diesen Status erreichen. Aktuell können nur der Owner und Maintainer diesen Status erreichen. Verifizierte Maintainer sind dabei Personen, welche bei PyPi registriert sind und dem Projekt zugeordnet sind. Also das Projekt auf PyPi verwalten können.
% https://github.com/pypi/warehouse/issues/8635
% https://github.com/pypi/warehouse/issues/15903
% https://peps.python.org/pep-0740/
% https://github.com/pypi/warehouse/issues/15871 (Roadmap Pep 740)
% Aktuell gibt es noch keinen weg die verified details über die API zu beziehen (https://github.com/pypi/warehouse/issues/14799) aus diesem Grund muss es aus dem HTML code extrahiert werden.
% Verifizierte "Maintainer" in PyPi sind Personen, welche neue Releases veröffentlichen dürfen (https://github.com/pypi/warehouse/issues/13366)
% Die nicht verifizierten Daten stammen aus der Python toml
% Es existiert aktuell keine API für user (https://github.com/pypi/warehouse/issues/15769) aus diesem Grund muss ich die aus dem HTML besorgen.
\subsection{CRAN}
\label{subsec:cran}
\subsection{Beschreibung}
\label{subsec:beschreibung}
% TODO Es werden alle Namen ausgegeben aber auch eben solche, die gar nichts mit dem Paket zu tun haben wie im fall von highr für CRAN: "Provides syntax highlighting for R source code. Currently it supports LaTeX and HTML output. Source code of other languages is supported via Andre Simon's highlight package (https://gitlab.com/saalen/highlight)." Es gibt noch weitere Beispiele z.B. in CRAN magrittr
\subsection{Citation File Format}
\label{subsec:cff}
\subsection{BibTeX}
\label{subsec:bibtex}
