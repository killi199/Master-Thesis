\section{Datenbeschaffung}
\label{sec:datenbeschaffung}
\subsection{Git}
\label{subsec:datenbeschaffung_git}
% TODO Beschreiben wie die Commits gezählt wurden.
\subsection{PyPI}
\label{subsec:datenbeschaffung_pypi}
% TODO erklären warum BigQuery nicht eingesetzt wird aktuell aber dennoch sinnvoll eingesetzt werden könnte bei anderen anforderungen (z.B. liste mit repo links auf github)
% TODO erklären warum die API verwendet wird und nicht nur die TOML beispielsweise ausgelesen wird -> habe ein pypi paket und brauche die GitHub URL
% TODO sagen warum beides also verifizierte Owner und Maintainer sowhl als auch die Daten aus der TOML abgefragt werden -> sind unterschiedliche Leute und werden unterschiedlich angegeben die einen dürfen sachen auf pypi machen die anderen werden in der toml angegeben und dürfen ggf. nichts in pypi machen
% Checken was verifizierte Nutzer im PyPI Universum bedeutet. Sind es nur verifizierte User die z.B. eine E-Mail hinterlegt haben oder sind es Nutzer die an dem Projekt arbeiten und von PyPI verifiziert sind? Falls es mehrwert hat diese Daten ebenfalls automatisch abfragen und auch in der MA beschreiben, was es nun ist und warum es Mehrwert hat oder auch nicht.
% In MA beschreiben, warum oder warum nicht Mehrwert (PyPI Verifizierte Nutzer)
\subsection{CRAN}
\label{subsec:datenbeschaffung_cran}
\subsection{Beschreibung}
\label{subsec:datenbeschaffung_beschreibung}
% TODO Es werden alle Namen ausgegeben aber auch eben solche, die gar nichts mit dem Paket zu tun haben wie im fall von highr für CRAN: "Provides syntax highlighting for R source Code. Currently it supports LaTeX and HTML output. Source Code of other languages is supported via Andre Simon's highlight package (https://gitlab.com/saalen/highlight)." Es gibt noch weitere Beispiele z.B. in CRAN magrittr -> vllt eher ergebnis?
\subsection{Citation File Format}
\label{subsec:datenbeschaffung_cff}
\subsection{Bib\TeX{}}
\label{subsec:datenbeschaffung_bibtex}
