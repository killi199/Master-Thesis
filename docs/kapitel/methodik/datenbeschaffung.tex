\section{Datenbeschaffung}
\label{sec:datenbeschaffung}
% TODO MEINE SOFTWARE ZITIEREN! Sonst bin ich nicht besser als alle anderen und auch Probleme und Konfiguration darstellen -> steht in dem Paper smith_software_2016 wie zitiert werden soll? \autocite{richardson_beautifulsoup4_2024}
In diesem Abschnitt wird beschrieben wie das Skript zur Datenbeschaffung aus den einzelnen Quellen aufgebaut ist.
Es wird \emph{tqdm} in der Version 4.66.5 verwendet, um den Fortschritt der Datenbeschaffung anzuzeigen \autocite{costa-luis_tqdm_2024}.
Die Datenbeschaffung wird in die einzelnen Quellen untergliedert.

Aus den Quellen \nameref{subsec:datenbeschaffung_git}, \nameref{subsec:datenbeschaffung_beschreibung}, \nameref{subsec:datenbeschaffung_cff} und \nameref{subsec:datenbeschaffung_bibtex} können zeitliche Informationen extrahiert werden, da diese in Git verwaltet werden.
Aus diesem Grund werden die Daten jeweils zu der Änderung der Quelle gespeichert.
Dabei ist die maximale Anzahl der Änderungen in die Vergangenheit für die Beschreibung auf 50 beschränkt, um die Laufzeit des Skripts zu begrenzen.
Die anderen Quellen haben keine Beschränkung, da diese nicht so häufig aktualisiert werden.

In \nameref{subsec:datenbeschaffung_cran} ist es nicht möglich die Änderungen in der Zeit zu betrachten.
In der \nameref{subsec:datenbeschaffung_pypi} Quelle ist es teilweise mit BigQuery möglich die Änderungszeitpunkte zu erhalten, jedoch ist dies mit Kosten verbunden und erfordert eine andere Vorgehensweise als bei den anderen zeitlichen Daten, da diese nicht direkt aus Git extrahiert werden können.
Die beiden Quellen werden aus diesem Grund nur in der neusten Version betrachtet und enthalten keine Änderungshistorie.

\subsection{Git}
\label{subsec:datenbeschaffung_git}
Die Git-Daten sind die grundlegenden Daten, welche für die weiteren Schritte benötigt werden.
Sämtliche anderen Quellen werden mit den Git-Daten über den in \autoref{sec:abgleich} beschriebenen Prozess abgeglichen.
Zu Beginn muss das Repository von GitHub heruntergeladen werden, um die Daten lokal verarbeiten zu können.
Dabei kommt die \gls{oss} \emph{GitPython} in der Version 3.1.43 zum Einsatz.
Die Bibliothek bietet eine Schnittstelle, um Git-Befehle in Python zu verwenden \autocite{thiel_gitpython_2024}.
Beim Aufruf der Funktion wurde kein Branch spezifiziert, weshalb der Branch ausgewählt wird, auf welchen aktuell der \emph{HEAD} vom remote Repository zeigt.
Dieser Branch ist in der Regel der Standardbranch, auf welchem die Analyse durchgeführt werden soll.
Für das Herunterladen muss außerdem der Link zum GitHub-Repository angegeben werden, welcher aus der \gls{pypi} oder \gls{cran} Quelle stammt.
Auf diese Quellen wird in den nächsten Abschnitten eingegangen.

Die Auswertung des Repositorys wird mit \emph{git-quick-stats} in der Version 2.3.0 durchgeführt \autocite{arzzen_git-quick-stats_2021}.
\emph{Git-quick-stats} bietet einfache und effiziente Möglichkeiten, um verschiedene Statistiken in einem Git-Repository zu ermitteln.
Das Tool wird in dem Python-Skript mit dem Befehl \emph{git-quick-stats -T} aufgerufen, um detaillierte Statistiken zu erhalten.
Ausgegeben wird eine Liste aller Autoren, welche in dem Repository Änderungen vorgenommen haben.
Diese Liste enthält den Namen, die E-Mail, die Anzahl der Einfügungen, Löschungen, geänderten Zeilen, Dateien, Commits, sowie den ersten und letzten Commit, welche alle am Ende des Prozesses gespeichert werden.
Die Anzahl der Commits beinhaltet keine Merge-Commits.
Dieses Verhalten von \emph{git-quick-stats} ist erwünscht, da diese nicht relevant für die Analyse sind.
In dieser Masterarbeit werden ausschließlich die Autoren betrachtet, welche Änderungen an dem Code vorgenommen haben.

Die E-Mail-Adresse wird anschließend in Kleinbuchstaben umgewandelt, um die Daten zu vereinheitlichen.
Zusätzlich wird eine Gruppierung auf der E-Mail durchgeführt und die anderen Werte summiert, mit Ausnahme des ersten und letzten Commits, bei denen der älteste und neueste Commit ausgewählt werden.
Der Name wird hintereinander gehängt, sodass beispielsweise bei einer Namensänderung von \glqq Max M.\grqq{} zu \glqq Max Mustermann\grqq{} der Name \glqq Max M.Max Mustermann\grqq{} entsteht.
Das Gruppieren ist notwendig, da die Autoren den Namen und die E-Mail-Adressen in Git eigenständig festlegen können.
Durch dieses Vorgehen wird gewährleistet, dass zumindest keine E-Mail-Adressen doppelt vorhanden sind, falls ein Autor unterschiedliche Schreibweisen für seinen Namen verwendet.
Die gruppierten Daten werden nach der Anzahl der Commits sortiert in der Datei \path{git_contributors.csv} gespeichert.
In \autoref{tab:git_contributors} sind die Felder der Datei aufgelistet.

Außerdem bietet das Tool die Möglichkeit, mit der gesetzten Umgebungsvariablen \texttt{\_GIT\_UNTIL=}, alle Änderungen nur bis zu einem bestimmten Zeitpunkt zu betrachten.
Diese Funktion wird verwendet, um die Änderungen bis zur Aktualisierung einer Quelle zu betrachten.
Die Daten werden beispielsweise in der Datei \path{20210819_161452-0400_git_contributors.csv} gespeichert, wobei der erste Teil des Dateinamens den konkreten Tag und Uhrzeit mit zugehöriger Zeitzone angibt.
Dabei hat die Datei die gleichen Felder wie die \path{git_contributors.csv}-Datei, jedoch nur für die Änderungen bis zu dem angegebenen Zeitpunkt.
Für die Verarbeitung der Zeiten in unterschiedlichen Zeitzonen wird das Modul \emph{pytz} in der Version 2024.2 verwendet \autocite{bishop_pytz_2024}.

\begin{table}
    \begin{tabularx}{\textwidth}{XL{10.2cm}}
        \toprule
        \textbf{Feld}         & \textbf{Beschreibung}                   \\ \midrule
        \emph{name}           & Name des Autors                         \\
        \emph{email}          & E-Mail des Autors                       \\
        \emph{insertions}     & Hinzugefügte Zeilen des Autors          \\
        \emph{deletions}      & Gelöschte Zeilen des Autors             \\
        \emph{lines\_changed} & Geänderte Zeilen des Autors             \\
        \emph{files}          & Geänderte Dateien des Autors            \\
        \emph{commits}        & Anzahl der Commits des Autors           \\
        \emph{first\_commit}  & Zeitpunkt des ersten Commits des Autors \\
        \emph{last\_commit}  & Zeitpunkt des letzten Commits des Autors \\
        \bottomrule
    \end{tabularx}
    \caption{Felder der \texttt{git\_contributors.csv}-Datei}
    \label{tab:git_contributors}
    \small
    Die Tabelle zeigt die Felder der \path{git_contributors.csv} oder der \path{TIMESTAMP_git_contributors.csv}-Datei. Für jeden Autor der betrachteten Software werden die dargestellten Werte gespeichert.
\end{table}

\subsection{PyPI}
\label{subsec:datenbeschaffung_pypi}
Zu Beginn des Prozesses wird die Datei eingelesen, in denen die Top 100 Pakete von \gls{pypi} aufgelistet sind.
In dieser Datei ist ausschließlich der Name des Pakets und die Anzahl der Downloads auf \gls{pypi} enthalten.
Für die Beschaffung der Repository Daten von GitHub wird jedoch der Link zu der Versionsverwaltung benötigt.
Dieser wird mithilfe von \emph{aiohttp} in der Version 3.10.3 von der JSON API von \gls{pypi} abgefragt \autocite{noauthor_aio-libsaiohttp_2024}.
Dabei muss ein Paket nicht zwangsweise ein GitHub Repository haben, weshalb die Daten nicht immer vorhanden sind.
In diesem Fall wird das Paket übersprungen und nicht weiter betrachtet.

Anschließend werden die weiteren Daten verarbeitet, welche von der JSON API abgefragt wurden.
Dabei werden die nicht verifizierten Autoren und Maintainer jeweils mit Name und E-Mail des Pakets extrahiert.
Welche Werte dort enthalten sind, können die Paketentwickler selbst entscheiden.
Beispielsweise geben einige Paketentwickler mehrere Autoren mit Komma separiert an.
Diese werden aufgeteilt und als einzelne Autoren gespeichert.
Ebenfalls werden die E-Mail Adressen anhand des Kommas separiert und jeweils mit dem Namen des Autors verbunden.
Dies geschieht so, dass der 1. Name mit der 1. E-Mail verbunden wird, der 2. Name mit der 2. E-Mail und so weiter.
Falls ein Autor keine E-Mail angegeben hat, wird der Name ohne E-Mail gespeichert und falls nur eine E-Mail angegeben ist, wird diese ohne einen Namen gespeichert.

Ebenfalls geben einige Paketentwickler keine E-Mail im E-Mail Feld an, sondern nur den Namen und schreiben in das Namensfeld zusätzlich eine E-Mail Adresse auch der andere Weg ist möglich.
Es gibt noch verschiedene weitere Sonderfälle, welche jedoch nicht alle aufgezählt werden, da sie gleich behandelt werden.
In diesen Sonderfällen wird keine weitere Betrachtung vorgenommen und die Daten so gespeichert, wie sie von der API zurückgegeben wurden.
Anschließend werden die Daten mit dem in \autoref{sec:abgleich} beschriebenen Prozess mit den zuvor beschafften Git Daten abgeglichen, dabei werden die verschiedenen Sonderfälle, welche auftreten können berücksichtigt.
Anschließend werden die Daten in den Dateien \path{python_authors.csv} und \path{python_maintainers.csv} gespeichert.
Inhalt ist dabei der Name und die E-Mail des Autors oder Maintainers, sowohl als auch die Daten aus dem Abgleich mit Git.
Die Felder der Dateien sind in \autoref{tab:python_authors} aufgelistet.

Zusätzlich zu den Informationen der Autoren und Maintainer wird die Beschreibung des Pakets von der API zurückgegeben.
Einige Pakete haben in der Beschreibung ebenfalls Autoren angegeben, welche zusätzlich verarbeitet werden.
Die Beschreibung wird als unstrukturierter Text zurückgegeben.
Dieser Text wird mittels der \gls{ner} Bibliothek \emph{spaCy} in der Version 3.7.6 verarbeitet, um die Autoren zu extrahieren \autocite{honnibal_spacy_2020}.
Dabei wird das Programm so verwendet, dass nur die Entitäten \emph{PERSON} extrahiert werden.
Als Modell wurde das \texttt{en\_core\_web\_trf} verwendet, welches auf Englisch trainiert ist und eine höhere Genauigkeit aufweist als das \texttt{en\_core\_web\_sm} Modell.
Durch die Verwendung des genaueren Modells ist die Laufzeit erhöht. 
In Experimenten hat sich jedoch gezeigt, dass das kleinere Modell keine guten Ergebnisse für die Beschreibungen liefert, da sehr viele Entitäten fälschlicherweise als Autoren erkannt werden.
Die extrahierten Autoren werden ebenfalls mit den Git Daten abgeglichen und in der Datei \path{description_authors.csv} gespeichert.
Inhalt ist dabei der Name des Autors, sowohl als auch die Daten aus dem Abgleich mit Git.
Die Felder der Datei sind in \autoref{tab:description_authors} aufgelistet.

Außerdem werden die verifizierten Maintainer, welche auf \gls{pypi} angegeben sind verarbeitet, da diese unter Umständen nicht den Autoren entsprechen, welche durch die API ausgegeben werden.
Diese werden durch \gls{pypi} nicht mittels der JSON API ausgegeben, weshalb die Daten mittels der XML-RPC API abgefragt werden.
Die API liefert den Benutzernamen sowie die Rolle des Autors.
Die Rolle kann dabei \emph{Maintainer} oder \emph{Owner} sein.
\emph{Owner} ist hierbei nicht der \emph{Owner}, welcher in \autoref{fig:pypi_verified_unverified_details} unter \emph{Owner} aufgeführt ist, sondern in diesem konkreten Fall sind alle drei Betreuer des Pakets als \emph{Owner} angegeben.
Der Benutzer \grqq Matplotlib\glqq{}, welcher unter \emph{Owner} aufgeführt ist, wird nicht über die API zurückgegeben.
Dieser wird jedoch auch nicht benötigt, da unter \emph{Owner} immer eine Organisation angeführt wird und in der Masterarbeit nur Personen betrachtet werden.
Im Allgemeinen ist die Dokumentation der XML-RPC API nicht gut beschrieben und es wird nicht deutlich, welche Daten zurückgegeben werden.

Anschließend werden alle verifizierten Autoren, welche durch die API zurückgegeben wurden analysiert, unabhängig von der angegebenen Rolle.
Da die Autoren ausschließlich mit dem Benutzernamen zurückgegeben werden, wird ein Webscraper benötigt, um den vollständigen Namen des Autor zu erhalten.
Falls ein Autor einen Namen angegeben hat, wird dieser auf der Profilseite des Benutzers dargestellt.
Um diesen zu erhalten wird eine Anfrage mittels \emph{aiohttp} an die Profilseite des Benutzers gestellt.
Anschließend wird mittels \emph{BeautifulSoup} in der Version 4.12.3 der Name aus dem HTML extrahiert \autocite{richardson_beautifulsoup4_2024}.
Die Daten werden, nachdem sie mit den Git Daten abgeglichen wurden, in der Datei \path{pypi_maintainers.csv} gespeichert.
Inhalt ist dabei der Benutzername und der Name des Autors, sowohl als auch die Daten aus dem Abgleich mit Git.
Die Felder der Datei sind in \autoref{tab:pypi_maintainers} aufgelistet.

\begin{table}
    \centering
    \setlength{\tabcolsep}{8pt}
    \begin{tabular}{p{3.5cm}|p{10cm}}
        \toprule
        \textbf{Feld} & \textbf{Beschreibung} \\ \midrule
        \emph{rank} & Rang des Autors sortiert nach der Anzahl der Commits \\
        \emph{insertions} & Hinzugefügte Zeilen des Autors \\
        \emph{deletions} & Gelöschte Zeilen des Autors \\
        \emph{lines\_changed} & Geänderte Zeilen des Autors \\
        \emph{files} & Geänderte Dateien des Autors \\
        \emph{commits} & Anzahl der Commits des Autors \\
        \emph{first\_commit} & Zeitpunkt des ersten Commits des Autors \\
        \emph{last\_commit} & Zeitpunkt des letzten Commits des Autors \\
        \emph{score} & Zu wie viel Prozent der Abgleich mit Git übereinstimmt \\
        \bottomrule
    \end{tabular}
    \caption{Felder, welche durch den Abgleich mit Git entstehen}
    \label{tab:abgleich_felder}
    \small
    \raggedright
    Die Tabelle zeigt die Felder, welche durch den Abgleich mit Git entstehen. Für jeden Autor oder Maintainer der betrachteten Software werden die dargestellten Werte gespeichert, falls der Autor abgeglichen werden konnte. Falls der Autor nicht abgeglichen werden konnte sind die Felder leer und der Score ist 0.
\end{table}

\begin{table}
    \centering
    \setlength{\tabcolsep}{8pt}
    \begin{tabular}{p{3.5cm}|p{10cm}}
        \toprule
        \textbf{Feld} & \textbf{Beschreibung} \\ \midrule
        \emph{name} & Name des Autors \\
        \emph{email} & E-Mail des Autors \\
        \bottomrule
    \end{tabular}
    \caption{Felder der \texttt{python\_authors.csv}, \texttt{python\_maintainers.csv} und \texttt{cran\_maintainers.csv} Datei}
    \label{tab:python_authors}
    \small
    \raggedright
    Die Tabelle zeigt die Felder der \texttt{python\_authors.csv}, \texttt{python\_maintainers.csv} und \texttt{cran\_maintainers.csv} Datei. Für jeden Autor oder Maintainer der betrachteten Software wird der Name und die E-Mail gespeichert, falls diese angegeben wurde. Außerdem werden weitere Felder durch den Abgleich mit Git in der Datei gespeichert, welche in \autoref{tab:abgleich_felder} aufgelistet sind.
\end{table}

\begin{table}
    \centering
    \setlength{\tabcolsep}{8pt}
    \begin{tabular}{p{3.5cm}|p{10cm}}
        \toprule
        \textbf{Feld} & \textbf{Beschreibung} \\ \midrule
        \emph{login} & Benutzername des Autors \\
        \emph{name} & Name des Autors \\
        \bottomrule
    \end{tabular}
    \caption{Felder der \texttt{pypi\_maintainers.csv} Datei}
    \label{tab:pypi_maintainers}
    \small
    \raggedright
    Die Tabelle zeigt die Felder der \path{pypi_maintainers.csv} Datei. Für jeden Maintainer der betrachteten Software wird der Benutzername angegeben. Der Name kann leer sein, da er nicht angegeben werden muss. Außerdem werden weitere Felder durch den Abgleich mit Git in der Datei gespeichert, welche in \autoref{tab:abgleich_felder} aufgelistet sind.
\end{table}

\begin{table}
    \centering
    \setlength{\tabcolsep}{8pt}
    \begin{tabular}{p{3.5cm}|p{10cm}}
        \toprule
        \textbf{Feld} & \textbf{Beschreibung} \\ \midrule
        \emph{name} & Name des Autors \\
        \bottomrule
    \end{tabular}
    \caption{Felder der \texttt{description\_authors.csv}, \texttt{TIMESTAMP\_readme\_authors(\_new).csv} und \texttt{TIMESTAMP\_bib\_authors(\_new).csv} Datei}
    \label{tab:description_authors}
    \small
    \raggedright
    Die Tabelle zeigt die Felder der \texttt{description\_authors.csv}, \texttt{TIMESTAMP\_readme\_authors(\_new).csv} und \texttt{TIMESTAMP\_bib\_authors(\_new).csv} Datei. Für jeden Autor der betrachteten Software wird der Name gespeichert, welcher durch die \gls{ner} ermittelt wurde. Außerdem werden weitere Felder durch den Abgleich mit Git in der Datei gespeichert, welche in \autoref{tab:abgleich_felder} aufgelistet sind.
\end{table}

\subsection{CRAN}
\label{subsec:datenbeschaffung_cran}
Ähnlich wie bei \gls{pypi} werden zu Beginn die Daten der Top 100 Pakete aus der zuvor beschafften Datei eingelesen.
In dieser Datei sind ebenfalls ausschließlich der Name des Pakets und die Anzahl der Downloads auf \gls{cran} enthalten.
Aus diesem Grund wird zu Beginn eine Anfrage mittels \emph{aiohttp} an die von METACRAN bereitgestellte API gestellt, um die Metadaten des Pakets zu erhalten.
In den Metadaten kann der Link eines GitHub Repositorys enthalten sein.
Falls kein Link zu einem GitHub Repository vorhanden ist, wird das Paket ebenfalls nicht weiter betrachtet.

Anschließend werden die weiteren Daten verarbeitet, welche von der API bereitgestellt werden.
Von der API wird das Feld \glqq Authors@R\grqq{} bereitgestellt.
Dieses Feld beinhaltet die Autoren mit dem Namen, der E-Mail und einer ORCID-ID des Pakets in einer in R formatierten Zeichenfolge.
Dabei müssen nicht zwingend alle Informationen vorhanden sein.
Des Weiteren haben Autoren eine Rolle zugeordnet.
Die Rolle ist nicht fest definiert und kann von den Autoren frei gewählt werden.
Es existieren allerdings Standards welche eingehalten werden sollten.
Einem Autor können mehrere Rollen zugewiesen sein.
Im R Journal wurden folgende Rollen definiert \autocite{hornik_who_2011}:

\begin{itemize}
    \item \glqq \emph{aut}\grqq{} (Autor): Vollständige Autoren, die wesentliche Beiträge zu dem Paket geleistet haben und in der Zitation des Pakets auftauchen sollten.
    \item \glqq \emph{com}\grqq{} (Complier): Personen, die Code (möglicherweise in anderen Sprachen) gesammelt, aber keine weiteren wesentlichen Beiträge zum Paket geleistet haben.
    \item \glqq \emph{ctb}\grqq{} (Mitwirkender): Autoren, die kleinere Beiträge geleistet haben (z. B. Code-Patches usw.), die aber nicht in der Auflistung der Autoren auftauchen sollten.
    \item \glqq \emph{cph}\grqq{} (Urheberrechtsinhaber): Personen, die das Urheberrecht an dem Paket besitzen.
    \item \glqq \emph{cre}\grqq{} (Maintainer): Paket Maintainer
    \item \glqq \emph{ths}\grqq{} (Betreuer der Thesis): Betreuer der Thesis, wenn das Paket Teil einer Thesis ist.
    \item \glqq \emph{trl}\grqq{} (Übersetzer): Übersetzer nach R, wenn der R-Code eine Übersetzung aus einer anderen Sprache (typischerweise S) ist.
\end{itemize}

Die Daten aus der Zeichenfolge werden mit der Software \emph{rpy2} in der Version 3.5.16 verarbeitet \autocite{noauthor_rpy2rpy2_2024}.
\emph{Rpy2} ist eine Software, welche es ermöglicht R-Code in Python auszuführen.
Die Software wird mit dem R Befehl \code{eval(parse(text = '\{cran\_author\}'))} ausgeführt, wobei \code{cran\_author} die R Zeichenfolge der Autoren ist.
Anschließend werden die Autoren, welche die Rolle \emph{aut} zugeordnet haben nach dem Abgleich mit den Git Daten in der Datei \path{cran_authors.csv} gespeichert.
In der Datei wird, falls vorhanden, der Name, die E-Mail-Adresse und die ORCID-ID der Autoren gespeichert.
Die Felder der Datei sind in \autoref{tab:cran_authors} aufgelistet.

Falls das Feld \glqq Authors@R\grqq{} keine Zeichenfolge enthält oder beim Verarbeiten der Zeichenfolge ein Fehler auftritt, wird das durch die API zurückgegebene Feld \glqq Author\grqq{} verarbeitet.
In dem Feld stehen ebenfalls die Autoren des Pakets, jedoch ohne die zusätzliche Information der E-Mail.
Außerdem ist das Feld nicht in R formatiert, sodass die Zeichenfolge mittels eines regulären Ausdrucks verarbeitet wird.
Die Folge ist dabei unterschiedlich in verschiedenen Paketen, sodass keine allgemeine Regel definiert werden kann.
Einige Paketautoren geben keine ORCID-ID an, sodass die gesamte Zeichenfolge anders aufgebaut ist.
Andere Autoren wiederum geben keine Rollenbezeichnungen an, sodass die Zeichenfolge beispielsweise nur den Namen enthält.
Falls eine Rolle angegeben ist werden nur jene Autoren verarbeitet, welche als Rolle \emph{aut} zugeordnet haben.
Andernfalls werden alle Autoren verarbeitet.
Die Autoren werden in der Datei \path{cran_authors.csv} nachdem sie mit den Daten aus Git abgeglichen wurden gespeichert.
Dabei wird nur der Name gespeichert ohne zusätzlichen Informationen wie der ORCID-ID, da dieses Vorgehen die Verarbeitung bei den unterschiedlichen aufgebauten Zeichenfolgen vereinfacht hat.
Außerdem wird diese Methode nur verwendet, falls die Verarbeitung der Zeichenfolge \glqq Authors@R\grqq{} fehlschlägt.

Ein weiteres Feld, welches von der API zurückgegeben wird und verarbeitet wird, ist das Feld \glqq Maintainer\grqq{}.
Dieses Feld ist nicht in R formatiert, sondern eine einfache Zeichenfolge.
Die Zeichenfolge enthält weniger Informationen als die \glqq Author\grqq{} Zeichenfolge.
In ihr ist lediglich der Name und die E-Mail-Adresse des Maintainers enthalten.
Außerdem ist immer nur ein Maintainer angegeben, welcher verarbeitet wird.
Die Daten werden nach dem Abgleich mit den Daten aus Git in der Datei \path{cran_maintainers.csv} gespeichert.
Ausgegeben wird in der Datei der Name und die E-Mail Adresse des Maintainers.
Die Felder der Dateien sind in \autoref{tab:python_authors} aufgelistet.

Das letzte Feld, welches von der API verarbeitet wird ist das Feld \glqq Description\grqq{}.
In diesem Feld ist die Beschreibung des Pakets enthalten.
Dabei wird die Verarbeitung wie in \autoref{subsec:datenbeschaffung_pypi} für die Beschreibung von \gls{pypi} durchgeführt.
Es wird ebenfalls die \gls{ner} Bibliothek \emph{spaCy} verwendet, um die Autoren zu extrahieren.
Anschließend werden Namen der extrahierten Autoren in der Datei \path{description_authors.csv} nach einem Abgleich mit den Git Daten gespeichert.
Die Felder der Datei sind in \autoref{tab:description_authors} aufgelistet.

\begin{table}
    \centering
    \setlength{\tabcolsep}{8pt}
    \begin{tabular}{p{3.5cm}|p{10cm}}
        \toprule
        \textbf{Feld} & \textbf{Beschreibung} \\ \midrule
        \emph{name} & Name des Autors \\
        \emph{email} & E-Mail des Autors \\
        \emph{orcid} & ORCID-ID des Autors \\
        \bottomrule
    \end{tabular}
    \caption{Felder der \texttt{cran\_authors.csv}, \texttt{TIMESTAMP\_cff\_authors(\_new).csv} und \texttt{TIMESTAMP\_cff\_preferred\_citation\_authors(\_new).csv} Datei}
    \label{tab:cran_authors}
    \small
    Die Tabelle zeigt die Felder der \texttt{cran\_authors.csv}, \texttt{TIMESTAMP\_cff\_authors(\_new).csv} und \texttt{TIMESTAMP\_cff\_preferred\_citation\_authors(\_new).csv} Datei. Für jeden Autor der betrachteten Software wird der Name, die E-Mail und die ORCID-ID angegeben, falls diese vorhanden sind. Außerdem werden weitere Felder durch den Abgleich mit Git in der Datei gespeichert, welche in \autoref{tab:abgleich_felder} aufgelistet sind.
\end{table}
 % 2 Seiten
\subsection{Beschreibung}
\label{subsec:datenbeschaffung_beschreibung}
% TODO auf f1 Score eingehen von der NER und somit ggf. zeigen, dass die Werte nicht so qualitativ sind
In diesem Abschnitt wird anders als zuvor keine API angefragt, sondern auf den bereits heruntergeladenen Git Daten die Analyse durchgeführt.
Untersucht wird die Beschreibung, welche in der \path{README.md} Datei des Repositorys enthalten ist.
Dabei wird die Beschreibung wie zuvor mit \emph{spaCy} untersucht.
Anders ist jedoch, dass die \path{README.md} Datei in Git verwaltet wird und somit die Änderungen in der Zeit betrachtet werden können.
Aus diesem Grund wird das Programm \emph{GitPython} verwendet, um die letzten 50 Änderungen der Datei auf dem Standardbranch zu erhalten.
Anschließend wird für jede Änderung die Beschreibung wie zuvor mit \emph{spaCy} analysiert.
Für jeden Zeitpunkt wird eine CSV-Datei mit dem Namen \path{TIMESTAMP_readme_authors_new.csv} erstellt, wobei \emph{TIMESTAMP} durch den konkreten Zeitpunkt der Änderung der Datei ersetzt wird.
Hierbei wird der \emph{commit date} Zeitpunkt aus Git verwendet.
Die erstellte Datei enthält die extrahierten Namen aus der Beschreibung, welche mit den neusten Git Daten abgeglichen wurden.
Die Felder der Datei sind in \autoref{tab:description_authors} aufgelistet.
Der neuste Zeitpunkt entspricht hierbei dem Zeitpunkt, an dem die Repositorys heruntergeladen wurden, also der Tag an dem der Prozess durchlaufen wurde.

Zusätzlich wird zu jedem Änderungszeitpunkt der Git Prozess aus \autoref{subsec:datenbeschaffung_git} bis zu diesem Zeitpunkt durchlaufen.
Dies dient dazu, eine Liste aller Autoren zu erhalten, welche bis zu diesem Zeitpunkt Änderungen vorgenommen haben.
Die Liste wird in der Datei \path{TIMESTAMP_git_contributors.csv} gespeichert, wobei \emph{TIMESTAMP} durch den konkreten Zeitpunkt ersetzt wird.
Der Inhalt ist der gleiche wie in \autoref{subsec:datenbeschaffung_git}, jedoch nur bis zu dem Zeitpunkt, an dem die \path{README.md} Datei jeweils geändert wurde.
Diese Daten werden ebenfalls mit den Autoren, welche aus der Beschreibung extrahiert wurden, abgeglichen und in der Datei \path{TIMESTAMP_readme_authors.csv} gespeichert.
Die Felder der Datei sind ebenfalls in \autoref{tab:description_authors} aufgelistet.
Inhalt ist hierbei der Name des Autors und weitere Informationen, welche durch den Abgleich ermittelt wurden.

Im Prozess werden ebenfalls allgemeine Informationen zur Beschreibung beschafft und anschließend in der Datei \path{readme.csv} gespeichert.
Die Datei beinhaltet die Zeitpunkte \emph{commit date} und \emph{author date} für jede Änderung der Beschreibung.
Die Felder der Datei sind in \autoref{tab:readme} dargestellt.

\begin{table}
    \centering
    \setlength{\tabcolsep}{8pt}
    \begin{tabular}{p{4cm}|p{9.5cm}}
        \toprule
        \textbf{Feld} & \textbf{Beschreibung} \\ \midrule
        \emph{committed\_datetime} & Zeitpunkt des \emph{commit date} \\
        \emph{authored\_datetime} & Zeitpunkt des \emph{author date} \\
        \bottomrule
    \end{tabular}
    \caption{Felder der \texttt{readme.csv} Datei}
    \label{tab:readme}
    \small
    \raggedright
    Die Tabelle zeigt die Felder der \path{readme.csv} Datei. Für jede Änderung der Beschreibung werden die dargestellten Werte gespeichert.
\end{table}
 % 2 Seiten
\subsection{Citation File Format}
\label{subsec:datenbeschaffung_cff}
In diesem Unterabschnitt wird ähnlich wie im vorherigen Unterabschnitt die \gls{cff}-Datei untersucht.
Der Unterabschnitt ist dabei aufgeteilt in die allgemeinen Daten, welche für \gls{cff} und \grqq preferred-citation\glqq{} gelten und die spezifischen Daten, welche ausschließlich für die \grqq preferred-citation\glqq{} gelten.
Die beiden Daten werden jeweils in einer \path{CITATION.cff}-Datei gespeichert, wie es in den Grundlagen erläutert wurde.
Es wird ebenfalls ein zeitlicher Verlauf der Daten betrachtet, um die Änderungen über einen Zeitraum zu analysieren.
Dabei wird der Verlauf nicht auf 50 Zeitpunkte beschränkt, sondern in der gesamten Historie betrachtet.
Da die \gls{cff}-Datei in YAML geschrieben wird, wird die Bibliothek \emph{pyyaml} in der Version 6.0.2 verwendet, um die Datei zu lesen \autocite{simonov_pyyaml_2024}.

In der \gls{cff} können für die eigentliche Zitation und für die \grqq preferred-citation\glqq{} Autoren angegeben werden.
Diese werden jeweils getrennt extrahiert, anschließend mit den neuesten Git-Daten abgeglichen und in separaten Dateien gespeichert.
Dabei werden nur die Autoren betrachtet, welche Personen sind und keine Entitäten wie beispielsweise Organisationen.
Die Autoren werden in der Datei \path{TIMESTAMP_cff_authors_new.csv} gespeichert, wobei \emph{TIMESTAMP} durch den Zeitpunkt der Änderung ersetzt wird.
Die Autoren aus der \grqq preferred-citation\glqq{} werden in der Datei \path{TIMESTAMP_cff_preferred_citation_authors_new.csv} gespeichert.
In beiden Dateien sind der Name, die E-Mail, die ORCID iD und die Git-Daten, welche durch den Abgleich ermittelt wurden, enthalten.
Die Felder der Dateien sind in \autoref{tab:cran_authors} aufgelistet.

Zusätzlich zu den neuesten Daten wird zu jedem Änderungszeitpunkt der Git-Prozess aus \autoref{subsec:datenbeschaffung_git} bis zu diesem Zeitpunkt durchlaufen.
Die Liste der Autoren wird in der Datei \path{TIMESTAMP_git_contributors.csv} gespeichert.
Die Liste enthält die Autoren, welche bis zu diesem Zeitpunkt Änderungen vorgenommen haben.
Außerdem wird die Liste der Autoren, welche aus der \gls{cff} extrahiert wurden, mit den Git-Autoren bis zu diesem Zeitpunkt abgeglichen.
Dadurch entstehen die beiden Dateien \path{TIMESTAMP_cff_authors.csv} und \path{TIMESTAMP_cff_preferred_citation_authors.csv}.
Enthalten sind die gleichen Informationen wie in den \glqq new\grqq{} Dateien.

Zudem können in beiden Fällen ähnlich wie bei der Beschreibung allgemeine Daten extrahiert werden.
Hierbei wird wie in der Beschreibung eine weitere CSV-Datei erstellt, welche die allgemeinen Daten enthält.
In dem Fall der \gls{cff} werden zwei allgemeine Dateien erstellt, einmal für die \gls{cff} und einmal für die \glqq preferred-citation\grqq{}.
Die Dateien heißen \path{cff.csv} und \path{cff_preferred_citation.csv}.
Die Felder der \path{cff.csv}-Datei sind in \autoref{tab:cff} aufgelistet und die der \path{cff_preferred_citation.csv}-Datei in \autoref{tab:cff_preferred_citation}.
In den beiden Dateien werden ebenfalls die Zeitpunkte \emph{commit date} und \emph{author date} gespeichert.

In den Dateien wird ebenfalls gespeichert, ob die \gls{cff}-Datei valide ist.
Hierbei wird das Programm \emph{cffconvert} in der Version 2.0.0 verwendet, um die Validität zu überprüfen \autocite{spaaks_cffconvert_2021}.
Außerdem benötigt \emph{cffconvert} die Programme \emph{jsonschema} und \emph{pykwalify}, welche in dieser Masterarbeit in der Version 4.23.0 und 1.8.0 verwendet werden \autocites{berman_jsonschema_2024}{grokzen_pykwalify_2020}.
Diese werden von \emph{cffconvert} benötigt, um die \gls{cff}-Datei zu validieren.
Zusätzlich wird gespeichert, ob die \gls{cff}-Datei mit dem Tool \emph{cffinit} erstellt wurde \autocite{spaaks_cffinit_2023}.
Dies ist möglich, da \emph{cffinit} in der \gls{cff}-Datei mehrere Kommentare einfügt, welche auf das Tool hinweisen.
Es wird davon ausgegangen, dass \emph{cffinit} benutzt wurde, falls der Kommentar \glqq This CITATION.cff file was generated with cffinit.\grqq{} in der \gls{cff}-Datei vorhanden ist.
Dies ist jedoch kein sicherer Indikator, da der Kommentar auch manuell hinzugefügt und vor allem entfernt werden kann.
Diese Werte werden jeweils in den allgemeinen Dateien gespeichert.
Dies führt zu einer doppelten Speicherung, allerdings ist dadurch in beiden Dateien separat zu erkennen, ob die \gls{cff}-Datei, aus denen die Daten extrahiert wurden, valide ist und ob \emph{cffinit} benutzt wurde.

Alle weiteren Informationen werden direkt aus der \gls{cff}-Datei extrahiert und in die eigene CSV Datenstruktur übertragen.
Falls ein Wert dabei nicht in der \gls{cff}-Datei vorhanden ist, wird der Eintrag in der CSV-Datei leer gelassen.
Im Folgenden werden die Daten, welche direkt extrahiert werden, beschrieben.
Eine extrahierte Information, welche in beiden Dateien, also der \path{cff.csv} und \path{cff_preferred_citation.csv} gespeichert wird, ist die \gls{doi} aus dem \gls{cff} Feld \glqq identifiers\grqq{}.
In dieser Masterarbeit wird ausschließlich die erste \gls{doi} in der Liste der \glqq identifiers\grqq{} betrachtet.
Diese wird in der allgemeinen Datei gespeichert, um später zu überprüfen, ob eine \gls{doi} in diesem Feld vorhanden ist.

Zusätzlich wird in beiden Fällen das Feld \glqq date-released\grqq{} und die \glqq doi\grqq{} extrahiert und gespeichert.
Die \glqq doi\grqq{} wird in den allgemeinen Dateien gespeichert, um im weiteren Verlauf zu überprüfen, ob eine \gls{doi} in der \gls{cff}-Datei vorhanden ist.
Als letzte gemeinsame Information wird die \glqq type\grqq{} extrahiert und in den allgemeinen Dateien gespeichert.
Dabei können wie in den Grundlagen erwähnt für die allgemeinen Informationen in der \gls{cff} nur \glqq software\grqq{} und \glqq dataset\grqq{} vorkommen.
Für die allgemeinen Daten der \glqq preferred-citation\grqq{} können weitere Typen vorkommen, welche in den Grundlagen erläutert wurden.

Alle weiteren Daten sind spezifisch für die \glqq preferred-citation\grqq{}.
Hierbei wird das Feld \glqq date-published\grqq{} extrahiert und in der allgemeinen Datei \path{cff_preferred_citation.csv} gespeichert.
Die Felder \glqq year\grqq{}, \glqq month\grqq{} und \glqq collection-doi\grqq{} werden ebenfalls extrahiert und in der allgemeinen Datei für die \glqq preferred-citation\grqq{} gespeichert.
Die Felder der Datei sind in \autoref{tab:cff_preferred_citation} aufgelistet.

\begin{table}
    \begin{tabularx}{\textwidth}{XL{10.2cm}}
        \toprule
        \textbf{Feld}              & \textbf{Beschreibung}                               \\ \midrule
        \emph{cff\_valid}          & Gibt an, ob die \gls{cff}-Datei valide ist          \\
        \emph{cff\_init}           & Gibt an, ob \emph{cff\_init} benutzt wurde          \\
        \emph{type}                & Typ der \gls{cff}-Datei                             \\
        \emph{date-released}       & Datum an dem das Werk zugänglich gemacht wurde      \\
        \emph{doi}                 & Zu zitierende \gls{doi}                             \\
        \emph{identifier-doi}      & Die erste \gls{doi} in den \glqq identifiers\grqq{} \\
        \emph{committed\_datetime} & Zeitpunkt des \emph{commit date}                    \\
        \emph{authored\_datetime}  & Zeitpunkt des \emph{author date}                    \\
        \bottomrule
    \end{tabularx}
    \caption{Felder der \texttt{cff.csv}-Datei}
    \label{tab:cff}
    \small
    Die Tabelle zeigt die Felder der \path{cff.csv}-Datei. Für jede Änderung der \gls{cff}-Datei werden die dargestellten Werte gespeichert. Die Felder \emph{date-released}, \emph{doi} und \emph{identifier-doi} sind optional.
\end{table}

\begin{table}
    \begin{tabularx}{\textwidth}{XL{10.2cm}}
        \toprule
        \textbf{Feld}              & \textbf{Beschreibung}                               \\ \midrule
        \emph{cff\_valid}          & Gibt an, ob die \gls{cff}-Datei valide ist          \\
        \emph{cff\_init}           & Gibt an, ob \emph{cff\_init} benutzt wurde          \\
        \emph{type}                & Typ der \gls{cff}-Datei                             \\
        \emph{date-released}       & Datum an dem das Werk zugänglich gemacht wurde      \\
        \emph{date-published}      & Veröffentlichungsdatum                              \\
        \emph{year}                & Veröffentlichungsjahr                               \\
        \emph{month}               & Veröffentlichungsmonat                              \\
        \emph{doi}                 & Zu zitierende \gls{doi}                             \\
        \emph{collection-doi}      & \gls{doi} der Sammlung                              \\
        \emph{identifier-doi}      & Die erste \gls{doi} in den \glqq identifiers\grqq{} \\
        \emph{committed\_datetime} & Zeitpunkt des \emph{commit date}                    \\
        \emph{authored\_datetime}  & Zeitpunkt des \emph{author date}                    \\
        \bottomrule
    \end{tabularx}
    \caption{Felder der \texttt{cff\_preferred\_citation.csv}-Datei}
    \label{tab:cff_preferred_citation}
    \small
    Die Tabelle zeigt die Felder der \path{cff_preferred_citation.csv}-Datei. Für jede Änderung der \gls{cff}-Datei werden die dargestellten Werte gespeichert, falls das Feld \grqq preferred-citation\glqq{} angegeben ist. Die Felder \emph{date-released}, \emph{date-published}, \emph{year}, \emph{month}, \emph{doi}, \emph{collection-doi} und \emph{identifier-doi} sind optional.
\end{table}
 % 2 Seiten
\subsection{\hologo{BibTeX}}
\label{subsec:datenbeschaffung_bibtex}
Dieser Unterabschnitt betrachtet die \path{CITATION.bib}-Datei zu jedem Änderungszeitpunkt, falls sie in dem zu untersuchenden Paket vorhanden ist.
Es wird die Software \emph{bibtexparser} in der Version 2.0.0b7 verwendet, um die \hologo{BibTeX}-Datei zu analysieren \autocite{weiss_python-bibtexparser_2024}.
Die Software bietet Möglichkeiten, um in Python \hologo{BibTeX}-Dateien zu lesen und zu schreiben.
Sie muss in einer Beta-Version verwendet werden, da Features, welche benötigt werden, noch nicht im offiziellen Release vorhanden sind.

Die Autoren werden aus der \hologo{BibTeX}-Datei extrahiert und mit den neuesten Git-Daten abgeglichen.
Die erstellte Liste wird in der Datei \path{TIMESTAMP_bib_authors_new.csv} gespeichert, wobei \emph{TIMESTAMP} durch den Zeitpunkt der Änderung ersetzt wird.
Dabei wird in der Datei ausschließlich der Name des Autors zuzüglich der Daten aus dem Abgleich mit Git gespeichert.
Die Felder der Datei sind in \autoref{tab:description_authors} aufgelistet.

Ebenfalls wird erneut der Git-Prozess aus \autoref{subsec:datenbeschaffung_git} bis zu diesem Zeitpunkt durchlaufen.
Die Liste wird in der Datei \path{TIMESTAMP_git_contributors.csv} gespeichert.
Der Inhalt ist identisch, wie bereits in den anderen Quellen beschrieben.
Außerdem wird die Liste der Autoren, welche aus der \hologo{BibTeX} extrahiert wurden, mit den Git-Autoren bis zu diesem Zeitpunkt abgeglichen.
Die Daten werden in der Datei \path{TIMESTAMP_bib_authors.csv} gespeichert und die Inhalte sind identisch mit der \glqq new\grqq{} Datei.
Hierbei wird ebenfalls ausschließlich der Name des Autors und die Daten aus dem Git-Abgleich gespeichert.

Zudem können erneut allgemeine Daten für die \hologo{BibTeX}-Datei extrahiert werden.
Hierbei wird eine weitere CSV-Datei erstellt, welche die allgemeinen Daten enthält.
Die Datei wird \path{bib.csv} genannt und enthält, wie bereits in den anderen Quellen, die Zeitpunkte \emph{commit date} und \emph{author date}.
Außerdem wird der Literaturtyp aus der Datei extrahiert und in der Datei gespeichert.
Die Literaturtypen können dabei unterschiedlich sein, wie in \autoref{subsec:bibtex_format} erläutert wurde.
Zusätzlich wird das Jahr und der Monat der Veröffentlichung extrahiert und gespeichert.
Diese Werte müssen nicht für jeden Literaturtypen vorhanden sein und können aus diesem Grund leer sein.
Ebenfalls müssen die \gls{doi} und die ISBN nicht für jeden Literaturtypen vorhanden sein, falls sie jedoch vorhanden sind, werden sie extrahiert und gespeichert.
Die Felder der Datei sind in \autoref{tab:bib} aufgelistet.

\begin{table}
    \begin{tabularx}{\textwidth}{XL{10.2cm}}
        \toprule
        \textbf{Feld}              & \textbf{Beschreibung}                  \\ \midrule
        \emph{type}                & Literaturtyp der \hologo{BibTeX}-Datei \\
        \emph{year}                & Veröffentlichungsjahr                  \\
        \emph{month}               & Veröffentlichungsmonat                 \\
        \emph{doi}                 & Zu zitierende \gls{doi}                \\
        \emph{isbn}                & Zu zitierende ISBN                     \\
        \emph{committed\_datetime} & Zeitpunkt des \emph{commit date}       \\
        \emph{authored\_datetime}  & Zeitpunkt des \emph{author date}       \\
        \bottomrule
    \end{tabularx}
    \caption{Felder der \texttt{bib.csv}-Datei}
    \label{tab:bib}
    \small
    Die Tabelle zeigt die Felder der \path{bib.csv}-Datei. Für jede Änderung der \hologo{BibTeX}-Datei werden die dargestellten Werte gespeichert. Die Felder \emph{year}, \emph{month}, \emph{doi} und \emph{isbn} sind abhängig vom Literaturtyp optional.
\end{table}
 % 2 Seiten
