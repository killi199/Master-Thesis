\section{Datenbeschaffung}
\label{sec:datenbeschaffung}
\subsection{Git}
\label{subsec:datenbeschaffung_git}
% TODO Beschreiben wie die Commits gezählt wurden.
\subsection{PyPI}
\label{subsec:datenbeschaffung_pypi}
% TODO erklären warum BigQuery nicht eingesetzt wird aktuell aber dennoch sinnvoll eingesetzt werden könnte bei anderen anforderungen (z.B. liste mit repo links auf github)
% TODO erklären warum die API verwendet wird und nicht nur die TOML beispielsweise ausgelesen wird -> habe ein pypi paket und brauche die GitHub URL
% TODO sagen warum beides also verifizierte Owner und Maintainer sowhl als auch die Daten aus der TOML abgefragt werden -> sind unterschiedliche Leute und werden unterschiedlich angegeben die einen dürfen sachen auf pypi machen die anderen werden in der toml angegeben und dürfen ggf. nichts in pypi machen
% Checken was verifizierte Nutzer im PyPI Universum bedeutet. Sind es nur verifizierte User die z.B. eine E-Mail hinterlegt haben oder sind es Nutzer die an dem Projekt arbeiten und von PyPI verifiziert sind? Falls es mehrwert hat diese Daten ebenfalls automatisch abfragen und auch in der MA beschreiben, was es nun ist und warum es Mehrwert hat oder auch nicht.
% TODO In MA beschreiben, warum oder warum nicht Mehrwert (PyPI Verifizierte Nutzer)
% TODO Sagen warum die Owner nicht berücksichtigt werden hatte dafür auch irgendwo eine Qulle -> es sind immer org.
% TODO erklären, dass ich den Namen des Betreuers brauche und nicht nur den Benutzernamen über die API und wie ich das gelöst habe mit einem WebScraper
\subsection{CRAN}
\label{subsec:datenbeschaffung_cran}
\subsection{Beschreibung}
\label{subsec:datenbeschaffung_beschreibung}
% TODO Es werden alle Namen ausgegeben aber auch eben solche, die gar nichts mit dem Paket zu tun haben wie im fall von highr für CRAN: "Provides syntax highlighting for R source Code. Currently it supports LaTeX and HTML output. Source Code of other languages is supported via Andre Simon's highlight package (https://gitlab.com/saalen/highlight)." Es gibt noch weitere Beispiele z.B. in CRAN magrittr -> vllt eher ergebnis?
\subsection{Citation File Format}
\label{subsec:datenbeschaffung_cff}
% TODO darauf eingehen, dass ich händisch aktuell geprüft habe ob das GH repo in cran oder pypi ist und daraus meine listen gebaut habe. Dies wäre aber auch automatisch möglich aber mit viel Aufwand und man bräuchte google big query für 100 pakete war dieser weg schneller außerdem gibt es auf PyPi mehrere Pakete die auf ein GitHub repo verlinken. Da müsste einiges beachtet werden um das genau zu machen für 100 war dies besser. Falls es für mehr gemacht werden sollte müsste das überdacht werden.
% TODO darauf eingehen wie die CFF geholt und verarbeitet wurde mit stars verknüpft und dann händisch pypi cran raus gesucht und den namen auf pypi
% TODO in der gegebenen Liste waren mehrere Repos auf GH doppelt mit unterschiedlichen Links. diese wurden entfernt.
\subsection{\hologo{BibTeX}}
\label{subsec:datenbeschaffung_bibtex}
