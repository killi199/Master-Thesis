\section{Abgleich}
\label{sec:abgleich}
Dieser Abschnitt beschreibt den Abgleich von Autoren aus verschiedenen Quellen mit den Git-Autoren eines Pakets.
Der entwickelte Algorithmus nutzt die in den Quellen verfügbaren Daten (Name, E-Mail, Benutzername), um Übereinstimmungen zu ermitteln.
Dabei wird keine \gls{ned} verwendet, sondern ein an die Daten angepasster Ansatz.

Die Git-Autorenliste wird nach der Anzahl der Commits sortiert, um Autoren mit mehr Commits zu bevorzugen.
Anschließend werden Autoren aus den Quellen mit den Git-Autoren verglichen. Der Abgleich erfolgt anhand von:

\begin{enumerate}
    \item \textbf{Namen:} Namen werden in Kleinbuchstaben umgewandelt, um Schreibweisen zu vereinheitlichen. Mithilfe des Python-Keywords \emph{in} wird geprüft, ob ein Teil des Namens in einem anderen enthalten ist. Dieses Keyword erlaubt eine einfache Teilstring-Suche. Ergänzend wird \emph{thefuzz} in der Version 0.22.1 für eine unscharfe Suche verwendet. Dabei wird eine Ähnlichkeit von mindestens 80 \% als Übereinstimmung gewertet. Dies hilft, Namen mit leichten Abweichungen korrekt zuzuordnen.
    \item \textbf{E-Mails:} Die E-Mails werden ebenfalls mit \emph{in} abgeglichen, um zu prüfen, ob eine der Adressen (ganz oder teilweise) in der anderen enthalten ist. Zusätzlich wird \emph{thefuzz} verwendet, um Ähnlichkeiten von mindestens 80 \% zu erkennen.
    \item \textbf{Benutzernamen:} Benutzernamen aus den Quellen werden mit Teilen der Git-E-Mails (lokaler oder Domain-Teil) verglichen, da Git keine Benutzernamen direkt bereitstellt. Auch hier kommt das Keyword \emph{in} zum Einsatz, um einfache Übereinstimmungen zu finden, ergänzt durch die unscharfe Suche mit \emph{thefuzz}.
\end{enumerate}

Für jeden Vergleich wird ein Score berechnet, der die Übereinstimmungen relativ zur maximal möglichen Anzahl von Feldern darstellt.
Der Git-Autor mit dem höchsten Score wird zugeordnet.
Falls zwei Einträge in der Git-Liste existieren, welche einen gleichen Score erreichen, wird der Autor ausgewählt, welcher die meisten Commits hat.
Falls keine Übereinstimmung gefunden wird, bleiben die Felder leer.

Das Ergebnis enthält die Autoren aus den Quellen, ergänzt um die Ergebnisse des Abgleichs, welche in \autoref{tab:abgleich_felder} dargestellt sind.
