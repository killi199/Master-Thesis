\section{Abgleich} % 2 Seiten
\label{sec:abgleich}
% 2,5 Seiten
% TODO Erwähnen, dass immer der mit dem besten Score ausgewählt wird. Und falls beide den gleichen Score dann der mit mehr commits.
% TODO Es wird immer passieren, dass manche Leute falsch zugeordnet werden. Aktuelles bsp. in scipy heißt jemand pv auf PyPI und jemand hat eine E-Mail mit: pvanmulbregt@users.noreply.github.com die werden gematcht... Also nur weil ich einen Match habe bedeutet es nicht, dass es ein richtiger ist! Dies unbedingt in der Arbeit berücksichtigen. Ebenfalls, dass falls kein match gefunden wurde heißt es nicht zwangsweise, dass mein Script schlecht arbeitet es kann auch sein, das jemand als Autor genannt wird aber keinen Code geschrieben hat oder das eine Organisation als Autor angegeben wurde.
% TODO beschreiben, dass all contributors ein Problem im  matching darstellen, da autoren ohne commits natürlich nicht mit den contributorn von github gematch werden können -> daher händisch durchgegangen für 200 pakete und geguckt welche nicht gemachts werden können um gefühlt für die dimension zu erhalten -> auf Ergebnis nochmal referenzieren wo die daten angegeben sind
% TODO sagen, dass gleiche leute nicht erkannt werden und der mit den meisten commits dann genommen wird
% thefuzz==0.22.1
