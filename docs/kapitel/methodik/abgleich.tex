\section{Abgleich}
\label{sec:abgleich}
% TODO Grafik erstellen, die das ganze nochmal visualisiert
In diesem Abschnitt wird beschrieben, wie die Autoren aus den unterschiedlichen Quellen, welche in der Datenbeschaffung erläutert wurden, abgeglichen werden.
Dabei werden jeweils die Git Autoren mit den Autoren aus den anderen Quellen abgeglichen.
Für diesen Prozess wird keine \gls{ned} verwendet, sondern ein eigener Algorithmus, welcher auf die Daten angepasst ist.
Dieser wurde entwickelt, indem die Autoren, welche durch die Datenbeschaffung erhalten wurden, in den Paketen eingesehen wurden.
So wurde durch viel Probieren eine möglichst gute Lösung gefunden, welche im Folgenden beschrieben wird.
Dies führt allerdings dazu, dass der entwickelte Algorithmus nicht in der Lage ist zwei Autoren mit dem gleichen Namen zu unterscheiden, falls keine weiteren Daten wie eine E-Mail vorhanden sind.

Als Eingabe erhält der Algorithmus eine Liste von Autoren, welche in einer Quelle (z. B. \gls{cff}) gefunden wurden und jeweils die Liste der Git Autoren für das zu untersuchende Paket.
In der Datenbeschaffung wurde gezeigt, dass die Autoren aus den Quellen einen Namen, eine E-Mail und einen Benutzernamen enthalten können, abhängig von der Quelle.
In dem Algorithmus wird ermittelt, wie viele von den Daten vorhanden sind, um im Verlauf einen Score berechnen zu können wie viele der Daten übereinstimmen.
Falls Beispielsweise die Quelle \gls{cff} ist, kann der Name und die E-Mail vorhanden sein.
Dadurch ergibt sich, dass maximal zwei Daten mit denen von Git übereinstimmen können.
Falls die Quelle die Beschreibung ist, ist maximal eine Übereinstimmung möglich.
Diese Daten werden für den Abgleich verwendet.
Die eingegebene Liste der Git Autoren muss für den Algorithmus nach der Anzahl der Commits sortiert sein, da der Algorithmus Personen mit den meisten Commits bevorzugt.
Anschließend wird jeder Autor aus der Quelle mit jedem Git Autor abgeglichen.
Dabei werden alle Werte in Kleinbuchstaben umgewandelt, um zu gewährleisten, dass auch bei unterschiedlicher Schreibweise eine Übereinstimmung gefunden werden kann.

Dabei wird so vorgegangen, dass die Daten, welche vorhanden sind, zum Vergleich herangezogen werden.
Falls der Name vorhanden ist, wird dieser mit dem Namen des Git Autors über das Keyword \emph{in} in Python verglichen.
Dieses sorgt dafür, dass nur ein Teil der Zeichenkette in der anderen Zeichenkette vorhanden sein muss.
Außerdem wird die andere Richtung ebenfalls probiert.
Zusätzlich wird das Programm \emph{thefuzz} in der Version 0.22.1 verwendet, um eine unscharfe Suche zu ermöglichen.
Hierbei wird der Name des Autors aus der Quelle mit dem Namen des Git Autors verglichen und auf eine Übereinstimmung von 80 \% oder mehr geprüft.
Die Datenbeschaffung hat gezeigt, dass die Namen der Autoren manchmal Klammern enthalten, welche den beschriebenen Abgleich stören.
Aus diesem Grund wird der zuvor beschriebene Abgleich erneut durchgeführt, wobei die Klammern in dem Namen entfernt worden sind.

Falls eine E-Mail vorhanden ist, wird diese mit der E-Mail des Git Autors über das Keyword \emph{in} verglichen.
Die andere Richtung wird ebenfalls probiert, also die E-Mail des Git Autors mit der E-Mail des Autors aus der Quelle über das Keyword \emph{in} zu vergleichen.
Zusätzlich wird erneut die unscharfe Suche zwischen der E-Mail des Autors aus der Quelle und der E-Mail des Git Autors durchgeführt und auf eine Übereinstimmung von 80 \% oder mehr geprüft.
Falls kein Abgleich über den Namen stattgefunden hat, wird ein Vergleich zwischen der E-Mail des Autors aus der Quelle und dem Namen des Git Autors über das Keyword \emph{in} durchgeführt.
Außerdem wird der Abgleich erneut über die unscharfe Suche mit einer Übereinstimmung von 80 \% oder mehr durchgeführt.
Dies führt dazu, dass weitere Übereinstimmungen gefunden werden können, falls der Name zu keiner Übereinstimmung geführt hat.
Dies liegt daran, dass viele Autoren in ihrer E-Mail-Adresse ihren Namen enthalten haben.
Außerdem hat die Datenbasis gezeigt, dass es vorkommt, dass in dem E-Mail Feld keine E-Mail angegeben wurde, sondern ein Name und als Resultat das Namensfeld nicht ausgefüllt wurde.

Falls ein Benutzername vorhanden ist, wird dieser mit der E-Mail des Git Autors verglichen, da aus den Git Daten kein Benutzername extrahiert werden kann.
Dies ist möglich, da viele Autoren für ihren Benutzernamen den lokal Teil der E-Mail verwenden.
Für den Vergleich wird die E-Mail, falls sie ein @ Symbol enthält, an dieser Stelle getrennt.
Anschließend wird der vordere lokale Teil für den Vergleich mit dem Benutzernamen aus der Quelle verwendet.
Dabei wird der Vergleich erneut über das Keyword \emph{in} in beiden Richtungen durchgeführt.
Außerdem wird über die unscharfe Suche mit einer Übereinstimmung von 80 \% oder mehr geprüft, ob eine Übereinstimmung gefunden wurde.
Dieser Prozess wird ebenfalls erneut für den Domain-Teil der E-Mail durchgeführt, sodass dieser ebenfalls mit dem Benutzernamen verglichen wird.

Falls keiner dieser drei Vergleiche zu einem Erfolg geführt hat, wird der Name des Autors mit dem lokalen Teil der E-Mail des Git Autors über die unscharfe Suche verglichen.
Dabei muss erneut eine Übereinstimmung von 80 \% oder mehr vorhanden sein, um den Vergleich als gelungen zu bewerten.

Anschließend wird der Score berechnet.
Dabei wird die Anzahl der Übereinstimmungen durch die Anzahl der maximal möglichen Übereinstimmungen geteilt.
Falls beispielsweise der Name und die E-Mail vorhanden sind, sind maximal zwei Übereinstimmungen möglich.
Bei einer Übereinstimmung des Namens und die E-Mail ergibt sich ein Score von 1.
Falls nur der Name übereinstimmt, ergibt sich ein Score von 0,5.
Dieser Score wird für jeden Autor aus der Quelle mit jedem Git Autor berechnet und in der Tabelle der Autoren aus der Quelle gespeichert.
Anschließend wird der Autor aus der Quelle mit dem Git Autor mit dem besten Score ausgewählt.
Falls es zwei Einträge in der Git Liste gibt, welche einen gleichen Score erreichen, wird der Autor ausgewählt, welcher die meisten Commits hat.
Außerdem wird der Rang, welcher der Autor in der Git Autoren Liste belegt zurückgegeben.
Als Ergebnis des Abgleichs werden alle Autoren aus der Quelle, welche in den Algorithmus eingegeben wurde, zurückgegeben.
Die Autoren werden dabei mit den Ergebnissen des Abgleichs verbunden, welche in \autoref{tab:abgleich_felder} dargestellt sind.
Das Ergebnis des Abgleichs ist in \autoref{tab:abgleich_felder} dargestellt.
Falls kein Abgleich für einen Autor möglich war, werden die Felder für diesen Autor leer gelassen.
Anschließend wird das Ergebnis nach dem Rang sortiert, also nach der Anzahl der Commits, welche der Autor hat.
