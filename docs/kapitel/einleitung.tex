\chapter{Einleitung}
\label{chap:einleitung}
\section{Motivation}
\label{sec:motivation}
Die Wissenschaft ist in den letzten Jahren immer stärker digitalisiert worden und ist viel stärker auf \gls{oss} angewiesen als noch vor einigen Jahren.
In der Wissenschaft werden längst nicht mehr nur Paper und Bücher veröffentlicht, sondern auch Software, Daten und andere digitale Artefakte.
Dies steigert die Ausführbarkeit und Nachvollziehbarkeit der Arbeiten.
Der Wandel der Wissenschaft bringt allerdings die Herausforderung mit sich, digitale Artefakte angemessen zu würdigen.
Neue Zitierformate und -weisen werden unerlässlich, um diesen Ansprüchen gerecht zu werden.
Außerdem benötigt es eine gerechte Würdigung der einzelnen Autoren\footnote{Aus Gründen der besseren Lesbarkeit wird auf die gleichzeitige Verwendung der Sprachformen männlich, weiblich und divers (m/w/d) verzichtet. Sämtliche Personenbezeichnungen gelten gleichermaßen für alle Geschlechter.}, die an der Entwicklung von \gls{oss} beteiligt sind.

Bei der Entwicklung von \gls{oss} arbeiten Softwareentwickler zum Teil in ihrer Freizeit, um die Software zu verbessern.
Dabei kommt es immer wieder vor, dass ihre Arbeit nicht gewürdigt wird, da sie nicht als Autor der Software oder sogar ihres Beitrags genannt werden.
Dies ist beispielsweise dem Entwickler Ariel Miculas widerfahren, welcher einen Beitrag zum Linux-Kernel geleistet hat und nicht als Autor genannt wurde \autocite{miculas_how_2023}.
Im Gegensatz zu wissenschaftlichen Arbeiten ist die Autorschaft von Software nicht semantisch geklärt \autocite{schmidt_software_nodate}.

Um zu prüfen, welchen Einfluss dies auf \gls{oss} hat, stellen sich folgende Forschungsfragen:

\begin{itemize}
    \item[\textbf{F1}] Wie gut können Autoren abgeglichen werden?
    \item[\textbf{F2}] Was muss ein Softwareentwickler leisten, um als Autor genannt zu werden?
    \item[\textbf{F3}] Wie gut werden Autoren gepflegt?
\end{itemize}

\section{Vorgehen}
\label{sec:vorgehen}
In der Arbeit werden Autoren aus unterschiedlichen Quellen extrahiert.
Eine wichtige Quelle ist Git, welche als Referenz genutzt wird.
Andere Quellen, wie beispielsweise das \gls{cff}, werden mit den Git-Autoren abgeglichen.
Dadurch ist es möglich, Daten zu erheben, die zeigen, wie einzelne Git-Autoren in den Quellen repräsentiert werden.
Die Daten werden anschließend Graphisch aufbereitet und präsentiert.

Es muss berücksichtigt werden, dass in der Masterarbeit ausschließlich Git-Autoren betrachtet werden.
Dadurch werden nur Personen betrachtet, die Code beigetragen haben.
Allerdings wird dadurch nicht das gesamte Spektrum von Beiträgen zur Software in der Masterarbeit abgedeckt, da beispielsweise Grafikdesigner nicht genannt werden.
Für die Nennung jener Autoren ohne Code-Beiträge gibt es andere Arbeitsgruppen, die sich mit dem Problem beschäftigen.
Diese sind ebenfalls wichtig für die Entwicklung von Software und sollten berücksichtigt werden.

\section{Gliederung}
\label{sec:gliederung}
In der Masterarbeit wird Software erzeugt, die es erlaubt, für eine beliebige Liste an \gls{cran} oder \gls{pypi} Repositorys Autorenangaben aus verschiedenen Datenquellen zu extrahieren, diese im Anschluss zu aggregieren und graphisch aufzuarbeiten.
Dabei werden in \autoref{chap:grundlagen} die benötigten Grundlagen geschaffen, um beispielsweise die Themen Versionsverwaltung, Paketverwaltung, Software-Verzeichnisse und im Allgemeinen verschiedene Zitierformate für Software verstehen zu können.
Sie werden benötigt, um anschließend in \autoref{chap:methodik} Software zu entwickeln, über die Autorenangaben aus verschiedenen Datenquellen extrahiert und abgeglichen werden können.
Anschließend werden die Ergebnisse in \autoref{chap:ergebnisse} präsentiert.
Dabei wird auf Ergebnisse der Gegenwart und jene mit Zeitverlauf eingegangen.
Im anschließenden \autoref{chap:diskussion} werden die Probleme und Herausforderungen der Arbeit diskutiert.
Außerdem werden die Forschungsfragen beantwortet.
Im \autoref{cap:fazit_ausblick} wird die Arbeit zusammengefasst und ein Ausblick auf weitere Forschungsansätze gegeben, welche mit dieser Arbeit möglich geworden sind.
