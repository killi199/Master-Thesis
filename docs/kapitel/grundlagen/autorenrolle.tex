\section{Autorenrolle und Anerkennung in Open-Source-Software}
\label{sec:autorenrolle-oss}
% TODO ggf. noch mehr schreiben?
% TODO ggf. weitere Paper?
In der Wissenschaft existieren bereits Analysen zur Rolle von Autoren in \gls{oss}.
Ein Paper analysiert die Zuschreibung von Entwicklern in \gls{oss} \autocite{young_which_2021}.
Sie beantworten in dem Paper folgende Fragen:

\begin{itemize}
  \item Wie unterscheiden sich die Modelle für die Anerkennung von Beiträgen?
  \item Wie viele Informationen gehen verloren, bei einer Festlegung auf Repository-Änderungen als Modell für den Beitrag?
  \item Wie entwickeln sich die Beiträge zu \gls{oss} über die Zeit?
  \item Können wir Projekte auf der Grundlage von Beitragsmustern klassifizieren?
\end{itemize}

In dem Paper werden vier Modelle für die Anerkennung unterschieden.
Im ersten Modell werden die Änderungen an einem Repository als Beitrag betrachtet.
Dabei wird auf die Top 100 Benutzer in den Projekten geschaut, wie sie durch die GitHub API ausgegeben werden.
Das zweite Modell betrachtet die Autoren, welche automatisch über ein Tool identifiziert werden.
In dem Paper wird hierbei das Programm \textit{octohatrack} verwendet \autocites{young_which_2021}{noauthor_labhroctohatrack_2024}.
Als drittes Modell werden die Autoren mittels Taxonomien identifiziert.
Hierbei wird die \mintinline{text}{.all-contributorsrc}-Datei von \glqq All Contributors\grqq{} analysiert \autocites{young_which_2021}{all_contributors_recognize_2024}.
Das vierte Modell betrachtet die Autoren, welche mittels ad hoc Methoden identifiziert werden.
Diese können beispielsweise durch die Analyse von nicht standardisierten Quellen stammen wie beispielsweise Webseiten oder unstrukturierten Textdateien.
Das Modell wird in dem Paper aufgrund der Komplexität nicht fokussiert betrachtet.

Die Autoren betrachten die Fragen und Modelle dabei von einer gehobenen Perspektive.
Dabei werden die einzelnen Fragen mithilfe von verschiedenen generellen Metriken wie die Anzahl der Autoren, welche Commits erstellt haben oder als Autoren in \glqq All Contributors\grqq{} genannt werden, beantwortet.
Sie betrachten keine einzelnen Autoren oder Projekte, sondern analysieren die gesamte \gls{oss}-Landschaft primär auf Basis von der Anzahl der Autoren.
Es werden keine genaueren Analysen durchgeführt wie beispielsweise in dieser Masterarbeit, in der unter anderem betrachtet wird, ob die genannten Autoren noch aktiv an dem Projekt beteiligt sind.
Außerdem wird nicht auf Daten eingegangen, welche aus weiteren Quellen wie dem \gls{cff} oder \hologo{BibTeX} Format stammen.
