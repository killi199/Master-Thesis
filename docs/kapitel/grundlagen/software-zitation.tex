\section{Zitation von Software}
\label{sec:software-zitation}
Software ist ein wesentlicher Bestandteil moderner Forschung.
In der wissenschaftlichen Literatur ist es üblich, Quellen zu zitieren, um die Nachvollziehbarkeit und Reproduzierbarkeit von wissenschaftlichen Arbeiten zu gewährleisten.
Im Gegensatz dazu ist dies bei wissenschaftlicher Software aktuell in diesem Umfang noch nicht gegeben.
Hier gibt es aktuell kaum Anerkennung und Unterstützung für die Leistungen einzelner Autoren.
Aus diesem Grund hat die \glqq FORCE11 Software Citation Working Group\grqq{} Prinzipien der Software-Zitation erstellt, welche eine breite Akzeptanz in der wissenschaftlichen Gemeinschaft finden sollen.
Im Folgenden werden die Prinzipien vorgestellt und erläutert \autocite{smith_software_2016}:

\begin{enumerate}
    \item \textbf{Wichtigkeit:} Software sollte ein seriöses und zitierbares Produkt wissenschaftlicher Arbeit sein. Software-Zitierungen sollten im wissenschaftlichen Kontext die gleiche Bedeutung zugeschrieben bekommen wie Zitierungen anderer Forschungsprodukte. Sie sollten wie Publikationen auch in der Arbeit enthalten sein, z.~B. in der Referenzliste eines Artikels.
    \item \textbf{Anerkennung und Zuschreibung:} Softwarezitate sollten die wissenschaftliche Anerkennung und die normative, rechtliche Würdigung aller Mitwirkenden an der Software ermöglichen, wobei anerkannt wird, dass ein einziger Stil oder ein Mechanismus für die Namensnennung nicht auf jede Software anwendbar sein kann.
    \item \textbf{Eindeutige Identifikation:} Ein Softwarezitat sollte eine Methode zur Identifikation enthalten, die maschinell verwertbar, weltweit eindeutig und interoperabel ist und zumindest von einer Gemeinschaft der entsprechenden Fachleute und vorzugsweise von allgemeinen Forschern anerkannt wird.
    \item \textbf{Persistenz:} Eindeutige Identifikatoren und Metadaten, die die Software und ihre Verwendung beschreiben, sollten bestehen bleiben – auch über die Lebensdauer der Software hinaus.
    \item \textbf{Zugänglichkeit:} Softwarezitate sollten den Zugang zur Software selbst und zu den zugehörigen Metadaten, Dokumentationen, Daten und anderen Materialien erleichtern, die sowohl für Menschen als auch für Maschinen notwendig sind, um die referenzierte Software sachkundig nutzen zu können.
    \item \textbf{Spezifizität:} Softwarezitate sollten die Identifikation und den Zugang zu der spezifischen Version der verwendeten Software erleichtern. Die Identifizierung der Software sollte so spezifisch wie nötig sein, z.~B. durch Versionsnummern, Revisionsnummern oder Varianten wie Plattformen.
\end{enumerate}

In dieser Arbeit wird verstärkt auf das Prinzip der Wichtigkeit eingegangen, da im \gls{cff}- und \hologo{BibTeX}-Format die Option besteht, anstelle der Software beispielsweise einen Artikel anzugeben.
Diese Zitierweise würde dann das Prinzip der Wichtigkeit verletzen, da die Software nicht die gleiche Bedeutung zugeschrieben bekommt wie andere Forschungsprodukte.
Diese Diskrepanz wird in \autoref{chap:diskussion} dargestellt.

Im Folgenden werden einige Gründe genannt, warum die Zitation von Software ebenfalls wichtig ist und auch, dass Standards der Zitation eingehalten werden \autocite{smith_software_2016}.

\begin{itemize}
    \item \textbf{Forschungsfelder verstehen:} Software ist ein Produkt der Forschung und wenn sie nicht zitiert wird, werden Lücken in der Aufzeichnung der Forschung über den Fortschritt in diesem Forschungsfeld entstehen.
    \item \textbf{Anerkennung:} Akademische Forscher auf allen Ebenen, einschließlich Studenten, Postdocs, Dozenten und Mitarbeiter, sollten für die Softwareprodukte, die sie entwickeln und zu denen sie beitragen, anerkannt werden, insbesondere wenn diese Produkte die Forschung anderer ermöglichen oder fördern. Nicht-akademische Forscher sollten ebenfalls für ihre Softwarearbeit anerkannt werden, obwohl die spezifischen Formen der Anerkennung sich von denen für akademische Forscher unterscheiden.
    \item \textbf{Software entdecken:} Mithilfe von Zitaten kann die in einem Forschungsprodukt verwendete Software gefunden werden. Weitere Forscher können dann dieselbe Software für andere Zwecke verwenden, was zu einer Anerkennung der Softwareverantwortlichen führt.
    \item \textbf{Reproduzierbarkeit:} Die Angabe der verwendeten Software ist für die Reproduzierbarkeit notwendig, aber nicht ausreichend. Zusätzliche Informationen wie Konfigurationen und Probleme auf der Plattform sind ebenfalls erforderlich.
\end{itemize}

Wie bereits erwähnt, werden Autoren in unterschiedlichen Quellen genannt.
Einige dieser Quellen werden in dieser Masterarbeit untersucht, wie beispielsweise das \gls{cff}- und \hologo{BibTeX}-Format.
Die Autoren in diesen Quellen werden vom jeweiligen Softwareprojekt angegeben.
Falls an dem Projekt nicht nur Softwareentwickler beteiligt sind, sondern beispielsweise auch Grafikdesigner oder Übersetzer, so sollten diese ebenfalls in den Quellen zitiert werden.
Dies ist wichtig, da diese Personen ebenfalls einen Beitrag zum Projekt geleistet haben und somit auch Anerkennung verdienen.

Es gibt ebenfalls Projekte, welche sich für die halb-automatische Nennung von Autoren ohne Codebeitrag einsetzen.
Ein Beispiel für ein solches Projekt ist \glqq All Contributors\grqq{} \autocite{bolam_recognize_2024}.
Bei der Verwendung von \glqq All Contributors\grqq{} werden die Mitwirkenden standardmäßig in der README-Datei angegeben, welche im Projekt enthalten ist.
Außerdem wird eine \mintinline{text}{.all-contributorsrc}-Datei erstellt, welche im JSON-Format die Mitwirkenden und deren Beiträge enthält.
Die Liste wird automatisch generiert und aktualisiert, sodass sie den Spezifikationen von \glqq All Contributors\grqq{} entspricht.
Über einen Befehl in GitHub kann ein neuer Mitwirkender hinzugefügt werden.
Autoren, welche keinen Code beigetragen haben, stellen im weiteren Verlauf dieser Masterarbeit ein Problem dar.
Sie können nicht mit den Autoren aus den Quellen abgeglichen werden, da sie keine Commits haben.
Dies wird in \autoref{sec:abgleich} genauer erläutert.

% TODO ggf. darauf noch eingehen? https://google.github.io/opencasebook/authorship/
In dieser Arbeit werden Pakete untersucht, bei denen der Quellcode öffentlich auf GitHub zugänglich ist und von einer Gemeinschaft an Entwicklern weiterentwickelt werden kann.
Die Entwickler arbeiten dabei in der Regel ehrenamtlich und ohne Bezahlung an der Software, wobei einige der untersuchten Pakete auch von Unternehmen veröffentlicht werden, wie beispielsweise die \glqq google-auth-library-python\grqq{} von Google.
Welche Autoren in den Quellen angegeben werden sollten, ist jedoch nicht so genau definiert, wie es beispielsweise im Bereich von wissenschaftlich-medizinischen Artikeln der Fall ist.
In diesem Bereich hat das \glqq International Commitee of Medical Journal Editors\grqq{} Richtlinien für die Rolle von Autoren und Beitragenden in wissenschaftlichen Artikeln definiert \autocite{icmje_icmje_2024}.
Aus diesem Grund ist die Menge der Autoren, welche in den Quellen angegeben werden, unterschiedlich und hängt von dem jeweiligen Projekt ab.
Außerdem ist es möglich, dass ausschließlich Autoren in den Quellen angegeben werden, welche aktuell nicht mehr an dem Projekt beteiligt sind.
