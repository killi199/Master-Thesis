\section{Zitation von Software}
\label{sec:software-zitation}
% TODO auf die einzelnen Prinzipien eingehen besonders auf Wichtigkeit weil das hier für die Arbeit interessant ist
% TODO beschreiben warum Zitation von Software genauso wichtig ist wie die Zitation von anderen wissenschaftlichen Arbeiten
% TODO auf notwendigkeit von Software Zitation eingehen und warum Zitation von Software wichtig ist
% TODO Auch Zitation von Personen, welche nicht aktiv an der Software entwickeln (All Contributors) beschreiben warum diese auch zitiert werden sollten
% TODO auf Autorenrolle in Open Source Software eingehen
Software ist ein wesentlicher Bestandteil moderner Forschung.
In der wissenschaftlichen Literatur ist es üblich, Quellen zu zitieren, um die Nachvollziehbarkeit und Reproduzierbarkeit von wissenschaftlichen Arbeiten zu gewährleisten.
Im Gegensatz dazu ist dies bei wissenschaftlicher Software aktuell in diesem Umfang noch nicht gegeben.
Hier gibt es aktuell kaum Anerkennung und Unterstützung für die Leistungen einzelner Autoren.
Aus diesem Grund hat die \glqq FORCE11 Software Zitier Arbeitsgruppe\grqq{} Prinzipien der Software Zitation erstellt, welche eine breite Akzeptanz in der wissenschaftlichen Gemeinschaft finden sollen \cite{smith_software_2016}.
Im Folgenden werden die Prinzipien vorgestellt und erläutert.

\begin{enumerate}
    \item \textbf{Wichtigkeit:} Software sollte ein seriöses und zitierbares Produkt wissenschaftlicher Arbeit sein. Software Zitierungen sollten im wissenschaftlichen Kontext die gleiche Bedeutung zugeschrieben bekommen wie Zitierungen anderer Forschungsprodukte, wie Publikationen. Sie sollten wie Publikationen auch in der Arbeit enthalten sein, zum Beispiel in der Referenzliste eines Artikels. Software sollte auf derselben Grundlage zitiert werden wie jedes andere Forschungsprodukt auch, wie zum Beispiel ein Aufsatz oder ein Buch. Das bedeutet, dass Autoren die entsprechend verwendete Software zitieren sollten, so wie sie die entsprechenden Publikationen zitieren würden.
    \item \textbf{Anerkennung und Zuschreibung:} Softwarezitate sollten die wissenschaftliche Anerkennung und die normative, rechtliche Würdigung aller Mitwirkenden an der Software ermöglichen, wobei anerkannt wird, dass ein einziger Stil oder ein Mechanismus für die Namensnennung nicht auf jede Software anwendbar sein kann.
    \item \textbf{Eindeutige Identifikation:} Ein Softwarezitat sollte eine Methode zur Identifikation enthalten, die maschinell verwertbar, weltweit eindeutig und interoperabel ist und zumindest von einer Gemeinschaft der entsprechenden Fachleute und vorzugsweise von allgemeinen Forschern anerkannt wird.
    \item \textbf{Persistenz:} Eindeutige Identifikatoren und Metadaten, die die Software und ihre Verwendung beschreiben, sollten bestehen bleiben – auch über die Lebensdauer der Software hinaus, die sie beschreiben.
    \item \textbf{Zugänglichkeit:} Softwarezitate sollten den Zugang zur Software selbst und zu den zugehörigen Metadaten, Dokumentationen, Daten und anderen Materialien erleichtern, die sowohl für Menschen als auch für Maschinen notwendig sind, um die referenzierte Software sachkundig nutzen zu können.
    \item \textbf{Spezifizität:} Softwarezitate sollten die Identifizierung und den Zugang zu der spezifischen Version der verwendeten Software erleichtern. Die Identifizierung der Software sollte so spezifisch wie nötig sein, z. B. durch Versionsnummern, Revisionsnummern oder Varianten wie Plattformen.
\end{enumerate}
