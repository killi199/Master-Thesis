\section{Zitation von Software}
\label{sec:software-zitation}
Software ist ein wesentlicher Bestandteil moderner Forschung.
In der wissenschaftlichen Literatur ist es üblich, Quellen zu zitieren, um die Nachvollziehbarkeit und Reproduzierbarkeit von wissenschaftlichen Arbeiten zu gewährleisten.
Im Gegensatz dazu ist dies bei wissenschaftlicher Software aktuell in diesem Umfang noch nicht gegeben.
Hier gibt es aktuell kaum Anerkennung und Unterstützung für die Leistungen einzelner Autoren.
Aus diesem Grund hat die \glqq FORCE11 Software Zitier Arbeitsgruppe\grqq{} Prinzipien der Software Zitation erstellt, welche eine breite Akzeptanz in der wissenschaftlichen Gemeinschaft finden sollen.
Im Folgenden werden die Prinzipien vorgestellt und erläutert \cite{smith_software_2016}:

\begin{enumerate}
    \item \textbf{Wichtigkeit:} Software sollte ein seriöses und zitierbares Produkt wissenschaftlicher Arbeit sein. Software Zitierungen sollten im wissenschaftlichen Kontext die gleiche Bedeutung zugeschrieben bekommen wie Zitierungen anderer Forschungsprodukte, wie Publikationen. Sie sollten wie Publikationen auch in der Arbeit enthalten sein, zum Beispiel in der Referenzliste eines Artikels. Software sollte auf derselben Grundlage zitiert werden wie jedes andere Forschungsprodukt auch, wie zum Beispiel ein Aufsatz oder ein Buch. Das bedeutet, dass Autoren die entsprechend verwendete Software zitieren sollten, so wie sie die entsprechenden Publikationen zitieren würden.
    \item \textbf{Anerkennung und Zuschreibung:} Softwarezitate sollten die wissenschaftliche Anerkennung und die normative, rechtliche Würdigung aller Mitwirkenden an der Software ermöglichen, wobei anerkannt wird, dass ein einziger Stil oder ein Mechanismus für die Namensnennung nicht auf jede Software anwendbar sein kann.
    \item \textbf{Eindeutige Identifikation:} Ein Softwarezitat sollte eine Methode zur Identifikation enthalten, die maschinell verwertbar, weltweit eindeutig und interoperabel ist und zumindest von einer Gemeinschaft der entsprechenden Fachleute und vorzugsweise von allgemeinen Forschern anerkannt wird.
    \item \textbf{Persistenz:} Eindeutige Identifikatoren und Metadaten, die die Software und ihre Verwendung beschreiben, sollten bestehen bleiben – auch über die Lebensdauer der Software hinaus, die sie beschreiben.
    \item \textbf{Zugänglichkeit:} Softwarezitate sollten den Zugang zur Software selbst und zu den zugehörigen Metadaten, Dokumentationen, Daten und anderen Materialien erleichtern, die sowohl für Menschen als auch für Maschinen notwendig sind, um die referenzierte Software sachkundig nutzen zu können.
    \item \textbf{Spezifizität:} Softwarezitate sollten die Identifizierung und den Zugang zu der spezifischen Version der verwendeten Software erleichtern. Die Identifizierung der Software sollte so spezifisch wie nötig sein, z. B. durch Versionsnummern, Revisionsnummern oder Varianten wie Plattformen.
\end{enumerate}

In dieser Arbeit wird verstärkt auf das Prinzip der Wichtigkeit eingegangen, da besonders im \gls{cff} und Bib\TeX{} Format die Möglichkeit besteht nicht die Software, sondern beispielsweise ein Artikel anzugeben.
Diese Zitierweise würde dann das Prinzip der Wichtigkeit verletzen, da die Software nicht die gleiche Bedeutung zugeschrieben bekommt wie andere Forschungsprodukte.
Diese Diskrepanz wird in \autoref{chap:ergebnisse} dargestellt.

Es gibt verschiedene Gründe, warum die Zitation von Software ebenfalls wichtig ist und auch, dass Standards der Zitation eingehalten werden, ähnlich wie es der Fall bei anderen wissenschaftlichen Arbeiten ist.
Einige dieser Gründe werden im Folgenden genannt \autocite{smith_software_2016}:

\begin{itemize}
    \item Forschungsfelder verstehen: Software ist ein Produkt der Forschung und wenn sie nicht zitiert wird, werden Lücken in der Aufzeichnung der Forschung über den Fortschritt in diesem Forschungsfeld entstehen.
    \item Anerkennung: Akademische Forscher auf allen Ebenen, einschließlich Studenten, Postdocs, Dozenten und Mitarbeiter, sollten für die Softwareprodukte, die sie entwickeln und zu denen sie beitragen, anerkannt werden, insbesondere wenn diese Produkte die Forschung anderer ermöglichen oder fördern. Nicht-akademische Forscher sollten ebenfalls für ihre Softwarearbeit anerkannt werden, obwohl die spezifischen Formen der Anerkennung sich von denen für akademische Forscher unterscheiden.
    \item Software entdecken: Mithilfe von Zitaten kann die in einem Forschungsprodukt verwendete Software gefunden werden. Weitere Forscher können dann dieselbe Software für andere Zwecke verwenden, was zu einer Anerkennung der für die Software Verantwortlichen führt.
    \item Reproduzierbarkeit: Die Angabe der verwendeten Software ist für die Reproduzierbarkeit notwendig, aber nicht ausreichend. Zusätzliche Informationen wie Konfigurationen und Probleme auf der Plattform sind ebenfalls erforderlich.
\end{itemize}

Wie bereits erwähnt werden Autoren in unterschiedlichen Quellen angegeben.
Einige dieser Quellen werden in dieser Masterarbeit untersucht, wie beispielsweise das \gls{cff} und Bib\TeX{} Format.
Die Autoren in diesen Quellen werden von dem jeweiligen Softwareprojekt angegeben.
Falls an dem Projekt nicht nur Softwareentwickler beteiligt sind, sondern beispielsweise auch Grafikdesigner oder Übersetzer, so sollten diese ebenfalls in den Quellen angegeben werden.
Dies ist wichtig, da diese Personen ebenfalls einen Beitrag zum Projekt geleistet haben und somit auch Anerkennung verdienen.

Es gibt ebenfalls bereits Projekte, welche sich für die halb automatische Nennung von Autoren ohne Code Beitrag einsetzen.
Ein Beispiel für ein solches Projekt ist \glqq All Contributors\grqq{} \autocite{all_contributors_recognize_2024}.
Bei der Verwendung von \glqq All Contributors\grqq{} werden die Mitwirkenden jedoch nicht in einem speziellen Format angegeben, sondern standardmäßig in der README-Datei, welche im Projekt enthalten ist.
Die Liste wird dabei von einem Bot automatisch generiert und aktualisiert, sodass sie der Spezifikation von \glqq All Contributors\grqq{} entspricht.
Über einen Befehl in einem Issue oder einem Pull Request kann ein neuer Mitwirkender hinzugefügt werden.
Autoren, welche keinen Code beigetragen haben, stellen im weiteren Verlauf der Masterarbeit ein Problem dar, da diese nicht mit den Autoren aus den Quellen abgeglichen werden können, da sie keine Commits haben.
Dies wird in \autoref{sec:abgleich} genauer erläutert.

% TODO darauf noch eingehen? https://google.github.io/opencasebook/authorship/
Die untersuchten Pakete in dieser Arbeit sind Open Source, welche auf GitHub veröffentlicht werden.
Open Source Software ist Software, deren Quellcode öffentlich zugänglich ist und von einer Gemeinschaft von Entwicklern entwickelt wird.
Die Entwickler arbeiten dabei in der Regel ehrenamtlich und ohne Bezahlung an der Software, wobei einige der untersuchten Pakete auch von Organisationen veröffentlicht werden, wie beispielsweise die \glqq google-auth-library-python
\grqq{} von Google.
Diese wird primär von Google Mitarbeitern entwickelt und gepflegt.
Wie bereits beschrieben sind Prinzipien und Gründe definiert, warum Software ebenfalls zitiert werden sollte.
Welche Autoren in den Quellen angegeben werden sollten, ist jedoch nicht so genau definiert wie es beispielsweise im Bereich von wissenschaftlichen medizinischen Artikel der Fall ist.
In diesem Bereich hat das \glqq International Commitee of Medical Journal Editors\grqq{} Richtlinien für die Rolle von Autoren und Beitragenden in wissenschaftlichen Artikeln definiert \autocite{noauthor_icmje_nodate}.
Unter anderem aus diesem Grund ist die Menge der Autoren, welche in den Quellen angegeben werden, unterschiedlich und hängt von dem jeweiligen Projekt ab.
Außerdem ist es möglich, dass ausschließlich Autoren in den Quellen angegeben werden, welche aktuell nicht mehr an dem Projekt beteiligt sind, aber beispielsweise vor fünf Jahren aktiv das Projekt geführt haben.
Diese und weitere Auffälligkeiten werden in der Masterarbeit untersucht.
Auch dies ist in anderen Bereichen anders definiert, sodass auch die neuen Autoren genannt werden müssen wie beispielsweise bei Neuauflagen von Büchern.
