\section{Named Entity Disambiguation}
\label{sec:author-name-disambiguation}
% TODO ggf. noch weiter ausführen?
Die \gls{ned} ist ein automatischer Prozess, bei dem ein Name einer Entität einer gegebenen Datenmenge zugeordnet wird \autocites{cucerzan_large-scale_2007}{yamada_global_2022}.
Die Entitäten können beispielsweise durch die \gls{ner} extrahiert worden sein.
Dabei kann es dazu kommen, dass eine Entität mehrfach extrahiert wird und mehrfach in der Datenmenge vorhanden ist, jedoch mit anderen Bedeutungen.
In der natürlichen Sprachverarbeitung ist dies das Problem der Polysemie, auf das bereits eingegangen wurde.
Es ist aber auch möglich, dass Personen mit dem gleichen Namen, sogenannte Namensvetter extrahiert werden, welche unterschieden werden müssen.
Dies beschreibt die Author name disambiguation, welche konkret individuelle Personen disambiguiert und ein Teil der \gls{ned} ist.
Diese Probleme kann die \gls{ner} mittels Modellen lösen, die die Entitäten anhand von Kontexten disambiguieren.
Beispiele für erkannte Entitäten sind \autocite{cucerzan_large-scale_2007}:

\begin{itemize}
  \item George W. Bush (George W. Bush)
  \item George Bush (George W. Bush)
  \item Bush (George W. Bush)
  \item Reggie Bush (Reggie Bush)
  \item Bush (Reggie Bush)
  \item Bush (Rock band)
\end{itemize}

In dem Beispiel ist die Entität dargestellt, gefolgt von der Entität in der Datenmenge, welche in Klammern steht.
Hierbei ist auffällig, dass Nennungen von \glqq Bush\grqq{} mehrfach vorkommen und durch den Kontext, in dem sie stehen, durch die \gls{ned} unterschieden werden müssen.

\gls{ned} wird in vielen Bereichen eingesetzt, wie z. B. Textanalysen, semantische Suche und der Gruppierung von Software-Nennungen in wissenschaftlichen Arbeiten \autocites{cucerzan_large-scale_2007}{yamada_global_2022}{schindler_somesci-_2021}.
In der Masterarbeit kann die \gls{ned} verwendet werden, um Autoren aus verschiedenen Quellen miteinander abzugleichen.

Ähnlich zu der \gls{ner} gibt es für die \gls{ned} verschiedene Modelle.
Viele Modelle sind jedoch spezifisch auf einzelne Aufgabenbereiche trainiert.
Allgemeine Modelle sind primär für die Disambiguierung von Entitäten in Texten trainiert, welche in der Arbeit nicht immer vorhanden sind.
Ein Beispiel ist ein Modell, welches auf BERT basiert und Wörter, sowohl als auch Entitäten als Tokens erhält und diese mit Entitäten aus einem Text disambiguiert \autocite{yamada_global_2022}.
