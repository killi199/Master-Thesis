\section{Unscharfe Suche}
\label{sec:unscharfe-suche}
% TODO ggf. Levenshtein Distance und Damerau-Levenshtein-Distanz leicht erklären
% TODO ggf. noch weiter ausführen?
Die unscharfe Suche ist ein Verfahren, um ähnliche Zeichenfolgen zu finden, die sich in ihrer Schreibweise unterscheiden \autocite{hall_approximate_1980}.
Dieses Verfahren hat viele Anwendungsgebiete in der Informatik.
Ein Beispiel ist das Finden eines Personennamens in einem Index.
Falls der Name exakt in dem Index vorhanden ist, ist die Suche trivial.
Falls der Name unterschiedlich geschrieben ist, beispielsweise durch Abkürzungen oder Tippfehler, wird die triviale Suche fehlschlagen.
Die unscharfe Suche kann in diesem Fall helfen, den Namen trotzdem zu finden.
Ein weiteres Anwendungsgebiet ist eine allgemeine Suche Beispielsweise von Produkten in einem Online-Shop.
Hierbei müssen auch Tippfehler, welche häufig auftreten berücksichtigt werden.
Dadurch ist die Suche einer Zeichenfolge, welcher nahezu korrekt ist ein häufiges Problem in der Informatik.

Die unscharfe Suche basiert auf der Levenshtein-Distanz \autocite{levenshtein_binary_1965}.
Als Ergebnis der unscharfen Suche wird in vielen Implementierungen die Distanz zwischen zwei Zeichenfolgen in Prozent angegeben.
Die Levenshtein-Distanz verwendet drei Arten von einzelzeichen-basierten Editieroperationen, um die Distanz zwischen zwei Zeichenfolgen zu berechnen.

\begin{enumerate}
    \item Einfügen eines Zeichens zur Zeichenkette (Suhe \rightarrow{} Su\textbf{c}he)
    \item Löschen eines Zeichens aus der Zeichenkette (Suche\textbf{e} \rightarrow{} Suche)
    \item Ersetzen eines Zeichens in der Zeichenkette (S\textbf{i}che \rightarrow{} S\textbf{u}che)
\end{enumerate}

Zusätzlich zu der Levenshtein-Distanz existiert eine Erweiterung, die Damerau-Levenshtein-Distanz \autocite{damerau_technique_1964}.
Diese erweitert die Levenshtein-Distanz um eine vierte Editieroperation.
Mittels dieser vier Operationen werden ungefähr 80 \% der menschlichen Tippfehler abgedeckt \autocite{damerau_technique_1964}.

\begin{enumerate}
    \setcounter{enumi}{3}
    \item Vertauschen von zwei benachbarten Zeichen in der Zeichenkette (Su\textbf{hc}e \rightarrow{} Su\textbf{ch}e)
\end{enumerate}

Eine Herausforderung bei der unscharfen Suche ist die Laufzeit.
Sie ist ungefähr proportional zum Produkt der beiden Zeichenfolgenlängen, wodurch eine unscharfe Suche von langen Zeichenfolgen unpraktisch wird.
Eine weitere Herausforderung ist das Finden des richtigen Prozentsatzes, um die unscharfe Suche zu verwenden.
Ein zu hoher Prozentsatz führt dazu, dass die Suche zu viele Ergebnisse zurückgibt, während ein zu niedriger Prozentsatz dazu führt, dass die Suche zu wenige Ergebnisse zurückgibt.

Für die unscharfe Suche gibt es viele Implementierungen, die auf verschiedenen Algorithmen basieren.
Ein Programm, welches in Python implementiert ist, ist \emph{TheFuzz} \autocite{noauthor_seatgeekthefuzz_2024}.
Das Programm basiert auf \emph{RapidFuzz}, welches die Levenshtein-Distanz in Python und C++ implementiert \autocites{noauthor_rapidfuzzrapidfuzz_2024}.
