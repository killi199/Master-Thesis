\chapter{Grundlagen}
\label{chap:grundlagen}
% TODO ggf. TP, FN, FP, TN erklären und auch f1 score, f Score
In \autoref{sec:software-zitation} wird auf die Prinzipien der Software-Zitation eingegangen.
Es wird beschrieben, warum die Zitation von Software ebenfalls wichtig ist, ähnlich wie die Zitation von anderen wissenschaftlichen Arbeiten.
Außerdem wird darauf eingegangen, dass ebenfalls Personen zitiert werden sollten, welche nicht aktiv an der Software programmieren.
Zusätzlich dazu wird in \autoref{sec:autorenrolle-oss} auf die Rolle von Autoren in \gls{oss} eingegangen und ein wissenschaftliches Paper dargestellt, welches dies bereits analysiert hat.

Autoren von Software werden in unterschiedlichen Quellen zitiert.
Einige dieser Quellen sind stark mit der Softwareentwicklung verbunden.
Es existieren verschiedene Systeme, die Entwicklern zur Verfügung stehen, um ihre Arbeit effizienter zu gestalten oder überhaupt praktikabel zu machen.
In diesen Systemen können Sie außerdem als Autoren genannt werden.
In den Abschnitten \ref{sec:versionsverwaltung} und \ref{sec:paketverwaltung} wird auf die Versions- und Paketverwaltung eingegangen, welche zwei dieser Systeme darstellen.
Des Weiteren existieren spezielle Zitierformate, in welchen Autoren explizit angegeben werden können.
Auf diese Formate wird in \autoref{sec:zitierformate} eingegangen.
Zudem können in Fließtexten, beispielsweise der Beschreibung einer Software, ebenfalls Autoren genannt werden.
In \autoref{sec:named-entity-recognition} wird auf die \emph{Named Entity Recognition} eingegangen, welche eine Methode darstellt, um Personen in Texten zu erkennen.

Alle Quellen, welche beschrieben werden, dienen im Verlauf der Masterarbeit als Grundlage für die Extraktion von Autoren und deren Metainformationen.
Die extrahierten Autoren müssen anschließend zugeordnet werden.
Der Prozess dafür heißt \emph{Author Name Disambiguation}, welcher in \autoref{sec:author-name-disambiguation} beschrieben wird.
Eine weitere Möglichkeit des Abgleichs ist ein Abgleich von Zeichenfolgen.
Dieser funktioniert jedoch nicht immer, da Autoren unterschiedliche Schreibweisen ihres Namens verwenden können.
Aus diesem Grund wird in \autoref{sec:unscharfe-suche} auf die unscharfe Suche eingegangen, welche eine Möglichkeit darstellt, um ähnliche Zeichenfolgen miteinander zu vergleichen, beispielsweise für den Abgleich von Namen mit oder ohne genannten Zwischennamen.
\section{Zitation von Software}
\label{sec:software-zitation}
% TODO Irgendwo auf die Sachen eingehen: https://github.com/citation-file-format/citation-file-format/blob/main/schema-guide.md#preferred-citation
% TODO https://peerj.com/articles/cs-86/
% TODO auf notwendigkeit von Software Zitation eingehen und warum Zitation von Software wichtig ist
% TODO Auch Zitation von Personen, welche nicht aktiv an der Software entwickeln (All Contributors)

\section{Autorenrolle und Anerkennung in Open-Source-Software}
\label{sec:autorenrolle-oss}
% TODO ggf. noch mehr schreiben?
% TODO ggf. weitere Paper?
In der Wissenschaft existieren bereits Analysen zur Rolle von Autoren in \gls{oss}.
Ein Paper analysiert die Zuschreibung von Entwicklern in \gls{oss} \autocite{young_which_2021}.
Sie beantworten in dem Paper folgende Fragen:

\begin{itemize}
  \item Wie unterscheiden sich die Modelle für die Anerkennung von Beiträgen?
  \item Wie viele Informationen gehen verloren, bei einer Festlegung auf Repository-Änderungen als Modell für den Beitrag?
  \item Wie entwickeln sich die Beiträge zu \gls{oss} über die Zeit?
  \item Können wir Projekte auf der Grundlage von Beitragsmustern klassifizieren?
\end{itemize}

In dem Paper werden vier Modelle für die Anerkennung unterschieden.
Im ersten Modell werden die Änderungen an einem Repository als Beitrag betrachtet.
Dabei wird auf die Top 100 Benutzer in den Projekten geschaut, wie sie durch die GitHub API ausgegeben werden.
Das zweite Modell betrachtet die Autoren, welche automatisch über ein Tool identifiziert werden.
In dem Paper wird hierbei das Programm \emph{octohatrack} verwendet \autocites{young_which_2021}{noauthor_labhroctohatrack_2024}.
Als drittes Modell werden die Autoren mittels Taxonomien identifiziert.
Hierbei wird die \mintinline{text}{.all-contributorsrc}-Datei von \glqq All Contributors\grqq{} analysiert \autocites{young_which_2021}{all_contributors_recognize_2024}.
Das vierte Modell betrachtet die Autoren, welche mittels ad hoc Methoden identifiziert werden.
Diese können beispielsweise durch die Analyse von nicht standardisierten Quellen stammen wie beispielsweise Webseiten oder unstrukturierten Textdateien.
Das Modell wird in dem Paper aufgrund der Komplexität nicht fokussiert betrachtet.

Die Autoren betrachten die Fragen und Modelle dabei von einer gehobenen Perspektive.
Dabei werden die einzelnen Fragen mithilfe von verschiedenen generellen Metriken wie die Anzahl der Autoren, welche Commits erstellt haben oder als Autoren in \glqq All Contributors\grqq{} genannt werden, beantwortet.
Sie betrachten keine einzelnen Autoren oder Projekte, sondern analysieren die gesamte \gls{oss}-Landschaft primär auf Basis von der Anzahl der Autoren.
Es werden keine genaueren Analysen durchgeführt wie beispielsweise in dieser Masterarbeit, in der unter anderem betrachtet wird, ob die genannten Autoren noch aktiv an dem Projekt beteiligt sind.
Außerdem wird nicht auf Daten eingegangen, welche aus weiteren Quellen wie dem \gls{cff} oder \hologo{BibTeX} Format stammen.

\section{Versionsverwaltung}
\label{sec:versionsverwaltung}
Die Versionsverwaltung ist ein System, um verschiedene Versionen von Software zu verwalten.
Es bietet Zugang zu Code und dessen Änderungen in der Vergangenheit.
Der Code und getätigte Änderungen werden in einem Repository gespeichert.
Dadurch ist die Versionsverwaltung eine Art Logbuch, in dem alle Änderungen festgehalten werden.
Dabei wird zusätzlich zu der Änderung der Autor und der Zeitpunkt der Änderung festgehalten \autocite{ponuthorai_version_2022}.
Dies ermöglicht es in der Masterarbeit empirisch die Menge an Arbeit der einzelnen Autoren zu ermitteln.

Zusätzlich zum Code können in einem Repository andere Dateien, wie beispielsweise eine README, eine Lizenz, und Zitationsinformationen, beispielsweise in Form einer \gls{cff}-Datei, gespeichert werden.
Eine README-Datei ist eine Datei, welche Informationen über das Projekt enthält, beispielsweise wie es installiert und verwendet wird.
Sie wird standardmäßig im Stammverzeichnis des Repositorys gespeichert und wird an dieser Stelle auch von Diensten wie GitHub dargestellt.

Es gibt zwei verschiedene Arten von Versionsverwaltungssystemen.
Zum einen gibt es die zentralen Systeme, bei denen alle Änderungen zentral verwaltet werden, beispielsweise SVN.
Zum anderen gibt es die verteilten Systeme, bei denen jeder Entwickler eine Kopie des gesamten Repositorys und dessen Vergangenheit hat \autocite{ponuthorai_version_2022}.
Ein solches System ist Git, welches sich mit einem Marktanteil von ungefähr 75 \% gegenüber anderen Systemen durchgesetzt hat \autocite{lindner_version_2024}.
Aus diesem Grund und weil Git-Repositorys in der Arbeit untersucht werden, wird auf Git eingegangen.
Dabei werden Begriffe erklärt, mit denen es möglich ist, die geleistete Arbeit von einzelnen Autoren innerhalb eines Repositorys zu untersuchen.
Außerdem wird auf grundlegende Funktionen von Git eingegangen, da diese für die Arbeit relevant sind.
Der Aufbau der einzelnen Git-Komponenten ist in \autoref{fig:git} dargestellt.
Dabei ist zu erkennen, dass Git in einen Git-Server und Git-Anwendungen aufgeteilt wurde.

\begin{figure}
    {   
        \begin{center}
            \begin{tikzpicture}
                [
                    block/.style = {rectangle, draw, fill=blue!20, text width=8cm, text centered, minimum height=1.5cm, rounded corners=0.2cm},
                    client/.style = {rectangle, draw, fill=blue!20, text width=3.5cm, text centered, minimum height=1.5cm, rounded corners=0.2cm},
                    arrow/.style = {thick,<->,>=stealth}
                ]
        
                \node (server) [block] {Git-Repository-Hosting-Plattform};
                \node (cli) [client, below=of server, xshift=-2.25cm] {Git-Befehlszeile};
                \node (gui) [client, below=of server, xshift=+2.25cm] {Git-GUI Anwendungen};
            
                \draw[arrow] (cli.north) -- (server.south -| cli.north);
                \draw[arrow] (gui.north) -- (server.south -| gui.north);
                \draw[arrow] (cli) -- (gui);

                \node[left=0.1cm of server] {Git-Server};
                \node[left=0.1cm of cli] {Git-Anwendungen};
            
            \end{tikzpicture}
        \end{center}
        \caption{Übersicht über die Git-Komponenten}
        \label{fig:git}
        \small
        Die Git-Komponenten bestehen aus einem Git-Server, welcher das Repository hostet, und Git-Anwendungen, welche auf das Repository zugreifen (indirekt aus \cite{ponuthorai_version_2022}).
    }
\end{figure}

Bei der Benutzung von Git ist ein Server nicht zwingend erforderlich, jedoch steigert dies die Komplexität der Verwaltung und ist komplizierter in der Handhabung.
Der Git-Server ermöglicht die einfache kollaborative Entwicklung von Code, da dieser ständig erreichbar ist und zentral verwaltet wird \autocite{ponuthorai_version_2022}.
Standardmäßig wird auf dem Git-Server die neueste Version des Repositorys gespeichert.
Ein möglicher Git-Server ist GitHub, auf welchen im späteren Verlauf weiter eingegangen wird.
Git-Anwendungen sind Programme, welche mit dem lokalen Repository interagieren \autocite{ponuthorai_version_2022}.
Diese können auf entfernte Git-Repositorys zugreifen, wie beispielsweise Repositorys, welche auf GitHub gehostet werden.
Anschließend arbeiten die Programme auf der lokalen Kopie und können die Änderungen, wenn nötig, auf das entfernte Repository übertragen.

In Repositorys werden verschiedene Arten von Statistiken gespeichert.
Git verwaltet Revisionen als \emph{Snapshot}.
Anders als in anderen Systemen wird keine Serie von Änderungen gespeichert, sondern ein \emph{Snapshot} der Änderungen zu einem bestimmten Zeitpunkt erstellt \autocite{ponuthorai_version_2022}.
Dies wird ein Commit genannt.
An einem Commit werden verschiedene Metainformationen gespeichert, beispielsweise eine Commit-Nachricht, der Autor und der Zeitpunkt der Änderungen.

Die Änderung wird dabei als zwei Zeitpunkte angegeben.
Zum einen wird der Zeitpunkt der Änderung des Autors angegeben, also jener Zeitpunkt, zu dem der Autor die Änderung vorgenommen und committet hat, dies wird in Git \emph{author date} genannt.
Zum anderen wird der Zeitpunkt des Einfügens des Commits in das Repository gespeichert, dies wird in Git \emph{commit date} genannt.
Der Commit kann von einer anderen Person, z. B. durch einen Projektverantwortlichen, mittels eines Pull Requests in das Repository übernommen worden sein.
Durch dieses Verhältnis ist der \emph{commit date} Zeitpunkt immer später oder gleich dem \emph{author date} Zeitpunkt.
Außerdem ist gewährleistet, dass beide verantwortlichen Anerkennung für die geleistete Arbeit erhalten \autocite{chacon_pro_2024}.

Die Commit-Nachricht, sowie der Autor mit E-Mail-Adresse und Namen können in den Einstellungen von Git frei gewählt werden, müssen jedoch vorhanden sein, um einen Commit erstellen zu können.
Mehrere Commits bilden die Commit-Historie bzw. die Vergangenheit eines Repositorys.
Weitere Eigenschaften, welche sich aus dem Repository exportieren lassen, sind die Anzahl der eingefügten und gelöschten Zeilen.
Zudem lässt sich die Anzahl der geänderten Dateien ermitteln.
Diese Werte können für das gesamte Repository oder für einzelne Autoren ermittelt werden.

Ein Repository kann verschiedene Branches enthalten, muss jedoch mindestens einen enthalten.
In der Vergangenheit wurde der Standardbranch \emph{master} genannt.
Seit 2020 wird dieser jedoch in \emph{main} umbenannt, um rassistische Konnotationen zu vermeiden \autocite{github_githubrenaming_2024}.
Ein Branch ist eine separate Entwicklungslinie, welche unabhängig von anderen Branches ist.
Beim Erstellen von einem Branch wird der aktuelle Zustand des Branches, auf welchem der neue Branch erstellt wird, kopiert \autocite{ponuthorai_version_2022}.
Dadurch können Änderungen in dem neuen Branch durchgeführt werden, ohne dass diese Änderungen den ursprünglichen Branch beeinflussen.
Diese Änderungen werden mittels Commits festgehalten.
Unterschiedliche Branches können anschließend zusammengeführt werden, um die Änderungen in einem Branch in einen anderen Branch zu übernehmen.

Die Statistiken der Repositorys können auf verschiedene Arten aufgearbeitet werden.
Zum einen können einige direkt mittels Git-Befehlen ausgelesen werden \autocite{chacon_git_2024}.
Andere wiederum benötigen komplexere Abfragen, welche beispielsweise mittels Skripten oder speziellen Programmen ausgelesen werden können.
Ein Beispiel für ein Programm, welches Git-Statistiken aufarbeitet, ist \emph{git-quick-stats} \autocite{mestan_git-quick-stats_2024}.
Außerdem bieten Onlinedienste zur Versionsverwaltung, wie GitHub, Statistiken über APIs an, welche jedoch im Umfang der Anfragen limitiert sind \autocite{github_rate_2022}.
Bei der Benutzung der API von GitHub zum Abfragen der Autoren eines Repositorys werden automatisch alle E-Mail-Adressen der Autoren in Git mit den E-Mail-Adressen, welche die Autoren in GitHub angegeben haben, abgeglichen \autocite{github_rest-api-endpunkte_2022}.
Dadurch werden die Autoren eindeutig zugeordnet und deren Commits addiert.
Diese Werte werden ebenfalls in der Weboberfläche von GitHub angezeigt.

GitHub ist eine Plattform, auf welcher Git-Repositorys gehostet werden können und dient somit als ein Git-Server.
GitHub bietet zusätzliche Funktionen an, welche über die Standardfunktionen von Git hinausgehen.
Diese umfassen unter anderem die kollaborative Entwicklung von Code, Automatisation mittels CI/CD, Sicherheitsaspekte, Projektmanagement, Team Administration und Client-Anwendungen zur Verwaltung von Repositorys \autocite{ponuthorai_version_2022}.
Aktuell benutzen GitHub über 100 Millionen Entwickler und mehr als 4 Millionen Organisationen.
Insgesamt verwaltet die Plattform über 420 Millionen Repositorys.
Außerdem ist GitHub in 90 \% der Fortune 100 Unternehmen im Einsatz \autocite{github_about_2024}.
Um die zusätzlichen Funktionen von GitHub bereitzustellen, werden sogenannte Issues, Pull Requests, geschützte Branches, Actions, Diskussionen und Wikis eingesetzt.
GitHub-Issues sind eine Möglichkeit, um Probleme und Aufgaben zu verfolgen.
Pull Requests dienen dazu, Änderungen in einem Branch eines Repositorys anzufragen und über diese zu informieren.
In dem Pull Requests kann der Code überprüft und diskutiert werden.

\section{Paketverwaltung}
\label{sec:paketverwaltung}

\section{Zitierformate}
\label{sec:zitierformate}
In diesem Abschnitt wird auf unterschiedliche Zitierformate eingegangen, welche die Datenstruktur hinter einer Zitation beschreiben.
In dieser Arbeit wird sich auf das \gls{cff} und das \hologo{BibTeX}-Format beschränkt.
Das \gls{cff} ist ein Format, welches speziell für die Zitation von Software entwickelt wurde, weshalb es in dieser Arbeit besonders interessant ist.
Das \hologo{BibTeX}-Format wird dazu verwendet, um zumeist in Verbindung mit \LaTeX{}, Bibliographien zu erstellen und ist daher ebenfalls von Interesse, da es auch für Software verwendet werden kann.

\subsection{Citation File Format}
\label{subsec:citation-file-format}
Das \gls{cff} ist ein Format, welches in der \mintinline{text}{CITATION.cff}-Datei gespeichert wird und in YAML 1.2 geschrieben wird. 
Das Format beschreibt die Zitation von Software und kann von Menschen und Maschinen gelesen werden.
Es enthält Metadaten, welche für die Zitation von Software benötigt werden.
Außerdem wird es öffentlich auf GitHub verwaltet.
Auf GitHub enthalten 2.512 Repositorys eine \mintinline{text}{CITATION.cff}-Datei (Stand 07.11.2024).
Softwareentwickler können das \gls{cff} in ihre Repositorys einbinden, um anderen die Zitation ihrer Software zu erleichtern und vorzugeben, wie die Software richtig zu zitieren ist \autocite{druskat_citation_2021}.

Da die Datei von Menschen gelesen werden kann, kann diese manuell erstellt werden und in das Repository eingebunden werden.
Die Spezifikationen für das \gls{cff} werden auf GitHub verwaltet und sind öffentlich einsehbar \autocite{druskat_citation_2021}.
Ebenfalls existieren Programme, welche das \gls{cff} verarbeiten können.
Beispielsweise kann das Programm \emph{cffinit} genutzt werden, um eine \mintinline{text}{CITATION.cff}-Datei zu erstellen, sodass der Prozess der Erstellung vereinfacht wird \autocite{spaaks_cffinit_2023}.
Ein weiteres Beispiel ist das Programm \emph{cffconvert}, welches das \gls{cff} in verschiedene Formate umwandeln kann, wie z.~B. \hologo{BibTeX} oder RIS.
Außerdem kann das Programm genutzt werden, um \gls{cff}-Dateien zu validieren \autocite{spaaks_cffconvert_2021}.

Zusätzlich wird das \gls{cff} von unterschiedlichen Plattformen unterstützt, wie z.~B. von GitHub.
Erkennt GitHub eine \mintinline{text}{CITATION.cff}-Datei im Repository auf dem Standardbranch, wird sie automatisch auf der Repository-Startseite verlinkt und kann direkt im \hologo{BibTeX}-Format kopiert werden \autocites{druskat_citation_2021}{github_about_2024-1}.
Ebenfalls ist es möglich, die in der Datei eingetragenen Autoren in der APA-Zitierweise zu kopieren.
Die Funktionen sind in \autoref{fig:gh_cff_link} dargestellt.

\begin{figure}
    \begin{center}
      \includegraphics[width=0.5\textwidth]{bilder/GH_CFF_link.png}
    \end{center}
    \caption{GitHub-Repository mit \mintinline{text}{CITATION.cff}-Datei}
    \label{fig:gh_cff_link}
    \small
    Die Abbildung stellt den Link auf die \mintinline{text}{CITATION.cff}-Datei dar, wie ihn GitHub aktuell darstellt.
    Außerdem ist die Möglichkeit sichtbar, die Datei im \hologo{BibTeX}-Format und in der APA-Zitierweise zu kopieren \autocite{druskat_citation_2021-1}.
\end{figure}

In dem \gls{cff} existieren verschiedene Felder, welche für die Zitation von Software relevant sind.
Das wichtigste Feld ist das \emph{authors}-Feld, welches die Autoren der Software enthält und zwingend erforderlich ist.
In diesem Feld können die Autoren als Liste angegeben werden.
Ein Autor ist dabei entweder eine Person oder eine Entität.
Eine Entität kann beispielsweise eine Organisation sein.
Die Entität kann mit einem Namen mittels \emph{name} angegeben werden \autocite{druskat_citation_2021}.
Sie kann ebenfalls eine OORCID iD und eine E-Mail-Adresse enthalten.
Besonders wichtig für diese Arbeit ist die Referenz auf eine Person, da dies die einzige Information ist, welche aus Git extrahiert werden kann.
Eine Person enthält ebenfalls die genannten Werte einer Entität und wird jedoch über den Vor- und Nachname separiert mittels \emph{given-names} und \emph{family-names} angegeben.
Dadurch ist es möglich, die Personen von den Entitäten zu unterscheiden.

Ein weiteres Feld ist das \emph{preferred-citation}-Feld.
Mit diesem Feld ist es möglich, die Anerkennung für die Arbeit auf eine andere Arbeit zu übertragen \autocite{druskat_citation_2021}.
Ein Beispiel hierfür ist ein Paper über die Software, welches bevorzugt zitiert werden soll, anstelle der eigentlichen Software.
Hierbei können ebenfalls Personen und Entitäten angegeben werden.
Durch die Angabe einer \emph{preferred-citation} kann das Prinzip der Wichtigkeit vernachlässigt werden.
Auf dieses Verhalten wird in der Diskussion konkreter eingegangen.

Weitere in dieser Arbeit verwendete Felder sind \emph{type}, \emph{year}, \emph{month}, \emph{date-released}, \emph{date-published}, \emph{doi}, \emph{collection-doi} und \emph{identifiers}.
Das Feld \emph{type} ist zwingend erforderlich, hat jedoch als Standardwert \glqq software\grqq{}, sodass dies nicht angegeben werden muss.
Es gibt an, ob es sich um eine Software oder einen Datensatz handelt.
Dabei sind lediglich die Werte \glqq software\grqq{} oder \glqq dataset\grqq{} erlaubt.
Das Feld \emph{date-released} gibt an, wann die Software oder der Datensatz veröffentlicht wurde.
Das Feld \emph{doi} kann einen \glspl{doi} der Software oder des Datensatzes enthalten.
Mittels \emph{identifiers} können weitere Identifikatoren angeführt werden, wie z.~B. eine \gls{doi} oder eine URL, wobei die \emph{identifiers} mit einer Beschreibung erweitert werden können.
Die beschriebenen Felder können zusätzlich alle unter dem Feld \emph{preferred-citation} angegeben werden, um eine andere Arbeit zu referenzieren.

Die Felder \emph{year}, \emph{month} und \emph{date-published} können zusätzlich zu dem Feld \emph{date-released} unter dem Feld \emph{preferred-citation} angeführt werden, um das Jahr und das Datum der Veröffentlichung anzugeben.
Außerdem können weitere Typen mittels \emph{type} aufgeführt werden, wie beispielsweise \glqq thesis\grqq{} oder \glqq manual\grqq{}, sodass diese Arbeiten ebenfalls referenziert werden können.
Zusätzlich kann das Feld \emph{collection-doi} verwendet werden, um auf eine Sammlung von Arbeiten zu verweisen, die die Arbeit enthält.
Ein Beispiel einer \mintinline{text}{CITATION.cff}-Datei ist in \autoref{lst:cff_example} dargestellt.
Dabei wurde sich auf die beschriebenen und notwendigen Felder beschränkt.

\begin{listing}
    \inputminted{yaml}{../CITATION.cff}
    \caption{Beispiel einer \mintinline{text}{CITATION.cff}-Datei}
    \label{lst:cff_example}
    \small
    In dem Listing ist die \gls{cff}-Datei dieser Arbeit dargestellt. Dabei wurden primär Felder angegeben, welche in der Masterarbeit verwendet werden.
\end{listing}

\subsection{\hologo{BibTeX}}
\label{subsec:bibtex_format}
\hologo{BibTeX} ist eine Software, welche zur Erstellung von Literaturangaben und -verzeichnissen in \LaTeX{}-Dokumenten verwendet wird.
Außerdem existiert mit \hologo{BibTeX} ein gleichnamiges Format, welches in der \mintinline{text}{CITATION.bib}-Datei gespeichert wird und auf keinem anderen Format basiert.
\hologo{BibTeX} ist ein weit verbreiteter Standard und wird von vielen Autoren in der Wissenschaft verwendet.
Auf GitHub enthalten 2.144 Repositorys eine \mintinline{text}{CITATION.bib}-Datei (Stand 07.11.2024).
Das Format beschreibt die Zitation von Literatur und kann von Menschen und Maschinen gelesen werden.
Es beschränkt sich dabei nicht auf eine spezielle Art von Literatur, sondern kann für viele unterschiedliche Arten von Literatur verwendet werden.
Beispielsweise können Bücher und Masterarbeiten in \hologo{BibTeX} zitiert werden \autocite{patashnik_bibtexing_1988}.
Ein offizieller Literaturtyp für Software existiert nicht.
In der Datei können mehrere Einträge vorhanden sein, wobei jeder Eintrag eine Literaturangabe darstellt.

\hologo{BibTeX}-Dateien können von Menschen manuell erstellt und in das Repository eingebunden werden.
Außerdem existieren viele Literaturverwaltungsprogramme wie Zotero, welche \hologo{BibTeX}-Dateien erstellen und verarbeiten können \autocite{zotero_zotero_2024}.
Ebenfalls ist die Integration in andere Plattformen möglich, wie z.~B. in GitHub.
Hier wird die \mintinline{text}{CITATION.bib}-Datei auf der Repository-Startseite verlinkt, sie lässt sich jedoch im Gegensatz zu dem \gls{cff} nicht direkt kopieren oder in andere Formate umwandeln \autocite{github_about_2024-1}.

In dem \hologo{BibTeX}-Format existieren verschiedene Felder, welche für die Zitation von Literatur relevant sind.
Welche Felder zwingend erforderlich sind, hängt vom jeweiligen Literaturtyp ab, während andere optional bleiben.
Zudem sind die verfügbaren Felder ebenfalls vom Literaturtyp abhängig \autocite{patashnik_bibtexing_1988}.
In dieser Arbeit wird auf einige dieser Felder eingegangen, welche für die spätere Auswertung relevant sind.
Das wichtigste Feld ist das \emph{author}-Feld, welches die Autoren der Literatur enthält und zwingend erforderlich ist.
Die Vor- und Nachnamen der Autoren werden mit einem Komma separiert und mehrere Autoren werden über ein \glqq and\grqq{} getrennt.
Weitere Felder, welche für die Masterarbeit verwendet werden, sind \emph{year} und \emph{month}, welche das Jahr und den Monat der Veröffentlichung angeben.
Ein Beispiel einer \mintinline{text}{CITATION.bib}-Datei ist in \autoref{lst:bibtex_example} dargestellt.
Es ist zu erkennen, dass in dem \hologo{BibTeX}-Eintrag Informationen fehlen, welche in dem \gls{cff}-Eintrag vorhanden waren.
Dies liegt daran, dass in dem \hologo{BibTeX}-Eintrag nur eine Referenz auf die Masterarbeit möglich ist und nicht auf die entwickelte Software.

\begin{listing}
  \inputminted{text}{../CITATION.bib}
  \caption{Beispiel einer \mintinline{text}{CITATION.bib}-Datei}
  \label{lst:bibtex_example}
  \small
  In dem Listing ist die \hologo{BibTeX}-Datei dieser Arbeit dargestellt. Dabei wurden primär Felder angegeben, welche in der Masterarbeit verwendet werden.
\end{listing}

\section{Named Entity Recognition}
\label{sec:named-entity-recognition}

\section{Author name disambiguation}
\label{sec:author-name-disambiguation}
% TODO auf Entity Resolution oder Author name disambiguation eingehen?

\section{Fuzzy Suche}
\label{sec:fuzzy}

