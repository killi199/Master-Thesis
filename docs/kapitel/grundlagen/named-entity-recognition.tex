\section{Named Entity Recognition}
\label{sec:named-entity-recognition}
% TODO ggf. noch weiter ausführen?
Die \gls{ner} beschreibt den Prozess der automatischen Erkennung und Klasseneinteilung von Substantiven, sogenannten Entitäten im Text \autocite{mohit_named_2014}.
Typische Entitäten sind Personen, Orte und Organisationen.
Im folgenden Beispiel sind die Entitäten markiert, dabei ist nach erfolgreicher \gls{ner} die Klasse der Entität bekannt, welche hinter der Entität in Klammern gesetzt wurde.

\begin{addmargin}[25pt]{0pt}
    \emph{\textbf{PyTorch}} (Organisation) \emph{is currently maintained by \textbf{Soumith Chintala}} (Person)\emph{, \textbf{Gregory Chanan}} (Person)\emph{, \textbf{Dmytro Dzhulgakov}} (Person)\emph{, \textbf{Edward Yang}} (Person)\emph{, and \textbf{Nikita Shulga}} (Person) \emph{with major contributions coming from \textbf{hundreds}} (Ziffer) \emph{of talented individuals in various forms and means.}
\end{addmargin}

\gls{ner} wird in vielen Bereichen eingesetzt, wie z. B. Informationsextraktion, Frage-Antwort-Systeme und der maschinellen Übersetzung, um eine Wort-für-Wort-Übersetzung zu vermeiden.
Die \gls{ner} wurde in vielen Sprachen untersucht, darunter auch Arabisch und Hebräisch \autocite{mohit_named_2014}.
In dem Fall der Masterarbeit kann das System verwendet werden, um die Beschreibungen der Pakete zu verarbeiten und die Namen der Entwickler zu extrahieren.

In der \gls{ner} gibt es verschiedene Herausforderungen, welche schwierig zu lösen sind.
Diese sind zum einen das Erkennen des Entitäten-Anfangs und Endes und zum anderen die Erkennung der korrekten Entitätenklasse, welche standardmäßig versucht werden, parallel zu lösen.
Ebenfalls gibt es ähnlich wie in vielen anderen Bereichen der natürlichen Sprachverarbeitung das Problem der Polysemie, also dass ein Wort mehrere Bedeutungen haben kann \autocite{mohit_named_2014}.
Ein Beispiel hierfür ist das Wort \glqq Bank\grqq{}, welches sowohl eine Sitzgelegenheit als auch ein Finanzinstitut sein kann.
Dieses Problem ist besonders schwerwiegend in der \gls{ned}, auf welche im nächsten Abschnitt eingegangen wird.

Für die \gls{ner} gibt es verschiedene Programme und trainierte Modelle, welche verwendet werden können.
Ein Programm im Python-Umfeld für die \gls{ner} ist \emph{spaCy}, welches eine Open-Source-Bibliothek für die Verarbeitung natürlicher Sprache ist.
Die Bibliothek ist nicht beschränkt auf die \gls{ner}, sondern bietet auch Funktionalitäten wie Tagging, Parsing und Text Klassifikation.
Die Bibliothek hat zum Anspruch, fertige Modelle für den industriellen Einsatz bereitzustellen, welche sowohl schnell sind, als auch eine hohe Genauigkeit haben \autocite{honnibal_spacy_2024}.
Für die unterschiedliche Genauigkeit sowie für unterschiedliche Sprachen existieren verschiedene Modelle, welche verwendet werden können.
