\section{Versionsverwaltung}
\label{sec:versionsverwaltung}
Die Versionsverwaltung ist ein System, um zumeist Quellcode und dessen Änderungen zu verwalten.
Der Code und getätigte Änderungen werden in einem Repository gespeichert.
Dadurch ist die Versionsverwaltung eine Art Logbuch, in dem alle Änderungen festgehalten werden.
Dabei wird zusätzlich zu der Änderung der Autor und der Zeitpunkt der Änderung festgehalten \autocite{ponuthorai_version_2022}.
Dies ermöglicht es empirisch die Menge an Arbeit der einzelnen Autoren zu ermitteln.

Zusätzlich zum Code können in einem Repository andere Dateien, wie beispielsweise eine README, eine Lizenz, und Zitationsinformationen, beispielsweise in Form einer \gls{cff}-Datei, gespeichert werden.
Eine README-Datei ist eine Datei, welche in der Regel Informationen über das Projekt enthält, beispielsweise wie es installiert und verwendet wird.
Sie wird standardmäßig im Stammverzeichnis des Repositorys gespeichert und wird an dieser Stelle auch von Diensten wie GitHub dargestellt.

Es gibt zwei verschiedene Arten von Versionsverwaltungssystemen.
Zum einen gibt es die zentralen Systeme, bei denen alle Änderungen zentral verwaltet werden, beispielsweise SVN.
Zum anderen gibt es die verteilten Systeme, bei denen jeder Entwickler eine Kopie des gesamten Repositorys und dessen Vergangenheit hat \autocite{ponuthorai_version_2022}.
Ein solches System ist Git, welches sich mit einem Marktanteil von ungefähr 75 \% gegenüber anderen Systemen durchgesetzt hat \autocite{lindner_version_2024}.
Im Folgenden wird auf Git eingegangen und Begriffe erklärt, mit denen es möglich ist, die geleistete Arbeit von einzelnen Autoren innerhalb eines Repositorys zu untersuchen.
Außerdem wird auf grundlegende Funktionen von Git eingegangen, da diese für die Arbeit relevant sind.

Bei der Benutzung von Git kann ein Server verwendet werden.
Dieser ermöglicht eine einfache kollaborative Entwicklung von Code, da er ständig erreichbar ist und zentral verwaltet wird \autocite{ponuthorai_version_2022}.
Standardmäßig wird auf dem Git-Server die neueste Version des Repositorys gespeichert.
Ein Anbieter, welcher Git-Server bereitstellt ist GitHub, auf welchen im späteren Verlauf weiter eingegangen wird.
Für die Interaktion mit Git existieren verschiedene Programme, welche mit dem auf dem Computer befindlichen Repository interagieren \autocite{ponuthorai_version_2022}.
Die Programme können ebenfalls auf entfernte Git-Repositorys zugreifen und diese klonen.
Anschließend arbeiten die Programme auf der lokalen Kopie und können die Änderungen, wenn nötig, auf das entfernte Repository übertragen.

In Repositorys werden verschiedene Arten von Statistiken gespeichert.
Anders als in anderen Systemen wird keine Serie von Änderungen gespeichert, sondern ein \emph{Snapshot} der gesamten Datei zu einem bestimmten Zeitpunkt erstellt \autocite{ponuthorai_version_2022}.
Dies wird ein Commit genannt.
Mit einem Commit werden verschiedene Metainformationen gespeichert, beispielsweise eine Commit-Nachricht, der Autor und der Zeitpunkt der Änderungen.

Die Änderung wird mit zwei Zeitpunkten angegeben.
Zum einen wird der Zeitpunkt der Änderung des Autors angegeben, dies wird in Git \emph{author date} genannt.
Zum anderen wird der Zeitpunkt des Einfügens des Commits in das Repository gespeichert, dies wird in Git \emph{commit date} genannt.
Der Commit kann von einer anderen Person, z.~B. durch einen Projektverantwortlichen, mittels eines Pull Requests in das Repository übernommen worden sein.
Durch dieses Verhältnis ist der \emph{commit date} Zeitpunkt immer später oder gleich dem \emph{author date} Zeitpunkt.
Außerdem ist gewährleistet, dass beide Verantwortlichen Anerkennung für die geleistete Arbeit erhalten \autocite{chacon_pro_2024}.

Die Commit-Nachricht, sowie der Autor mit E-Mail-Adresse und Namen können in den Einstellungen von Git frei gewählt werden, müssen jedoch vorhanden sein, um einen Commit erstellen zu können.
Mehrere Commits bilden die Commit-Historie bzw. die Vergangenheit eines Repositorys.
Weitere Eigenschaften, welche sich aus dem Repository exportieren lassen, sind die Anzahl der eingefügten und gelöschten Zeilen.
Zudem lässt sich die Anzahl der geänderten Dateien ermitteln.
Diese Werte können für das gesamte Repository oder für einzelne Autoren ermittelt werden.

Ein Repository kann verschiedene Branches enthalten, muss jedoch mindestens einen Standardbranch enthalten.
Ein Branch ist eine separate Entwicklungslinie, welche unabhängig von anderen Branches ist.
Beim Erstellen von einem Branch wird der aktuelle Zustand des Branches, auf welchem der neue Branch erstellt wird, kopiert \autocite{ponuthorai_version_2022}.
Dadurch können Änderungen in dem neuen Branch durchgeführt werden, ohne dass diese Änderungen den ursprünglichen Branch beeinflussen.
Diese Änderungen werden mittels Commits festgehalten.
Unterschiedliche Branches können anschließend zusammengeführt werden, um die Änderungen aus einem Branch in einen anderen Branch zu übernehmen.
Ein Beispiel für einen solchen Workflow ist in \autoref{fig:git_workflow} dargestellt.

\begin{figure}
    {   
        \begin{center}
            \begin{tikzpicture}
                [
                    commit/.style={circle, draw, minimum size=1cm, inner sep=0pt},
                    edge/.style={->, thick},
                    main/.style={draw=colorbrewer@blue, fill=colorbrewer@lightblue},
                    feature/.style={draw=colorbrewer@green, fill=colorbrewer@lightgreen},
                ]
        
                % Main branch
                \node[commit, main] (c1) at (0, 4) {C1};
                \node[commit, main] (c2) at (2, 4) {C2};
                \node[commit, main] (c3) at (4, 4) {C3};

                % Feature branch
                \node[commit, feature] (f1) at (2, 2) {F1};
                \node[commit, feature] (f2) at (4, 2) {F2};

                % Connections
                \draw[edge, colorbrewer@blue] (c1) -- (c2);
                \draw[edge, colorbrewer@blue] (c2) -- (c3);

                \draw[edge, colorbrewer@green] (c1) -- (f1);
                \draw[edge, colorbrewer@green] (f1) -- (f2);

                % Merge back
                \draw[edge, dashed, gray] (f2) -- (c3);

                % Legend
                \draw[colorbrewer@blue, thick] (6, 4) -- ++(1, 0) node[right] {Main Branch};
                \draw[colorbrewer@green, thick] (6, 3.5) -- ++(1, 0) node[right] {Feature Branch};
                \draw[dashed, gray, thick] (6, 3) -- ++(1, 0) node[right] {Merge};
            
            \end{tikzpicture}
        \end{center}
        \caption{Git-Workflow}
        \label{fig:git_workflow}
        \small
        In dieser Abbildung stellen Kreise Commits dar. Die unterschiedlichen Farben repräsentieren Commits einzelner Branches. Die gestrichelte Linie stellt eine Zusammenführung der Branches dar.
    }
\end{figure}

Die Statistiken der Repositorys können auf verschiedene Arten aufgearbeitet werden.
Zum einen können einige direkt mittels Git-Befehlen ausgelesen werden \autocite{chacon_git_2024}.
Andere wiederum benötigen komplexere Abfragen, welche beispielsweise mittels Skripten oder speziellen Programmen ausgelesen werden können.
Ein Beispiel für ein solches Programm ist \emph{git-quick-stats} \autocite{mestan_git-quick-stats_2024}.
Außerdem bieten Onlinedienste zur Versionsverwaltung, wie GitHub, Statistiken über APIs an, welche jedoch im Umfang der Anfragen limitiert sind.
GitHub erlaubt beispielsweise für authentifizierte Nutzer 5.000 Anfragen pro Stunde \autocite{github_rate_2022}.
Bei der Benutzung der API von GitHub zum Abfragen der Autoren eines Repositorys werden automatisch alle E-Mail-Adressen der Autoren in Git mit den E-Mail-Adressen, welche die Autoren in GitHub angegeben haben, abgeglichen \autocite{github_rest-api-endpunkte_2022}.
Dadurch werden die Autoren eindeutig zugeordnet und deren Commits addiert.
Diese Werte werden ebenfalls in der Weboberfläche von GitHub angezeigt.

GitHub bietet neben der Bereitstellung eines Git-Servers zusätzliche Funktionen an, welche über die Standardfunktionen von Git hinausgehen.
Diese umfassen unter anderem die kollaborative Entwicklung von Code, Automatisierung mittels CI/CD, Sicherheitsaspekte, Projektmanagement, Team Administration und Client-Anwendun-gen zur Verwaltung von Repositorys \autocite{ponuthorai_version_2022}.
Aktuell benutzen GitHub über 100 Millionen Entwickler und mehr als 4 Millionen Organisationen.
Insgesamt verwaltet die Plattform über 420 Millionen Repositorys \autocite{github_about_2024}.
Um die zusätzlichen Funktionen von GitHub bereitzustellen, werden sogenannte Issues, Pull Requests, geschützte Branches, Actions, Diskussionen und Wikis eingesetzt.
GitHub-Issues sind eine Möglichkeit, um Probleme und Aufgaben zu verfolgen.
Pull Requests dienen dazu, Änderungen in einem Branch eines Repositorys anzufragen und über diese zu informieren.
In dem Pull Requests kann der Code überprüft und diskutiert werden.
