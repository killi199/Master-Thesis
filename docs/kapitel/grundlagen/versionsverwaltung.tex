\section{Versionsverwaltung}
\label{sec:versionsverwaltung}
% TODO noch mehr schreiben sollte auf 3 Seiten kommen habe jetzt 2,5 vllt reicht das
% TODO auf README eingehen, dass die Standardmäßig in GitHub oder auch anderen Versionsverwaltungen angegeben werden common sence vllt auch irgendwo eine Quelle
% TODO auf committed_datetime vs authored_datetime eingehen später wichtig
Die Versionsverwaltung ist ein System, um verschiedene Versionen von Software zu verwalten.
Es bietet Zugang zu Code und dessen Änderungen in der Vergangenheit.
Der Code und getätigte Änderungen werden in einem Repository gespeichert.
Dadurch ist die Versionsverwaltung eine Art Logbuch, in dem alle Änderungen festgehalten werden.
Dabei wird zusätzlich zu der Änderung der Autor und der Zeitpunkt der Änderung festgehalten \autocite{ponuthorai_version_2022}.
Dies ermöglicht es in der Masterarbeit empirisch die Menge an Arbeit der einzelnen Autoren zu ermitteln.

Es gibt zwei verschiedene Arten von Versionsverwaltungssystemen.
Zum einen gibt es die zentralen Systeme, bei denen alle Änderungen zentral verwaltet werden, beispielsweise SVN.
Zum anderen gibt es die verteilten Systeme, bei denen jeder Entwickler eine Kopie des gesamten Repository und dessen Vergangenheit hat \autocite{ponuthorai_version_2022}.
Ein solches System ist Git, welches sich mit einem Marktanteil von ungefähr 75 \% gegenüber anderen Systemen durchgesetzt hat \autocite{lindner_version_2024}.
Aus diesem Grund und weil Git-Repositorys in der Arbeit untersucht werden, wird auf Git eingegangen.
Dabei werden Begriffe erklärt, mit denen es möglich ist, die geleistete Arbeit von einzelnen Autoren innerhalb eines Repositorys zu untersuchen.
Außerdem wird auf grundlegende Funktionen von Git eingegangen, da diese für die Arbeit relevant sind.
Der Aufbau der einzelnen Git Komponenten ist in \autoref{fig:git} dargestellt.
Dabei ist zu erkennen, dass Git in einem Git-Server und Git-Anwendungen aufgeteilt wurde.

\begin{figure}
    {
        \centering
        \begin{tikzpicture}
            [
                block/.style = {rectangle, draw, fill=blue!20, text width=8cm, text centered, minimum height=1.5cm, rounded corners=0.2cm},
                client/.style = {rectangle, draw, fill=blue!20, text width=3.5cm, text centered, minimum height=1.5cm, rounded corners=0.2cm},
                arrow/.style = {thick,<->,>=stealth}
            ]
    
            \node (server) [block] {Git-Repository-Hosting-Plattform};
            \node (cli) [client, below=of server, xshift=-2.25cm] {Git-Befehlszeile};
            \node (gui) [client, below=of server, xshift=+2.25cm] {Git-GUI Anwendungen};
        
            \draw[arrow] (cli.north) -- (server.south -| cli.north);
            \draw[arrow] (gui.north) -- (server.south -| gui.north);
            \draw[arrow] (cli) -- (gui);

            \node[left=0.1cm of server] {Git-Server};
            \node[left=0.1cm of cli] {Git-Anwendungen};
        
        \end{tikzpicture}
        \caption{Übersicht über die Git-Komponenten}
        \label{fig:git}
        \small
        Die Git-Komponenten bestehen aus einem Git-Server, welcher das Repository hostet, und Git-Anwendungen, welche auf das Repository zugreifen (indirekt aus \cite{ponuthorai_version_2022}).
    }
\end{figure}

Bei der Benutzung von Git ist ein Server nicht zwingend erforderlich, jedoch steigert dies die Komplexität der Verwaltung und ist komplizierter in der Handhabung.
Der Git-Server ermöglicht die einfache kollaborative Entwicklung von Code, da dieser ständig erreichbar ist und zentral verwaltet wird \autocite{ponuthorai_version_2022}.
Standardmäßig wird auf dem Git-Server die neuste Version des Repositorys gespeichert.
Ein möglicher Git-Server ist GitHub, auf welchen im späteren Verlauf weiter eingegangen wird.
Git-Anwendungen sind Programme, welche mit dem lokalen Repository interagieren \autocite{ponuthorai_version_2022}.
Diese können auf entfernte Git-Repositorys zugreifen wie beispielsweise Repositorys, welche auf GitHub gehostet werden.
Anschließend arbeiten die Programme auf der lokalen Kopie und können die Änderungen, wenn nötig, auf das entfernte Repository übertragen.

In Repositorys werden verschiedene Arten von Statistiken gespeichert.
Git verwaltet Revisionen als \emph{Snapshot}.
Anders als in anderen Systemen wird keine Serie von Änderungen gespeichert, sondern ein \emph{Snapshot} der Änderungen zu einem bestimmten Zeitpunkt erstellt \autocite{ponuthorai_version_2022}.
Dies wird ein Commit genannt.
An einem Commit werden verschiedene Metainformationen gespeichert.
Unter anderem wird eine Commit-Nachricht, der Autor und der Zeitpunkt der Änderungen gespeichert.
Die Commit-Nachricht, sowie der Autor mit E-Mail-Adresse und Name können in den Einstellungen von Git frei gewählt werden, müssen jedoch vorhanden sein, um einen Commit erstellen zu können.
Mehrere Commits bilden die Commit-Historie bzw. die Vergangenheit eines Repositorys.
Weitere Eigenschaften, welche sich aus dem Repository exportieren lassen, sind die Anzahl der eingefügten und gelöschten Zeilen.
Außerdem lässt sich die Anzahl der geänderten Dateien ermitteln.
Diese Werte können für das gesamte Repository oder für einzelne Autoren ermittelt werden.

Ein Repository kann verschiedene Branches enthalten, muss jedoch mindestens einen enthalten.
In der Vergangenheit wurde der Standardbranch \emph{master} genannt.
Seit 2020 wird dieser jedoch in \emph{main} umbenannt, um rassistische Konnotationen zu vermeiden \autocite{github_githubrenaming_2024}.
Ein Branch ist eine separate Entwicklungslinie, welche unabhängig von anderen Branches ist.
Beim Erstellen von einem Branch wird der aktuelle Zustand des Branches, auf welchem der neue Branch erstellt wird, kopiert \autocite{ponuthorai_version_2022}.
Dadurch können Änderungen in dem neuen Branch durchgeführt werden, ohne dass diese Änderungen den ursprünglichen Branch beeinflussen.
Diese Änderungen werden mittels Commits festgehalten.
Unterschiedliche Branches können anschließend zusammengeführt werden, um die Änderungen in einem Branch in einen anderen Branch zu übernehmen.

Die Statistiken der Repositorys können auf verschiedene Arten aufgearbeitet werden.
Zum einen können einige direkt mittels Git-Befehlen ausgelesen werden \autocite{chacon_git_2024}.
Andere wiederum benötigen komplexere Abfragen, welche beispielsweise mittels Skripten oder speziellen Programmen ausgelesen werden können.
Ein Beispiel für ein Programm, welches Git-Statistiken aufarbeitet, ist \emph{git-quick-stats} \autocite{mestan_git-quick-stats_2024}.
Außerdem bieten Onlinedienste zur Versionsverwaltung, wie GitHub, Statistiken über APIs an, welche jedoch im Umfang der Anfragen limitiert sind \autocite{github_rate_2022}.
Bei der Benutzung der API von GitHub zum Abfragen der Autoren eines Repositorys werden automatisch alle E-Mail-Adressen der Autoren in Git mit den E-Mail-Adressen, welche die Autoren in GitHub angegeben haben abgeglichen \autocite{github_rest-api-endpunkte_2022}.
Dadurch werden die Autoren eindeutig zugeordnet und deren Commits addiert.
Diese Werte werden ebenfalls in der Weboberfläche von GitHub angezeigt.

GitHub ist eine Plattform, auf welcher Git-Repositorys gehostet werden können und dient somit als ein Git-Server.
GitHub bietet zusätzliche Funktionen an, welche über die Standardfunktionen von Git hinausgehen.
Diese umfassen unter anderem die kollaborative Entwicklung von Code, Automatisation mittels CI/CD, Sicherheitsaspekte, Projekt Management, Team Administration und Client-Anwendungen zur Verwaltung von Repositorys \autocite{ponuthorai_version_2022}.
Aktuell benutzen GitHub über 100 Millionen Entwickler und mehr als 4 Millionen Organisationen.
Insgesamt verwaltet die Plattform über 420 Millionen Repositorys.
Außerdem ist GitHub in 90 \% der Fortune 100 Unternehmen im Einsatz \autocite{github_about_2024}.
Um die zusätzlichen Funktionen von GitHub bereitzustellen werden sogenannte Issues, Pull Requests, geschützte Branches, Actions, Diskussionen und Wikis eingesetzt.
GitHub Issues sind eine Möglichkeit, um Probleme und Aufgaben zu verfolgen.
Pull Requests dienen dazu Änderungen in einem Branch eines Repositorys anzufragen und über diese zu informieren.
In dem Pull Requests kann der Code überprüft und diskutiert werden.
