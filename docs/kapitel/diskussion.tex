\chapter{Diskussion}
\label{chap:diskussion}
\section{Limitierungen}
\label{sec:limitierungen}
% TODO Doppelte Leute, da unterschiedliche Namen beim commiten angegeben -> Das Problem besteht beim benutzen der GitHub API nicht. -> Teilweise Gelöst mittels group auf E-Mail
% TODO Einen Abschnitt mit Limitierungen schreiben und da das Git Statistik Problem unter anderem erwähnen. Im Code kann ich das nicht lösen. Beschreiben wozu es dadurch kommt.
% TODO Unterschiedliche Programme unterschiedliche Commit Anzahl 
% Nochmal checken warum unterschiedliche Programme unterschiedliche Anzahlen an Commits ausgeben dafür eine Begründung suchen und das beste System auswählen -> wahrscheinlich ist es GitHub was aber blöd ist.
% GitHub hat andere Anzahl, da sie automatisch über die E-Mails machten, welche in GitHub registriert sind. Ich gruppiere nur die gleichen E-Mails.
% matplotlib Antony Lee:
%   git-quick-stats: 3864 (merges nicht inbegriffen https://github.com/arzzen/git-quick-stats?tab=readme-ov-file#git-merge-view-strategy):
%   GitHub: 4476 (merges nicht inbegriffen https://docs.github.com/en/repositories/viewing-activity-and-data-for-your-repository/viewing-a-projects-contributors#about-contributors gefühlt aber doch inbegriffen....) (E-Mails werden gemerged die GitHub bekannt sind)
%   git shortlog -s -n --no-merges: 3844 (merges nicht inbegriffen) 4370 (merges inbegriffen)
%   git fame: 4370 (merges inbegriffen)
% TODO Personen, welche als Autor genannt werden aber keinen Code geschrieben haben werden natürlich als nicht gematcht erkannt und sind somit ein fehler. Aber wie in Grundlagen beschrieben sind diese Personen ebenfalls wichtig und sollten zitiert werden.
% TODO Einige personen geben als Git namen nur einen buchstaben an, diese werden dann natürlich über in mit dem besten gematcht der diesen buchstaben im Namen hat, was natürlich nicht korrekt ist.
% TODO Nur auf Commits prüfen ist nicht ausreichend As we have already mentioned, contributorship, defined as the changing of code, does not cover the complete spectrum of contributions [6, 12]. aus (Which contributions count? Analysis of attribution in open source)
% TODO Sagen, warum nicht auf die .all-contributorsrc-Datei eingegangen wurde -> README wird analysiert dadurch Aufwand gespart könnte jedoch noch verbessert werden in dem Auch die Datei zusätzlich noch analysiert wird, da dort noch mehr informationen wie der Beitrag gelistet werden.
% TODO diskutieren, warum es nicht so notwendig ist Namensvetter zu unterscheiden
% TODO irgendwo auf pandas eingehen hier ist es so, dass die CFF auf eine Internet seite verlinkt, auf der alle contributor gelistet werden. Das parse ich natürlich nicht.
% TODO erklären warum die API verwendet wird und nicht nur die TOML beispielsweise ausgelesen wird -> habe ein pypi paket und brauche die GitHub URL (vllt eher diskussion oder so? Beschreibe hier ja nur was ich mache und nicht warum)
% TODO Diskuttieren warum ich bei der Kompletten CFF Liste auf so viele Daten verzichte
