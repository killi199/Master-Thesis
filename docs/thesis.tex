\documentclass{fiwthesis}
%\documentclass[en]{fiwthesis}   % Default language is german but can be switched to english

% ========
%  Pakete
% ========

\usepackage{textgreek}           % griechische Buchstaben außerhalb des Math-Mode
\usepackage{amsmath}             % zentrierte Formeln
\usepackage{amssymb}             % erweiterter Formelsatz mathem. Symbole

\usepackage{boldline}            % breitere Linien in Tabellen
\usepackage{booktabs}            % typographisch richtige Tabellen setzen
\usepackage{tabularx}            % Erweiterte Tabellendarstellung
\usepackage{multirow}            % Spalte über mehrere Zeilen oder Spalten ausdehnen
\usepackage{xltabular}           % Zeilenumbrüche in tabularx erlauben

\usepackage{graphicx}            % ermöglicht das Einbinden von Grafiken
\usepackage{subcaption}          % mehrere Bilder in einem Bild
\usepackage{pgfplots}            % Grafiken erzeugen
\usepackage{smartdiagram}        % schnelle und einfache Grafiken
\usetikzlibrary{positioning}     % bessere Ortsbezeichnung
\usetikzlibrary{shapes}          % typische Formen wie Rechtecke, Ellipsen usw. einfach zeichnen
\usetikzlibrary{intersections}   % Schnittpunkt von Geraden adressieren
\usetikzlibrary{angles, quotes}  % einfacheres Zeichnen von Winkeln
\usetikzlibrary{                 % Symbole für Schaltpläne
  circuits.logic.US,
  circuits.logic.IEC,
  circuits.logic.CDH,
  circuits.ee.IEC
}

\usepackage{lipsum}

% ===========
%  Metadaten
% ===========

\thesis{Master-Thesis}
\title{Identifikation und Vergleich von Autorenangaben zu Software zwischen verschiedenen Datenquellen}
\author{Kevin Jahrens}
\date{\today}
\matrnr{480592}
\bdate{05.08.1999}
\bcity{Bad Oldesloe}
\supervisor{Prof.~Dr.~-Ing.~Frank Krüger} 
\secsupervisor{Stephan Druskat}
\keywords{Data, Science, Software, Author, Comparison, Identification}

% Metadaten in die PDF-Datei schreiben
\makepdfmetadata

% ==========
%  Präambel
% ==========

% PGF Kompatibilitätseinstellung
\pgfplotsset{width=0.95\textwidth,compat=newest}

% % Bibliographie einbinden
\bibliography{quellen}

% Glossar einbinden
\newglossaryentry{nosql}{%
  name = {NoSQL},
  description = {Kurzform für ,,Not Only SQL``; Überbegriff für Datenbanken, die das Konzept relationaler Datenbanken erweitern}
}

\newdualentry{dos}% label
{DoS}% short form
{Denial of Service}% long form
{Ein Denial of Service (im Deutschen: Dienstverweigerung) ist ein Angriffe auf Computer- oder Netzwerksysteme, wobei das Zielsystem durch Überlastung oder durch andere Mittel außer Betrieb gesetzt wird}% description

% Abkürzungen einbinden
%\gls{}         normal zu nutzen (erstes Mal: 'lange Form (kurze Form)'), danach nur 'kurze Form'
%\glspl{}       wie \gls{} nur als Plural
%\acrfull{qrc}  gibt volle Form ('lange Form (kurze Form)') egal wo
%\acrlong{qrc}  gibt lange Form ('lange Form') egal wo
%
%\newacronym{tag}{short}{long}
\newacronym{cff}{CFF}{Citation File Format}
\newacronym{pep}{PEP}{Python Enhancement Proposal}
\newacronym{pypi}{PyPI}{Python Package Index}
\newacronym{cran}{CRAN}{Comprehensive R Archive Network}
\newacronym{ner}{NER}{Named entity recognition}
\newacronym{ned}{NED}{Named entity disambiguation}
\newacronym{oss}{OSS}{Open-Source-Software}
\newacronym{doi}{DOI}{Digitaler Objektbezeichner}


% Symbole einbinden
\newglossaryentry{symb:phi}{
  name=$\phi$,
  description={Ein beliebiger Winkel},
  sort=symbolphi, type=symbolslist
}

\newglossaryentry{symb:e}{
  name=$e$,
  description={Die Eulersche Zahl},
  sort=symbole, type=symbolslist
}


% Glossar- und Abkürzungsverzeichniserstellung
\makeglossaries{}

% Index erzeugen
\makeindex[
  intoc=true,
  title=Index,
  columns=2]{}
\indexsetup{headers={\indexname}{\indexname}}

% ===============
%  Eigene Makros
% ===============

\newcommand*{\code}[1]{\texttt{#1}}

% ===============
%  Beginn Thesis
% ===============

\begin{document}

% ============
%  "Vorspann"
% ============

% Titelseite
\maketitle

% Aufgabenstellung
\maketask{Identifikation und Vergleich von Autorenangaben zu Software zwischen verschiedenen Datenquellen
\vspace*{-.5cm}

\noindent\makebox[\linewidth]{\rule{\linewidth}{0.4pt}}
Identification and comparison of authors of software across different data sources

\vspace*{-.5cm}
\noindent\makebox[\linewidth]{\rule{\linewidth}{0.4pt}}

\textbf{Disposition}
\vspace*{.1cm}

Software spielt eine zentrale Rolle in der Wissenschaft und sollte daher in wissenschaftlichen Arbeiten zitiert werden.
Insbesondere für Autoren wissenschaftlicher Software ist die Zitation wesentlicher Bestandteil der wissenschaftlichen Anerkennung, sodass diese auch zunehmend in wissenschaftlichen Lebensläufen genannt werden und Beachtung finden.
Anders als bei wissenschaftlichen Publikationen ist bei wissenschaftlicher Software aktuell noch unklar, welcher Anteil an der Entwicklung zu einer Nennung als Autor führt.
Darüber hinaus existieren in verschiedenen Datenquellen widersprüchliche Angaben für Zitationsvorschläge bzgl. der Autoren einer Software.

Ziel dieser Masterarbeit ist es zu untersuchen inwieweit sich die Angaben von Autoren für Open Source Software unterscheiden. 
Dazu sollen öffentlich verfügbare Repositorien mit R und Python Paketen -- als Stellvertreter für wissenschaftliche Software -- hinsichtlich ihrer Autorenangaben untersucht werden.
Insbesondere sollen die angegebenen Metadaten in den Repositorien (z.B. citation.cff) mit den Metadaten in Paketdatenbanken (\url{https://pypi.org/} und \url{https://cran.r-project.org/}) und den Entwicklungsanteilen automatisch verglichen werden.

\begin{enumerate}
    \item Literaturrecherche Autorenrolle in Open Source Software und zur Disambiguierung von Autorennamen 
    \item Datensammlung: Identifikation und Download verfügbarer Metadaten zu \enquote{wichtigen} Softwarepaketen
    \item Automatische Auflösung und Abgleich der Autorennennungen aller Datenquellen
    \item Analyse von Unterschieden in der Nennung von Autoren
    \item Dokumentation der Ergebnisse in einer schriftlichen Master-Thesis
\end{enumerate}
}

% Abstract
\makeabstract{
  Maximal eine halbe Seite.
}{
  English Version.
}

% Inhaltsverzeichnis (Schalter `compact' sorgt für einfachen Zeilenabstand)
\maketoc[compact]

% ==========
%  Textteil
% ==========

% Einleitung
\chapter{Einleitung}
\label{chap:einleitung}
\section{Motivation}
\label{sec:motivation}
\section{Vorgehen}
\label{sec:vorgehen}
% TODO erklären was wird vorhaben (Autoren aus unterschiedlichen quellen extrahieren zuordnen und checken wie gut das funktioniert und wie gut Autoren von Software gepflegt werden)
\section{Gliederung}
\label{sec:gliederung}


% weitere Kapitel hier jeweils einzeln einbinden
\chapter{Zusätzliche Abbildungen}
\label{chap:zusaetzliche_abbildungen}

\begin{figure}[H]
    \begin{subfigure}{.5\textwidth}
        \centering
        \includesvg[width=.95\linewidth,inkscapelatex=false]{bilder/common_authors/1_pypi.svg}
        \caption{\gls{pypi} nach Commits}
        \label{fig:common_authors_pypi}
    \end{subfigure}%
    \begin{subfigure}{.5\textwidth}
        \centering
        \includesvg[width=.95\linewidth,inkscapelatex=false]{bilder/common_authors_by_files/1_pypi_by_files.svg}
        \caption{\gls{pypi} nach geänderten Zeilen}
        \label{fig:common_authors_by_files_pypi}
    \end{subfigure}
    \begin{subfigure}{.5\textwidth}
        \centering
        \includesvg[width=.95\linewidth,inkscapelatex=false]{bilder/common_authors/1_cran.svg}
        \caption{\gls{cran} nach Commits}
        \label{fig:common_authors_cran}
    \end{subfigure}%
    \begin{subfigure}{.5\textwidth}
        \centering
        \includesvg[width=.95\linewidth,inkscapelatex=false]{bilder/common_authors_by_files/1_cran_by_files.svg}
        \caption{\gls{cran} nach geänderten Zeilen}
        \label{fig:common_authors_by_files_cran}
    \end{subfigure}
    \begin{subfigure}{.5\textwidth}
        \centering
        \includesvg[width=.95\linewidth,inkscapelatex=false]{bilder/common_authors/1_cff.svg}
        \caption{\gls{cff} nach Commits}
        \label{fig:common_authors_cff}
    \end{subfigure}%
    \begin{subfigure}{.5\textwidth}
        \centering
        \includesvg[width=.95\linewidth,inkscapelatex=false]{bilder/common_authors_by_files/1_cff_by_files.svg}
        \caption{\gls{cff} nach geänderten Zeilen}
        \label{fig:common_authors_by_files_cff}
    \end{subfigure}
    \caption{Anteil der Top Git Autoren in der Zitation}
    \label{fig:common_authors_anhang}
\end{figure}


% Schluss
\chapter{Fazit und Ausblick}
\label{cap:fazit_ausblick}
\section{Fazit}
\label{sec:fazit}
\section{Ausblick}
\label{sec:ausblick}


% =========
%  Anlagen
% =========

\begin{appendices}

  \chapter{Zusätzliche Abbildungen}
\label{chap:zusaetzliche_abbildungen}

\begin{figure}[H]
    \begin{subfigure}{.5\textwidth}
        \centering
        \includesvg[width=.95\linewidth,inkscapelatex=false]{bilder/common_authors/1_pypi.svg}
        \caption{\gls{pypi} nach Commits}
        \label{fig:common_authors_pypi}
    \end{subfigure}%
    \begin{subfigure}{.5\textwidth}
        \centering
        \includesvg[width=.95\linewidth,inkscapelatex=false]{bilder/common_authors_by_files/1_pypi_by_files.svg}
        \caption{\gls{pypi} nach geänderten Zeilen}
        \label{fig:common_authors_by_files_pypi}
    \end{subfigure}
    \begin{subfigure}{.5\textwidth}
        \centering
        \includesvg[width=.95\linewidth,inkscapelatex=false]{bilder/common_authors/1_cran.svg}
        \caption{\gls{cran} nach Commits}
        \label{fig:common_authors_cran}
    \end{subfigure}%
    \begin{subfigure}{.5\textwidth}
        \centering
        \includesvg[width=.95\linewidth,inkscapelatex=false]{bilder/common_authors_by_files/1_cran_by_files.svg}
        \caption{\gls{cran} nach geänderten Zeilen}
        \label{fig:common_authors_by_files_cran}
    \end{subfigure}
    \begin{subfigure}{.5\textwidth}
        \centering
        \includesvg[width=.95\linewidth,inkscapelatex=false]{bilder/common_authors/1_cff.svg}
        \caption{\gls{cff} nach Commits}
        \label{fig:common_authors_cff}
    \end{subfigure}%
    \begin{subfigure}{.5\textwidth}
        \centering
        \includesvg[width=.95\linewidth,inkscapelatex=false]{bilder/common_authors_by_files/1_cff_by_files.svg}
        \caption{\gls{cff} nach geänderten Zeilen}
        \label{fig:common_authors_by_files_cff}
    \end{subfigure}
    \caption{Anteil der Top Git Autoren in der Zitation}
    \label{fig:common_authors_anhang}
\end{figure}


\end{appendices}

% ===============
%  Verzeichnisse
% ===============

% Verzeichnisse mit einzeiligem Zeilenabstand
\singlespacing

% Literaturverzeichnis
\listofreferences

% Abbildungsverzeichnis einfügen
\listoffigures

% Tabellenverzeichnis einfügen
\listoftables

% Algorithmenverzeichnis einfügen
\listofalgorithms

% % Quelltextverzeichnis einfügen
\listoflistings

% Abkürzungsverzeichnis
\listofacronyms

% Symbolverzeichnis
\listofsymbols

% falls ein anderer Glossar-Stil genutzt wird und die zweite Spalte zu schmal ist:
% \setlength{\glsdescwidth}{0.8\linewidth}

% Glossar einfügen
\printglossary

% Index einfügen
\printindex

% wieder auf 1½-fachen Zeilenabstand umschalten
\normalspacing

% =========================================
%  Selbstständigkeitserklärung, CD, Thesen
% =========================================

% Inhalt der CD; nur für gedruckte Version wichtig
\makecd{\small

\begin{minipage}[t]{0.97\linewidth}
  \dirtree{%
    .1 /\DTcomment{Wurzelverzeichnis}.
    .2 OrdnerA\DTcomment{Ein Ordner auf dem Datenträger}.
    .3 OrdnerB\DTcomment{Ein Unterordner auf dem Datenträger}.
    .3 datei.xyz\DTcomment{Eine Datei}.
    .2 thesis.pdf\DTcomment{PDF-Datei dieser Bachelor-Thesis}.
  }
\end{minipage}

\vspace{2em}

Im Unterverzeichnis \code{tools} des Projekts findet sich das Perl-Skript \code{dirtree.pl}, mit welchem Inhalte für das dirtree-Environment (siehe oberhalb) semiautomatisch erstellt werden können.

Die Nutzung aus der Kommandozeile ist wie folgt:\\
\code{perl dirtree.pl /path/to/top/of/dirtree}

Quelle des Skripts:\\
\url{https://texblog.org/2012/08/07/semi-automatic-directory-tree-in-latex/}
\normalsize}

% Selbstständigkeitserklärung
\makedoiw

% Thesen (allerletzte Seite); Thesen bitte immer durch Semikolons trennen
\maketheses{%
  These 1;
  These 2;
  These 3;
  These 4;
  These 5;
  These 6;
  \ldots
}

\end{document}

% =============
%  Ende Thesis
% =============
