\documentclass[a4paper,DIV=12]{scrartcl}

\usepackage[utf8]{inputenc}
\usepackage{paralist}
\usepackage[table]{xcolor}
\usepackage{helvet}
\usepackage{pgfgantt}
\usepackage{textcomp}
\usepackage{paralist}
\usepackage{booktabs}
\usepackage{tabularx}
\usepackage{eurosym}
\usepackage[nolist]{acronym}
\usepackage{enumitem}
\usepackage{xspace}
\usepackage{multirow}
\usepackage{pifont}
\usepackage{subcaption}
\usepackage{breakurl}
\usepackage{amssymb}
\usepackage[originalparameters]{ragged2e}
\usepackage{csquotes}
\usepackage{tabularx}
\usepackage{booktabs}
\newcommand*{\corpus}{\textsc SoMeSci}
\usepackage{todonotes}
\usepackage[ngerman]{babel}

% \usepackage[numbers]{natbib}
\usepackage[headsep=15mm]{geometry}
\geometry{a4paper,left=2cm,right=2cm, top=3.0cm, bottom=2.5cm}

\usepackage[backend=biber, defernums=true, sorting=none,
             style=numeric, giveninits=true, natbib=true,
             url=true, doi=true, isbn=false, bibencoding=utf8,backref=false,maxbibnames=20,maxcitenames=3]{biblatex}
\usepackage{hyperref}

\addbibresource{Software_Authors.bib}

\renewcommand{\familydefault}{\sfdefault}



% \addbibresource{Beitritt.bib}
\setlength{\parindent}{0pt}
% \newcommand{\itodo}[1]{{\leavevmode\color{orange}[#1]}}


\begin{document}
\thispagestyle{empty}

\vspace*{.2cm}

\begin{minipage}{.7\textwidth}
Hochschule Wismar\\
University of Applied Sciences\\
Technology, Business and Design\\
Fakultät für Ingenieurwissenschaften\\
Bereich Elektrotechnik und Informatik\\
\end{minipage}\hfill
\begin{minipage}{.3\textwidth}
    Wismar, \today{}
    
    \vfill{}
\end{minipage}


{\Large\centerline{\textbf{Master-Thesis}}}

\vspace*{1em}

für: Herr \textbf{Kevin Jahrens}

\vspace*{1em}

Identifikation und Vergleich von Autorenangaben zu Software zwischen verschiedenen Datenquellen

\noindent\makebox[\linewidth]{\rule{\linewidth}{0.4pt}}
Identifikation and comparison of authors of software across different data sources

\noindent\makebox[\linewidth]{\rule{\linewidth}{0.4pt}}

\textbf{Disposition}
\vspace*{.1cm}

Software spielt eine zentrale Rolle in der Wissenschaft und sollte daher in wissenschaftlichen Arbeiten zitiert werden.
Insbesondere für Autoren wissenschaftlicher Software ist die Zitation wesentlicher Bestandteil der wissenschaftlichen Anerkennung, sodass diese auch zunehmend in wissenschaftlichen Lebensläufen genannt werden und Beachtung finden.
Anders als bei wissenschaftlichen Publikationen ist bei wissenschaftlicher Software aktuell noch unklar, welcher Anteil an der Entwicklung zu einer Nennung als Autor führt.
Darüber hinaus existieren in verschiedenen Datenquellen widersprüchliche Angaben für Zitationsvorschläge bzgl. der Autoren einer Software.

Ziel dieser Masterarbeit ist es zu untersuchen inwieweit sich die Angaben von Autoren für Open Source Software unterscheiden. 
Dazu sollen öffentlich verfügbare Repositorien mit R und Python Paketen -- als Stellvertreter für wissenschaftliche Software -- hinsichtlich ihrer Autorenangaben untersucht werden.
Insbesondere sollen die angegebenen Metadaten in den Repositorien (z.B. citation.cff) mit den Metadaten in Paketdatenbanken (\url{https://pypi.org/} und \url{https://cran.r-project.org/}) und den Entwicklungsanteilen automatisch verglichen werden.

\begin{compactenum}
    \item Literaturrecherche Autorenrolle in Open Source Software und zur Disambiguierung von Autorennamen 
    \item Datensammlung: Identifikation und Download verfügbarer Metadaten zu \enquote{wichtigen} Softwarepaketen
    \item Automatische Auflösung und Abgleich der Autorennennungen aller Datenquellen
    \item Analyse von Unterschieden in der Nennung von Autoren
    \item Dokumentation der Ergebnisse in einer schriftlichen Master-Thesis
\end{compactenum}

\vfill{}

\noindent\rule{5cm}{.4pt}\hfill\rule{5cm}{.4pt}\par
\noindent Prof.\ Dr.\ rer.\ nat.\ Litschke \hfill Prof.\ Dr.-Ing. Krüger\par
\noindent {\footnotesize Chairman of the Examination Committee} \hfill {\footnotesize Supervisor}

\printbibliography





\end{document}