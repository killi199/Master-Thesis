\documentclass[%
    handout,
    aspectratio=1610,
    10pt,
    onlytextwidth, % requires Beamer v3.65 or newer
]{beamer}

\usepackage[utf8]{luainputenc}
\usepackage[T1]{fontenc}
\usepackage[ngerman]{babel}
\usepackage[figurename=Fig.]{caption}
\usepackage{graphicx}
\usepackage{amsmath}
\usepackage{lipsum}
\usepackage{scrextend}
\usepackage{ragged2e}
\usepackage{tabularx}
\usepackage{csquotes}
\usepackage[  
    style=numeric,
    sorting=nyt,
    sortcites,
    backend=biber   % use modern bibliography backend
]{biblatex}

% alternative: FiraSans
\usepackage[scaled]{helvet}

\pdfstringdefDisableCommands{%
    \def\\{}%
}

\AtBeginSection[]{
    \begin{frame}
    \vfill
    \centering
    \begin{beamercolorbox}[sep=8pt,center,rounded=true]{title}
        \usebeamerfont{part title}\insertsectionhead\par%
    \end{beamercolorbox}
    \vfill
    \end{frame}
}

\definecolor{links}{HTML}{3333B2}
\hypersetup{colorlinks,linkcolor=,urlcolor=links,citecolor=links}

\usetheme{Wismar}

% presentation title (short version in brackets)
\title[Verteidigung]{Identifikation und Vergleich von Autorenangaben zu Software zwischen verschiedenen Datenquellen}

% subtitle (optional)
% \subtitle{Dolor sit amet}

% date (and place)
\date{Wismar, 30. Januar 2025}

% author / presenter (short version in brackets)
\author[Kevin Jahrens]{Kevin Jahrens \\
E-Mail: \href{mailto:k.jahrens@stud.hs-wismar.de}{k.jahrens@stud.hs-wismar.de}}

\institute{Fakultät für Ingenieurwissenschaften, Hochschule Wismar}

% project or faculty homepage (this is a custom macro from the Wismar theme)
% \homepage{\url{https://theuselessweb.com}}

% \usecolortheme{FIW}
% \usecolortheme{FWW}
% \usecolortheme{FG}

% disable navigation symbols globally
% \beamertemplatenavigationsymbolsempty

\addbibresource{quellen.bib}

% TODO ich kann in der Präsentation nicht davon ausgehen, dass die Arbeit gelesen wurde! Somit muss ich auch den kompletten Kontext rüber bringen und die Ziele und was wir überhaupt machen und warum mindestens 30min höchstens 45 min Vortrag sollte so 20 Minuten gehen
% TODO Schlachtplan für die Verteidigung überlegen. Die ganzen Entscheidungen gewisse Dinge auszuschließen und nicht auszuwerten sollte ich gut Begründen können
% TODO nochmal Simons Anmerkungen lesen und hier berücksichtigen!

% Schlachtplan für die Verteidigung:
% Warum habe ich kaum Graphen für alle CFF?
% 1. Habe eh schon super viele Grafiken da blickt kaum noch wer durch
% 2. Erstellung wäre aufwand gewesen + ich hätte die nicht so abhandeln können wie die anderen Graphen
% 3. Ich habe die MA aufgebaut anhand der Listen wollte das Konzept nicht brechen
% 4. Das wichtigste: Ich hab die Graphen mit den Linien zwar nicht erzeugt aber: ich gehe ganz stark davon aus, dass sie ähnlich zu denen sind die ich schon habe immerhin sind die auch alle gleich. Einige andere Graphen wie DOI oder Type der Zitation hatte ich erstellt aber für unnötig gehalten und wieder entfernt. Besonders bei Type erkennt man nichts weil es so viele verschiedene gibt dadurch, dass da jeder eingeben kann was er will

\begin{document}

\newcolumntype{P}[1]{>{\centering\arraybackslash}p{#1}}
\newcolumntype{L}[1]{>{\raggedright\arraybackslash}p{#1}}
\newcolumntype{R}[1]{>{\raggedleft\arraybackslash}p{#1}}

% title page
\maketitle

\begin{frame}{Gliederung}
    \tableofcontents
\end{frame}

% TODO Alles durch LanguageTool laufen lassen

\section{Einleitung}

% TODO Spannung aufbauen unter anderem mit Miculas, dass er nicht genannt wurde
% TODO das so genau aus Forschungsseminar übernommen noch ändern?
\begin{frame}{\secname}{Motivation}
    \begin{itemize}
        \item Software spielt zentrale Rolle in der Wissenschaft
        \item Zitation wesentlicher Bestandteil in wissenschaftlicher Publikation
        \item Bei wissenschaftlicher Software ist dies in diesem Umfang aktuell nicht gegeben
        \item Softwareautoren werden nicht immer genannt und manchmal sogar ihrer Beiträge beraubt \autocite{miculas_how_2023}
    \end{itemize}
\end{frame}

% TODO Konkret sagen was ich vor habe
\begin{frame}{\secname}{Vorgehen}
    \begin{itemize}
        \item Autoren aus unterschiedlichen Quellen extrahieren
        \item Autoren untereinander abgleichen
        \item Ausschließliche Betrachtung von Autoren die Code in Git beigetragen haben
        \item Ergebnisse aufbereiten
        \item Beantwortung von Forschungsfragen
    \end{itemize}
\end{frame}

% TODO auf Forschungsfragen eingehen
\begin{frame}{\secname}{Forschungsfragen}
    \begin{itemize}
      \item[\textbf{F1}] Wie gut können Autoren untereinander abgeglichen werden?
      \item[\textbf{F2}] Was muss ein Softwareentwickler leisten, um als Autor genannt zu werden?
      \item[\textbf{F3}] Wie gut werden Autoren in den einzelnen Quellen gepflegt?
    \end{itemize}
\end{frame}

\section{Grundlagen}
\section{Methodik}
\section{Ergebnisse}
\section{Diskussion}

\begin{frame}{Literaturverzeichnis}
    \printbibliography
\end{frame}

% \finalpage{Fragen}

\end{document}
