\documentclass[%
    handout,
    aspectratio=1610,
    10pt,
    onlytextwidth, % requires Beamer v3.65 or newer
]{beamer}

\usepackage[utf8]{luainputenc}
\usepackage[T1]{fontenc}
\usepackage[ngerman]{babel}
\usepackage[figurename=Fig.]{caption}
\usepackage{graphicx}
\usepackage{amsmath}
\usepackage{lipsum}
\usepackage{scrextend}
\usepackage{ragged2e}
\usepackage{tabularx}
\usepackage{csquotes}
\usepackage{hologo}
\usepackage{minted}
\usepackage{fontawesome5}
\usepackage{svg}
\usepackage{booktabs}
\usepackage[  
    style=numeric,
    sorting=nyt,
    sortcites,
    backend=biber   % use modern bibliography backend
]{biblatex}

% alternative: FiraSans
\usepackage[scaled]{helvet}

\setminted{
  autogobble=true,
  bgcolor=white,
  breakautoindent=true,
  breaklines=true,
  escapeinside=§§,
  fontfamily=tt,
  fontsize=\footnotesize,
  frame=leftline,
  framerule=0pt,
  framesep=0.2em, % sufficient for up to 4 digits
  numbers=left,
  numbersep=0.2em,
  showspaces=false,
  showtabs=false,
  style=vs, % see: https://pygments.org/styles/
  tabsize=4,
  xleftmargin=1.5em,
}

\pdfstringdefDisableCommands{%
    \def\\{}%
}

\AtBeginSection[]{
    \begin{frame}
    \vfill
    \centering
    \begin{beamercolorbox}[sep=8pt,center,rounded=true]{title}
        \usebeamerfont{part title}\insertsectionhead\par%
    \end{beamercolorbox}
    \vfill
    \end{frame}
}

\definecolor{links}{HTML}{3333B2}
\hypersetup{colorlinks,linkcolor=,urlcolor=links,citecolor=links}

\usetheme{Wismar}

% presentation title (short version in brackets)
\title[Verteidigung]{Identifikation und Vergleich von Autorenangaben zu Software zwischen verschiedenen Datenquellen}

% subtitle (optional)
% \subtitle{Dolor sit amet}

% date (and place)
\date{Wismar, 30. Januar 2025}

% author / presenter (short version in brackets)
\author[Kevin Jahrens]{Kevin Jahrens \\
E-Mail: \href{mailto:k.jahrens@stud.hs-wismar.de}{k.jahrens@stud.hs-wismar.de}}

\institute{Fakultät für Ingenieurwissenschaften, Hochschule Wismar}

% project or faculty homepage (this is a custom macro from the Wismar theme)
% \homepage{\url{https://theuselessweb.com}}

% \usecolortheme{FIW}
% \usecolortheme{FWW}
% \usecolortheme{FG}

% disable navigation symbols globally
% \beamertemplatenavigationsymbolsempty

\addbibresource{quellen.bib}

% TODO Schlachtplan für die Verteidigung überlegen. Die ganzen Entscheidungen gewisse Dinge auszuschließen und nicht auszuwerten sollte ich gut Begründen können
% TODO nochmal Simons Anmerkungen lesen und hier berücksichtigen!

% Schlachtplan für die Verteidigung:
% Warum habe ich kaum Graphen für alle CFF?
% 1. Habe eh schon super viele Grafiken da blickt kaum noch wer durch
% 2. Erstellung wäre aufwand gewesen + ich hätte die nicht so abhandeln können wie die anderen Graphen
% 3. Ich habe die MA aufgebaut anhand der Listen wollte das Konzept nicht brechen
% 4. Das wichtigste: Ich hab die Graphen mit den Linien zwar nicht erzeugt aber: ich gehe ganz stark davon aus, dass sie ähnlich zu denen sind die ich schon habe immerhin sind die auch alle gleich. Einige andere Graphen wie DOI oder Type der Zitation hatte ich erstellt aber für unnötig gehalten und wieder entfernt. Besonders bei Type erkennt man nichts weil es so viele verschiedene gibt dadurch, dass da jeder eingeben kann was er will

\begin{document}

\newcolumntype{P}[1]{>{\centering\arraybackslash}p{#1}}
\newcolumntype{L}[1]{>{\raggedright\arraybackslash}p{#1}}
\newcolumntype{R}[1]{>{\raggedleft\arraybackslash}p{#1}}

% title page
\maketitle

\section{Einleitung}

\begin{frame}{Motivation}
    \begin{itemize}
        \item Software spielt zentrale Rolle in der Wissenschaft
        \item Zitation wesentlicher Bestandteil in wissenschaftlicher Publikation
        \item Bei wissenschaftlicher Software ist dies in diesem Umfang aktuell nicht gegeben
        \item Softwareautoren werden nicht immer genannt und manchmal sogar ihrer Beiträge beraubt \autocite{miculas_how_2023}
    \end{itemize}
\end{frame}

\begin{frame}{Vorgehen}
    \begin{itemize}
        \item Autoren aus unterschiedlichen Quellen extrahieren % Dafür werden zu beginn die Quellen erläutert
        \item Autoren untereinander abgleichen
        \item Ausschließliche Betrachtung von Autoren, die Code in Git beigetragen haben
        \item Ergebnisse aufbereiten
        \item Beantwortung von Forschungsfragen
    \end{itemize}
\end{frame}

\begin{frame}{Forschungsfragen}
    \begin{itemize}
      \item[\textcolor{links}{\textbf{F1}}] Wie gut können Autoren untereinander abgeglichen werden?
      \item[\textcolor{links}{\textbf{F2}}] Was muss ein Softwareentwickler leisten, um als Autor genannt zu werden?
      \item[\textcolor{links}{\textbf{F3}}] Wie gut werden Autoren in den einzelnen Quellen gepflegt?
    \end{itemize}
\end{frame}

\section{Grundlagen}

\begin{frame}{Prinzipien der Software-Zitation \autocite{smith_software_2016}}
    \begin{enumerate}
        \item \textcolor{links}{\textbf{Wichtigkeit}}: Software sollte ein seriöses und zitierbares Produkt wissenschaftlicher Arbeit sein
        \item \textcolor{links}{\textbf{Anerkennung und Zuschreibung}}: Softwarezitate sollten die wissenschaftliche Anerkennung und die normative, rechtliche Würdigung aller Mitwirkenden an der Software ermöglichen
        \item \textcolor{links}{\textbf{Eindeutige Identifikation}}: Ein Softwarezitat sollte eine Methode zur Identifikation enthalten, die maschinell verwertbar, weltweit eindeutig und interoperabel ist
        \item \textcolor{links}{\textbf{Persistenz}}: Eindeutige Identifikatoren und Metadaten, die die Software und ihre Verwendung beschreiben, sollten bestehen bleiben – auch über die Lebensdauer der Software hinaus
        \item \textcolor{links}{\textbf{Zugänglichkeit}}: Softwarezitate sollten den Zugang zur Software selbst und weiteren Materialien erleichtern, um sie sachkundig nutzen zu können
        \item \textcolor{links}{\textbf{Spezifizität}}: Softwarezitate sollten die Identifikation und den Zugang zu der spezifischen Version der verwendeten Software erleichtern. Die Identifizierung der Software sollte so spezifisch wie nötig sein
    \end{enumerate}
\end{frame}

\begin{frame}{Versionsverwaltung}
    \begin{itemize}
        \item Verwaltet Quellcode und dessen Änderungen in einem Repository
        \item Git ist eine weit verbreitete Versionsverwaltung mit einem Marktanteil von ungefähr 75 \% \autocite{lindner_version_2024}
        \item Speichert Zeitpunkt und Autor, sowie die Änderungen in einem Commit
        \item Name und E-Mail des Autors frei wählbar
        \item In Git werden weitere Daten gespeichert, welche ausgelesen werden können:
        \begin{itemize}
            \item Anzahl der eingefügten und gelöschten Zeilen
            \item Anzahl der geänderten Dateien
            \item Anzahl der Commits
        \end{itemize}
        \item Repositorys können auf einem Server verwaltet werden
        \item Weit verbreiteter Anbieter eines Git-Servers ist GitHub
    \end{itemize}
\end{frame}

\begin{frame}{Software-Verzeichnisse und Paketverwaltung}
    \begin{itemize}
        \item Eine Paketverwaltung verwaltet fertige Softwarepakete, bspw. kompilierten Code
        \item Softwarepakete können in einem Software-Verzeichnis abgelegt werden
        \item Softwarepakete enthalten Metadaten, bspw. die Autoren des Pakets
        \item Es werden die Verzeichnisse PyPI (Python) und CRAN (R) untersucht
        \item Für beide Verzeichnisse stehen APIs zur Verfügung, welche die Metadaten bereitstellen
    \end{itemize}
\end{frame}

% Sagen, dass in yaml geschrieben ist
% Auf das rechte eingehen (Unterschied given-names und name)
% Auf preferred citation eingehen -> Anerkennung für die Arbeit auf eine andere Arbeit übertragen. Kann die Wichtigkeit vernachlässigen, wenn nur das zitiert wird
\begin{frame}{Zitierformat – Citation File Format}
    \begin{columns}
        \begin{column}[t]{0.49\textwidth}
            \justifying
            \begin{itemize}
                \item Wird als Datei z.~B. in einem Git-Repository gespeichert
                \item Auf GitHub \href{https://github.com/citation-file-format}{\color{black}{\faGithub}} verwaltet
                \item 2.512 Repositorys auf GitHub haben eine CFF-Datei (Stand 07.11.2024)
                \item Dient dazu anderen die Zitation der Software zu erleichtern
                \item Kann unter anderem durch Programme wie \emph{cffinit} erstellt werden \autocite{spaaks_cffinit_2023}
                \item \emph{cffconvert} bietet die Möglichkeit der Validation und der Umwandlung z.~B. in \hologo{BibTeX} \autocite{spaaks_cffconvert_2021}
            \end{itemize}
        \end{column}
        \begin{column}[t]{0.49\textwidth}
            \justifying
            \inputminted[fontsize=\scriptsize]{yaml}{./CITATION_small.cff}
        \end{column}
    \end{columns}
\end{frame}

% Sagen wofür die später notwendig ist (Anhand des Bsp. sieht man es)
\begin{frame}{Named Entity Recognition}
    \begin{itemize}
        \item Beschreibt den Prozess der automatischen Erkennung und Klasseneinteilung von Substantiven (Entitäten) im Text
        \item Typische Entitäten sind Personen, Orte oder Organisationen
        \item Viele Anwendungsgebiete bspw. Informationsextraktion aus Texten
    \end{itemize}
    \begin{exampleblock}{Pytorch README}
        \emph{\textbf{PyTorch}} (Organisation) \emph{is currently maintained by \textbf{Soumith Chintala}} (Person)\emph{, \textbf{Gregory Chanan}} (Person)\emph{, \textbf{Dmytro Dzhulgakov}} (Person)\emph{, \textbf{Edward Yang}} (Person)\emph{, and \textbf{Nikita Shulga}} (Person) \emph{with major contributions coming from \textbf{hundreds}} (Digit) \emph{of talented individuals in various forms and means.}
    \end{exampleblock}
\end{frame}

% Sagen wofür die später notwendig ist (Anhand des bsp. sieht man es)
\begin{frame}{Unscharfe Suche}
    \begin{itemize}
        \item Findet ähnliche Zeichenfolgen, die sich in ihrer Schreibweise unterscheiden
        \item Als Ergebnis wird zumeist eine Distanz zwischen zwei Zeichenfolgen in Prozent angegeben
        \item Viele Anwendungsgebiete bspw. die Suche eines Namens in einem Index
    \end{itemize}
    \begin{exampleblock}{Unscharfe Suche}
        Soumith Chintala \leftrightarrow{} Soumith \textbf{S.} Chintala: 91 \% \\
        S\textbf{ou}mith Chintala \leftrightarrow{} S\textbf{uo}mith Chintala: 94 \%
    \end{exampleblock}
\end{frame}

\section{Methodik}

\begin{frame}{Datenbeschaffung}
    \begin{columns}
        \begin{column}[t]{0.39\textwidth}
            \begin{itemize}
                \item Datenbeschaffung für jeweils ein Paket (PyPI oder CRAN)
                \item Git, \hologo{BibTeX}, CFF und README werden zu Änderungszeitpunkten beschafft
                \item Beschreibung und README setzt NER ein
            \end{itemize}
        \end{column}
        \begin{column}[t]{0.59\textwidth}
            \begin{center}
                \includesvg[height=.79\textheight,inkscapelatex=false]{../docs/bilder/Ablauf.svg}
            \end{center}
        \end{column}
    \end{columns}
\end{frame}

% Übergang zur vorherigen Folie: Welche Pakete werden denn untersucht? -> Top 100 Listen
\begin{frame}{Top-100-Listen}
    \begin{columns}
        \begin{column}[t]{0.39\textwidth}
            \begin{itemize}
                \item 5 Top-100-Listen wurden erstellt und untersucht:
                \begin{enumerate}
                    \item Top-100 PyPI
                    \item Top-100 CRAN
                    \item Top-100 CFF
                    \item Top-100 PyPI CFF
                    \item Top-100 CRAN CFF
                \end{enumerate}
                \item Top-100 definiert an der Anzahl der Downloads (PyPI, CRAN) bzw. Sterne (CFF) auf GitHub
            \end{itemize}
        \end{column}
        \begin{column}[t]{0.59\textwidth}
            \begin{center}
                \includesvg[height=.79\textheight,inkscapelatex=false]{../docs/bilder/Listen.svg}
            \end{center}
        \end{column}
    \end{columns}
\end{frame}

\begin{frame}{Abgleich}
    \begin{itemize}
        \item Abgleich jeweils zwischen Git und einer weiteren Quelle
        \item Vereinfachung: Abgleich der Git-Autoren des aktuellen Stands mit den Quellen
        \item Abgleich verwendet Python \emph{in} und die unscharfe Suche
        \item Abgleich erfolgt anhand:
        \begin{itemize}
            \item Name
            \item E-Mail
            \item Benutzername
        \end{itemize}
        \item Für jeden Vergleich zweier Autoren wird ein Score berechnet
        \item Die Kombination mit dem besten Score wird ausgewählt
        \item Autoren mit mehr Commits werden bevorzugt
    \end{itemize}
\end{frame}

\section{Ergebnisse}

\begin{frame}{Abgleich}
    \begin{itemize}
        \item Ergebnisse sind abhängig vom Abgleich % Deswegen wird die Qualität des Abgleichs untersucht
        \item Manuelle Prüfung von jeweils 2 Autoren jeder Quelle jedes Pakets
        \item Einteilung in:
        \begin{itemize}
            \item \textcolor{links}{\textbf{TP}}: richtig als positiv klassifiziert
            \item \textcolor{links}{\textbf{FN}}: fälschlicherweise als negativ klassifiziert (obwohl positiv)
            \item \textcolor{links}{\textbf{FP}}: fälschlicherweise als positiv klassifiziert (obwohl negativ)
            \item \textcolor{links}{\textbf{TN}}: richtig als negativ klassifiziert
        \end{itemize}
        \item Berechnung des F1-Scores, ein Maß für die Genauigkeit eines Modells, welcher zwischen 0 und 1 liegt, wobei 1 die höchste Genauigkeit darstellt
        \item Angabe, ob es sich bei dem betrachteten Autor um keine Person handelt
    \end{itemize}
\end{frame}

\begin{frame}{Abgleich}
    \begin{tabularx}{\textwidth}{XR{.81cm}R{.81cm}R{.81cm}R{.81cm}R{2.8cm}R{1.9cm}}
        \toprule
        \textbf{Quelle} & \textbf{TP} & \textbf{FN} & \textbf{FP} & \textbf{TN} & \textbf{Keine Person} & \textbf{F1-Score}    \\ \midrule
        Beschreibung           & 87   & 5  & 18  & \textcolor{links}{125} & 15                     & \textcolor{links}{0,8832} \\
        README                 & 81   & 21 & 11  & \textcolor{links}{143} & 26                     & \textcolor{links}{0,8351} \\
        CFF                    & 168  & 13 & 3   & 24                     & 3                      & 0,9545                    \\
        CFF preferred citation & 78   & 8  & 2   & 13                     & 0                      & 0,9398                    \\
        PyPI Maintainer        & 294  & 2  & 32  & 48                     & \textcolor{links}{50}  & 0,9453                    \\
        Python Autoren         & 155  & 4  & 35  & 43                     & \textcolor{links}{75}  & 0,8883                    \\
        Python Maintainer      & 30   & 0  & 5   & 21                     & 25                     & 0,9231                    \\
        CRAN Autoren           & 269  & 4  & 3   & 19                     & 1                      & 0,9872                    \\
        CRAN Maintainer        & 193  & 2  & 0   & 0                      & 0                      & 0,9948                    \\
        \hologo{BibTeX}        & 4    & 2  & 0   & 0                      & 0                      & 0,8000                    \\ \midrule
        Summe                  & 1359 & 61 & 109 & 436                    & 195                    & \textcolor{links}{0,9411} \\
        \bottomrule
    \end{tabularx}
\end{frame}

% Linien im Hintergrund erklären/ Gilt auch für die nachfolgenden Abbildungen
% Bei Betrachtung immer mehr Autoren sinkt der Anteil derer die genannt werden
% Einige Linien beginnen erst bei x > 1 -> der Top Autor wird in diesen Fällen nicht genannt
% Einige Linien laufen bei 100 % -> Bei diesen werden alle Git Autoren in der Quelle genannt (meist bei wenigen Git Autoren)
\begin{frame}{Top-Git-Autoren an genannten Autoren}
    \begin{columns}
        \begin{column}[t]{0.29\textwidth}
            \begin{itemize}
                \item Anteil der Top-Git-Autoren an den genannten Autoren
                \item Betrachtet PyPI CFF Liste
                \item Für CFF-Linie gilt: Die Top \emph{x} Commiter sind zu \emph{y} \% in der CFF gelistet
            \end{itemize}
        \end{column}
        \begin{column}[t]{0.69\textwidth}
            \begin{center}
                \includesvg[height=.79\textheight,inkscapelatex=false]{../docs/bilder/common_authors/1_pypi_cff.svg}
            \end{center}
        \end{column}
    \end{columns}
\end{frame}

% Bei Betrachtung immer mehr Autoren steigt der Anteil derer, die Code beigetragen haben
% Einige Linien beginnen erst bei x > 1 -> kein Autor in der Quelle ist unter den x Top-Git-Autoren
% Der Graph kann 100% erreichen, falls alle genannten Autoren tatsächlich code beigetragen haben
\begin{frame}{Genannte Autoren unter Top-Git-Autoren}
    \begin{columns}
        \begin{column}[t]{0.29\textwidth}
            \begin{itemize}
                \item Anteil der genannten Autoren unter den Top-Git-Autoren
                \item Betrachtet PyPI CFF Liste
                \item Für CFF-Linie gilt: Autoren in der CFF sind zu \emph{y} \% unter den Top \emph{x} Commitern
            \end{itemize}
        \end{column}
        \begin{column}[t]{0.69\textwidth}
            \begin{center}
                \includesvg[height=.79\textheight,inkscapelatex=false]{../docs/bilder/common_authors_2/2_pypi_cff.svg}
            \end{center}
        \end{column}
    \end{columns}
\end{frame}

% 2017 die ersten CFF dateien
% Mitte 2021 steigt die Anzahl der CFF Dateien schnell
% zwar mehr valide als invalide cff aber dennoch viel invalide
% CFF init wird kaum verwendet
\begin{frame}{Validität der CFF-Dateien}
    \begin{columns}
        \begin{column}[t]{0.29\textwidth}
            \begin{itemize}
                \item Validität der CFF-Dateien über die Zeit
                \item Betrachtet alle CFF-Dateien auf GitHub
                \item Jedes Repository nur einmalig betrachtet
            \end{itemize}
        \end{column}
        \begin{column}[t]{0.69\textwidth}
            \begin{center}
                \includesvg[height=.79\textheight,inkscapelatex=false]{../docs/bilder/valid_cff_by_time/overall_valid_cff_alle_cff.svg}
            \end{center}
        \end{column}
    \end{columns}
\end{frame}

\section{Diskussion}

\begin{frame}{Wie gut können Autoren untereinander abgeglichen werden?}
    \begin{itemize}
        \item Viele Autoren in Python-Quellen sind keine Personen, sondern Organisationen
        \item Diese verschlechtern die Ergebnisse, da sie falsch zugeordnet werden können
        \item README und Beschreibung haben schlechte F1-Scores, was in der NER begründet liegt, da diese bereits Fehler macht
        \item Viele FP in den Ergebnissen, durch die Verwendung von Python \emph{in} und Autoren mit wenigen Buchstaben in Git
        \item Im Allgemeinen funktioniert der Abgleich gut, da ein F1-Score von 0,9411 berechnet wurde, welcher größer als 0,9 ist
        \item F1-Score könnte noch besser sein, ohne die genannten Gegebenheiten
    \end{itemize}
\end{frame}

\begin{frame}{Was muss ein Softwareentwickler leisten, um als Autor genannt zu werden?}
    \begin{itemize}
        \item Frage lässt sich nur allgemeingültig beantworten, da Daten allgemeingültig untersucht wurden
        \item Autoren mit vielen Commits werden häufiger als Autor genannt \rightarrow{} viele Commits haben
        \item Allerdings: Autoren mit den meisten Commits werden auch manchmal nicht genannt
        \item Bei der Gründung des Pakets dabei sein, allerdings im Nachhinein nicht umsetzbar
        \item Autoren werden nachträglich selten hinzugefügt % Darauf geht die nächste Folie weiter ein
    \end{itemize}
\end{frame}

\begin{frame}{Wie gut werden Autoren in den einzelnen Quellen gepflegt?}
    \begin{itemize}
        \item In allen Paketen der 5 Listen nur 9 Autoren nachträglich dem CFF oder der \hologo{BibTeX}-Datei hinzugefügt
        \item Hohe Anzahl der invaliden CFF zeigt, dass die Pflege der Autoren nicht besonders wichtig ist
        \item Allgemein: Autoren in den betrachteten Listen werden nicht aktiv gepflegt und keine automatischen Prozesse vorhanden
        \item Begründung: Viele Pakete nicht in der Wissenschaft entstanden, sondern bspw. in Unternehmen % Bei diesen steht die Nennung einzelner Autoren nicht im Fokus, sondern die Nennung des Unternehmens
    \end{itemize}
\end{frame}

\begin{frame}{Fazit}
    \begin{itemize}
        \item Diskrepanzen bspw. bei der Nennung der Top-Git-Autoren in den Datenquellen
        \item Ein Abgleich von Autoren mit wenigen Datenpunkten (Name, E-Mail, Benutzername) ist möglich (\textcolor{links}{\textbf{F1}})
        \item Um als Autor genannt zu werden, ist es gut viele Commits getätigt zu haben, allerdings ist eine Nennung nicht garantiert (\textcolor{links}{\textbf{F2}})
        \item Autoren werden kaum gepflegt (\textcolor{links}{\textbf{F3}})
        \item Verbesserungsbedarf bei der Pflege und Nennung von Autoren in OSS
        \item Arbeit eines jeden Autors sollte angemessen gewürdigt werden
    \end{itemize}
\end{frame}

\begin{frame}{Literaturverzeichnis}
    \printbibliography
\end{frame}

\end{document}
